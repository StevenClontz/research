\documentclass{amsart}
\usepackage{amsmath}
\usepackage{amsthm}
\usepackage{amssymb}

\usepackage{tikz}
\usetikzlibrary{arrows}

\usepackage{../../clontzDefinitions}

\renewcommand{\vec}{\mathbf}

      \theoremstyle{plain}
      \newtheorem{theorem}{Theorem}
      \newtheorem{lemma}[theorem]{Lemma}
      \newtheorem{corollary}[theorem]{Corollary}
      \newtheorem{proposition}[theorem]{Proposition}
      \newtheorem{conjecture}[theorem]{Conjecture}
      \newtheorem{question}[theorem]{Question}
      \newtheorem{claim}[theorem]{Claim}

      \theoremstyle{definition}
      \newtheorem{definition}[theorem]{Definition}
      \newtheorem{notation}[theorem]{Notation}
      \newtheorem{example}[theorem]{Example}
      \newtheorem{game}[theorem]{Game}

      \theoremstyle{remark}
      \newtheorem{remark}[theorem]{Remark}

      \theoremstyle{plain}
      \newtheorem*{theorem*}{Theorem}
      \newtheorem*{lemma*}{Lemma}
      \newtheorem*{corollary*}{Corollary}
      \newtheorem*{proposition*}{Proposition}
      \newtheorem*{conjecture*}{Conjecture}
      \newtheorem*{question*}{Question}
      \newtheorem*{claim*}{Claim}

      \theoremstyle{definition}
      \newtheorem*{definition*}{Definition}
      \newtheorem*{example*}{Example}
      \newtheorem*{game*}{Game}

      \theoremstyle{remark}
      \newtheorem*{remark*}{Remark}




\begin{document}

\title{Dual selection games}



\author{Steven Clontz}
\address{Department of Mathematics and Statistics,
The University of South Alabama,
Mobile, AL 36688}
\email{sclontz@southalabama.edu}



\keywords{Selection principle, selection game,
limited information strategies}

\subjclass[2010]{54C30, 54D20, 54D45, 91A44}






\begin{abstract}
  (an investigation of dual selection games)
\end{abstract}


\maketitle







\section{Introduction}

\begin{definition}
  The \term{selection game} \(\schStrongSelGame{\mc A}{\mc B}\) 
  is an \(\omega\)-length game involving Players \(\plI\) and \(\plII\). 
  During round \(n\), \(\plI\) chooses
  \(A_n\in\mc A\), followed by \(\plII\) choosing \(B_n\in A_n\).
  Player \(\plII\) wins in the case that \(\{B_n:n<\omega\}\in\mc B\),
  and Player \(\plI\) wins otherwise.
\end{definition}

  For brevity, let 
  \[
    \schStrongSelGame{\mc A}{\neg \mc B}
      =
    \schStrongSelGame{\mc A}{\mc P\left(\bigcup \mc A\right)\setminus \mc B}
  .\]
  That is, \(\plII\) wins in the case that \(\{B_n:n<\omega\}\not\in\mc B\),
  and \(\plI\) wins otherwise.

\begin{definition}
  For a set \(X\), let \(\mathbf C(X)\) be the collection of all
  choice functions on \(X\), functions \(f:X\to\bigcup X\) 
  such that \(f(x)\in x\) for all \(x\in X\).
\end{definition}

\begin{definition}
  The set \(\mc A'\) is said to be a \term{reflection} of the set \(\mc A\)
  if \[\{\ran f:f\in\mathbf C(\mc A')\}=\mc A.\]
\end{definition}

  For example, a reflection of the collection \(\mc O_X\) of basic open covers
  of \(X\) would be \(\mc P_X=\{\mc T_{X,x}:x\in X\}\), where \(\mc T_{X,x}\) 
  is the corresponding point-base at \(x\in X\). 
  Likewise for the collection \(\Omega_{X,x}\)
  of sets with \(x\in X\) as a limit point, \(\mc T_{X,x}\) is itself
  a reflection.

\begin{theorem}
  Let \(\mc A'\) be a reflection of \(\mc A\). 

  Then
  \(\plI\prewin\schStrongSelGame{\mc A}{\mc B}\) if and only if
  \(\plII\markwin\schStrongSelGame{\mc A'}{\neg\mc B}\).
\end{theorem}

\begin{proof}
  Let \(\sigma\) witness 
  \(\plI\prewin\schStrongSelGame{\mc A}{\mc B}\).
  Since \(\sigma(n)\in\mc A\), \(\sigma(n)=\ran{f_n}\)
  for some \(f_n\in\mathbf C(\mc A')\). So let
  \(\tau(A,n)=f_n(A)\) for all \(A\in \mc A'\) and \(n<\omega\).
  Suppose \(A_n\in \mc A'\) for all \(n<\omega\).
  Note that since \(\sigma\) is winning and 
  \(\tau(A_n,n)=f_n(A_n)\in\ran{f_n}=\sigma(n)\),
  \(\{\tau(A_n,n):n<\omega\}\not\in\mc B\). Thus \(\tau\) witnesses
  \(\plII\markwin\schStrongSelGame{\mc A'}{\neg\mc B}\).

  Now let \(\sigma\) witness
  \(\plII\markwin\schStrongSelGame{\mc A'}{\neg\mc B}\).
  Let \(f_n\in\mathbf C(\mc A')\) be defined by \(f_n(A)=\sigma(A,n)\).
  Since \(\mc A=\{\ran f:f\in\mathbf C(\mc A')\}\), let
  \(\tau(n)=\ran{f_n}\). Suppose that \(B_n\in\tau(n)=\ran{f_n}\) for
  all \(n<\omega\). Choose \(A_n\in\mc A'\) such that 
  \(B_n=f_n(A_n)=\sigma(A_n,n)\). Since \(\sigma\) is winning,
  \(\{B_n:n<\omega\}\not\in\mc B\). Thus \(\tau\) witnesses
  \(\plI\prewin\schStrongSelGame{\mc A}{\mc B}\).
\end{proof}

\bibliographystyle{plain}
\bibliography{../../bibliography}

\end{document}
