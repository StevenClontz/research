\documentclass{amsart}
\usepackage{amsmath}
\usepackage{amsthm}
\usepackage{amssymb}

\usepackage{tikz}
\usetikzlibrary{matrix}

\usepackage{../../clontzDefinitions}


      \theoremstyle{plain}
      \newtheorem{theorem}{Theorem}
      \newtheorem{lemma}[theorem]{Lemma}
      \newtheorem{corollary}[theorem]{Corollary}
      \newtheorem{proposition}[theorem]{Proposition}
      \newtheorem{conjecture}[theorem]{Conjecture}
      \newtheorem{question}[theorem]{Question}
      \newtheorem{claim}[theorem]{Claim}

      \theoremstyle{definition}
      \newtheorem{definition}[theorem]{Definition}
      \newtheorem{example}[theorem]{Example}
      \newtheorem{game}[theorem]{Game}

      \theoremstyle{remark}
      \newtheorem{remark}[theorem]{Remark}

      \theoremstyle{plain}
      \newtheorem*{theorem*}{Theorem}
      \newtheorem*{lemma*}{Lemma}
      \newtheorem*{corollary*}{Corollary}
      \newtheorem*{proposition*}{Proposition}
      \newtheorem*{conjecture*}{Conjecture}
      \newtheorem*{question*}{Question}
      \newtheorem*{claim*}{Claim}

      \theoremstyle{definition}
      \newtheorem*{definition*}{Definition}
      \newtheorem*{example*}{Example}
      \newtheorem*{game*}{Game}

      \theoremstyle{remark}
      \newtheorem*{remark*}{Remark}

\parskip=0.5em




\begin{document}

% \title{Proximal compact spaces are Corson compact\tnoteref{t1}}
% \tnotetext[t1]{2010 Mathematics Subject Classification. 54E15, 54D30, 54A20.}
\title{Lexicographic products and idempotent inverse limits}

% \author[aub]{S.~Clontz\fnref{fn1}}
% \ead{steven.clontz@auburn.edu}
% \author[aub]{G.~Gruenhage\fnref{fn2}}
% \ead{gruengf@auburn.edu}

% \address[aub]{Department of Mathematics, Auburn University,
%  Auburn, AL 36830}


\author{Steven Clontz}
\address{Department of Mathematics and Statistics, The University of South Alabama}
\email{steven.clontz@gmail.com}

% \keywords{TODO}
% \subjclass[2010]{}






\begin{abstract}
  The inverse limit \(\varprojlim\{X,\gamma,L\}\) may
  be characterized as the quotient of a generalized lexicogrphic product of
  the compactification \(\hat L\) and \(X\). A special case of
  this fact is used to show that when \(f\) satisfies condition \(\Gamma\)
  and \(L\) is uncountable,
  the inverse limit \(\varprojlim\{X,f,L\}\) cannot be Corson compact,
  and therefore cannot be metrizable.
\end{abstract}


\maketitle






\section{Introduction}

All topological spaces are assumed to be \(T_1\).

\begin{definition}
  Let \(L, M\) be total orders. Define their \term{lexicographic product}
  \(L\lexTimes M\) by the total ordering \(\<l,m\><\<l',m'\>\) if and
  only if \(l<l'\) or both \(l=l'\) and \(m<m'\).
\end{definition}

\begin{definition}
  Let \(L\) be a total order, and let \(X\) be a topological space with
  two distinguished points \(0,1\). Define their \term{generalized lexicographic
  product} \(L\glTimes X\) to have the topology generated by the following
  neighborhood bases:
  \begin{itemize}
    \item For \(l\in L\) and \(t\in X\setminus\{0,1\}\), \(\<l,t\>\) has
      neighborhoods of the form \(\{l\}\times U\setminus\{0,1\}\) for all
      neighborhoods \(U\) of \(t\) in \(X\).
    \item For non-minimal \(l\in L\), \(\<l,0\>\) has neighborhoods of the form
    \(
      (\{l^-\}\times V\setminus\{0\})
        \cup
      ((l^-,l)\times X)
        \cup
      (\{l\}\times U\setminus \{1\})
    \)
      such that \(l^-<l\), \(U\) is a neighborhood of \(0\) in \(X\), and \(V\)
      is either empty or a neighborhood of \(1\) in \(X\).
    \item For \(l=\min L\), \(\<l,0\>\) has neighborhoods of the form
    \(
      (\{l\}\times U\setminus \{1\})
    \)
      such that \(U\) is a neighborhood of \(0\) in \(X\).
    \item For non-maximal \(l\in L\), \(\<l,1\>\) has neighborhoods of the form
    \(
      (\{l\}\times V\setminus\{0\})
        \cup
      ((l,l^+)\times X)
        \cup
      (\{l^+\}\times U\setminus \{1\})
    \)
      such that \(l<l^+\), \(U\) is either empty or a neighborhood of \(0\) in
      \(X\), and \(V\) is a neighborhood of \(1\) in \(X\).
    \item For \(l=\max L\), \(\<l,1\>\) has neighborhoods of the form
    \(
      (\{l\}\times V\setminus \{0\})
    \)
      such that \(V\) is a neighborhood of \(1\) in \(X\)
  \end{itemize}
\end{definition}

\begin{proposition}
  Let \(L,M\) be LOTS such that \(M\) has distinguished points \(0=\min M\)
  and \(1=\max M\). Then \(L\glTimes M \cong L\lexTimes M\).
\end{proposition}

\begin{definition}
  Let \(X\) be a topological space with
  two distinguished points \(0,1\). Define \(\gamma\subseteq X^2\) to be the
  idempotent bonding relation defined by \(\gamma(0)=X\) and
  \(\gamma(t)=\{1\}\) for all \(t\in X\setminus\{0\}\).
\end{definition}

\begin{theorem}
  Let \(L\) be a total order, and let \(X\) be a topological space with
  two distinguished points \(0,1\). Then
  \(\inverseLimit{X,\gamma,L}=(\hat L\glTimes X)/\sim\), where the
  equivalence relation \(\sim\) is given by
  \begin{itemize}
    \item \(\<A,t\>\sim\<A,1\>\) for all \(A\in\hat L\setminus \dot L\).
    \item \(\<A,1\>\sim\<B,0\>\) whenever \(|B\setminus A|=1\).
  \end{itemize}
\end{theorem}

\begin{proof}
  We first observe that for \(\vec x\in\inverseLimit{X,\gamma,L}\) there
  exists a maximal left-closed set
  \(A_{\vec x}\in\hat L\) such that \(\vec x[\interior A_{\vec x}]=\{1\}\)
  and \(\vec x[\exterior A_{\vec x}]=\{0\}\). Note that if \(A_{\vec x}\) has
  a maximum element, i.e. \(A_{\vec x}\in \dot L\),
  then \(\vec x(\max A_{\vec x})\) may be valued at almost any point of \(X\).
  The lone exception occurs when \(\max A_{\vec x}\) has a successor in \(L\);
  \(\vec x(\max A_{\vec x})=1\) would then violate the maximality of
  \(A_{\vec x}\).

  Define \(\theta:\inverseLimit{X,\gamma,L}\to(\hat L\glTimes X)/\sim\)
  by \(\theta(\vec x)=\<A_{\vec x},1\>\) whenever
  \(A_{\vec x}\in\hat L\setminus \dot L\), and
  \(\theta(\vec x)=\<A_{\vec x},\vec x(\max A_{\vec x})\>\) otherwise.
  It's then evident from our construction of \(A_{\vec x}\)
  that \(\theta\) is a bijection.

  Let \(B=(\leftarrow,l]\in\dot L\). If \(t\in X\setminus\{0,1\}\), note
  \(\<B,t\>\) has a base of neighborhoods of
  the form \(\{B\}\times U\setminus\{0,1\}\) which is equal to
  \(\theta[\pi_l^{-1}[U\setminus\{0,1\}]]\).

  Suppose \(|B\setminus A|=1\) for \(A=(\leftarrow,k]\in\dot L\).
  Then \(\<B,0\>\) has a base of neighborhoods of the form
  \(
    (\{A\}\times V\setminus\{0\})
      \cup
    (\{B\}\times U\setminus \{1\})
  \), which is equal to
  \(\theta[\pi_k^{-1}[V\setminus\{0\}]]\cap\theta[\pi_l^{-1}[U\setminus\{1\}]]\).


\end{proof}




\bibliographystyle{plain}
\bibliography{../../bibliography}

\end{document}
