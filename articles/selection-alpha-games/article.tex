\documentclass{amsart}
\usepackage{amsmath}
\usepackage{amsthm}
\usepackage{amssymb}

\usepackage{tikz}
\usetikzlibrary{arrows}

\usepackage{../../clontzDefinitions}

\renewcommand{\vec}{\mathbf}

      \theoremstyle{plain}
      \newtheorem{theorem}{Theorem}
      \newtheorem{lemma}[theorem]{Lemma}
      \newtheorem{corollary}[theorem]{Corollary}
      \newtheorem{proposition}[theorem]{Proposition}
      \newtheorem{conjecture}[theorem]{Conjecture}
      \newtheorem{question}[theorem]{Question}
      \newtheorem{claim}[theorem]{Claim}

      \theoremstyle{definition}
      \newtheorem{definition}[theorem]{Definition}
      \newtheorem{notation}[theorem]{Notation}
      \newtheorem{example}[theorem]{Example}
      \newtheorem{game}[theorem]{Game}

      \theoremstyle{remark}
      \newtheorem{remark}[theorem]{Remark}

      \theoremstyle{plain}
      \newtheorem*{theorem*}{Theorem}
      \newtheorem*{lemma*}{Lemma}
      \newtheorem*{corollary*}{Corollary}
      \newtheorem*{proposition*}{Proposition}
      \newtheorem*{conjecture*}{Conjecture}
      \newtheorem*{question*}{Question}
      \newtheorem*{claim*}{Claim}

      \theoremstyle{definition}
      \newtheorem*{definition*}{Definition}
      \newtheorem*{example*}{Example}
      \newtheorem*{game*}{Game}

      \theoremstyle{remark}
      \newtheorem*{remark*}{Remark}


\usepackage{lineno}
\linenumbers



\begin{document}

\title{Selection games and Arhangelskii's convergence principles}



\author{Steven Clontz}
\address{Department of Mathematics and Statistics,
The University of South Alabama,
Mobile, AL 36688}
\email{sclontz@southalabama.edu}



\keywords{Selection principle, selection game,
\(\alpha_i\) property, convergence}

%\subjclass[2010]{54D20, 54D45, 91A44}



\begin{abstract}
We prove the things.
\end{abstract}


\maketitle






%
%\section{Introduction}
%
%\begin{definition}
%  The \term{selection game} \(\schStrongSelGame{\mc A}{\mc B}\) 
%  is an \(\omega\)-length game involving Players \(\plI\) and \(\plII\). 
%  During round \(n\), \(\plI\) chooses
%  \(A_n\in\mc A\), followed by \(\plII\) choosing \(B_n\in A_n\).
%  Player \(\plII\) wins in the case that \(\{B_n:n<\omega\}\in\mc B\),
%  and Player \(\plI\) wins otherwise.
%\end{definition}
%
%\begin{definition}
%  The \term{selection game} \(\schSelGame{\mc A}{\mc B}\) 
%  is an \(\omega\)-length game involving Players \(\plI\) and \(\plII\). 
%  During round \(n\), \(\plI\) chooses
%  \(A_n\in\mc A\), followed by \(\plII\) choosing \(B_n\in [A_n]^{<\aleph_0}\).
%  Player \(\plII\) wins in the case that \(\bigcup\{B_n:n<\omega\}\in\mc B\),
%  and Player \(\plI\) wins otherwise.
%\end{definition}
%
\section{Clontz results}

\begin{definition}
Say a collection \(\mc A\) is \term{sequence-like} if it satisfies the following
for each \(A\in\mc A\).
\begin{itemize}
\item \(|A|\geq\aleph_0\).
\item If \(A'\subseteq A\) and \(|A'|\geq\aleph_0\), then \(A'\in\mc A\).
\end{itemize}
\end{definition}

\begin{definition}
Let \(\Gamma_X\) be the collection of open \term{\(\gamma\)-covers} \(\mc U\) of \(X\),
that is, infinite open covers of \(X\) such that for each \(x\in X\),
\(\{U\in\mc U:x\in U\}\) is cofinite in \(\mc U\).
\end{definition}

\begin{definition}
Let \(\Gamma_{X,x}\) be the collection of non-trivial sequences \(S\subseteq X\) converging to \(x\),
that is, infinite subsets of \(X\) such that for each neighborhood \(U\) of \(x\),
\(S\cap U\) is cofinite in \(S\).
\end{definition}

It follows that \(\Gamma_X,\Gamma_{X,x}\) are both sequence-like.

\begin{theorem}
Let \(\mc B\) be sequence-like. Then \(\alpha_1(\mc A,\mc B)\) holds if and only
if \(\plI\notprewin G_{cf}(\mc A,\mc B)\).
\end{theorem}

\begin{proof}
We first assume \(\alpha_1(\mc A,\mc B)\) and let \(A_n\in\mc A\) for \(n<\omega\)
define a predetermined strategy for \(\plI\). By \(\alpha_1(\mc A,\mc B)\), we
immediately obtain \(B\in\mc B\) such that \(|A_n\setminus B|<\aleph_0\). Thus
\(B_n=A_n\cap B\) is a cofinite choice from \(A_n\), and 
\(B'=\bigcup\{B_n:n<\omega\}\) is an infinite subset of \(B\),
so \(B'\in\mc B\). Thus \(\plII\) may defeat \(\plI\) by choosing
\(B_n\subseteq A_n\) each round, witnessing \(\plI\notprewin G_{cf}(\mc A,\mc B)\).

On the other hand, let \(\plI\notprewin G_{cf}(\mc A,\mc B)\). Given \(A_n\in\mc A\)
for \(n<\omega\), we note that \(\plII\) may choose a cofinite subset \(B_n\subseteq A_n\)
such that \(B=\bigcup\{B_n:n<\omega\}\in\mc B\). Then \(B\) witnesses \(\alpha_1(\mc A,\mc B)\)
since \(|A_n\setminus B|\leq|A_n\setminus B_n|\leq\aleph_0\).
\end{proof}

\begin{theorem}
Let \(\mc A,\mc B\) be sequence-like. Then \(\alpha_2(\mc A,\mc B)\) holds if and only
if \(\plI\notprewin G_1(\mc A,\mc B)\).
\end{theorem}

\begin{proof}
We first assume \(\alpha_2(\mc A,\mc B)\) and let \(A_n\in\mc A\) for \(n<\omega\)
define a predetermined strategy for \(\pl I\).
We may apply \(\alpha_2(\mc A,\mc B)\) to choose \(B\in\mc B\) such that
\(|A_n\cap B|\geq\aleph_0\). We may then choose \(a_n\in(A_n\cap B)\setminus\{a_i:i<n\}\)
for each \(n<\omega\). It follows that \(B'=\{a_n:n<\omega\}\in\mc B\) since
\(B'\) is an infinite subset of \(B\in\mc B\); therefore \(A_n\) does not define
a winning predetermined strategy for \(\plI\).

Now suppose \(\plI\notprewin G_1(\mc A,\mc B)\). Given \(A_n\in\mc A\) for \(n<\omega\),
first choose \(A_n'=\{a_{n,j}:j<\omega\}\subseteq A_n\) such that \(j<k\) implies
\(a_{n,j}\not=a_{n,k}\),
and then let \(A_{n,m}=\{a_{n,j}:m\leq j<\omega\}\), noting \(A_{n,m}\in\mc A\) since
\(A_{n,m}\) is an infinite subset of \(A_n\in\mc A\). Finally choose
some \(\theta:\omega\to\omega\) such that \(|\theta^{\leftarrow}(n)|=\aleph_0\) for
each \(n<\omega\).

Since playing \(A_{\theta(m),m}\) during round \(m\)
does not define a winning strategy for \(\plI\) in
\(G_1(\mc A,\mc B)\), \(\plII\) may choose \(x_m\in A_{\theta(m),m}\)
such that \(B=\{x_m:m<\omega\}\in\mc B\).
Choose \(i_m<\omega\) for each \(m<\omega\) such that
\(x_m=a_{\theta(m),i_m}\), noting \(i_m\geq m\).
It follows that 
\(A_n\cap B\supseteq\{a_{\theta(m),i_m}:m\in\theta^{\leftarrow}(n)\}\).
Since for each \(m\in\theta^{\leftarrow}(n)\) there exists
\(M\in\theta^{\leftarrow}(n)\) such that \(m\leq i_m<M\leq i_{M}\),
and therefore \(a_{\theta(m),i_m}\not=a_{\theta(m),i_{M}}=a_{\theta(M),i_{M}}\),
we have shown that \(A_n\cap B\) is infinite. Thus \(B\) witnesses
\(\alpha_2(\mc A,\mc B)\).
\end{proof}

\begin{theorem}
Let \(\mc A,\mc B\) be sequence-like. Then \(\alpha_4(\mc A,\mc B)\) holds if and only
if \(\plI\notprewin G_{<2}(\mc A,\mc B)\) if and only if
\(\plI\notprewin G_{fin}(\mc A,\mc B)\).
\end{theorem}

\begin{proof}
We first assume \(\alpha_4(\mc A,\mc B)\) and let \(A_n\in\mc A\) for \(n<\omega\)
define a predetermined strategy for \(\plI\) in 
\(G_{<2}(\mc A,\mc B)\). We then choose
\(A_n'=\{a_{n,j}:j<\omega\}\subseteq A_n\) such that \(j<k\) implies
\(a_{n,j}\not=a_{n,k}\), and then let \(A_n''=A_n'\setminus\{a_{i,j}:i,j<n\}\),
noting \(A_n''\in\mc A\) since it is an infinite subset of \(A_n\).

By applying \(\alpha_4(\mc A,\mc B)\) to \(A_n''\), we obtain \(B\in\mc B\)
such that \(A_n''\cap B\not=\emptyset\) for infintely-many \(n<\omega\).
We then let \(F_n=\emptyset\) when \(A_n''\cap B=\emptyset\), and
\(F_n=\{x_n\}\) for some \(x_n\in A_n''\cap B\) otherwise. Then we will have that
\(B'=\bigcup\{F_n:n<\omega\}\subseteq B\) belongs to \(\mc B\) once we show that
\(B'\) is infinite. To see this, for \(m\leq n<\omega\) note that either \(F_m\) is
empty (and we let \(j_m=0\)) or \(F_m=\{a_{m,j_m}\}\)
for some \(j_m\geq m\); choose \(N<\omega\) such that \(j_m<N\) for all
\(m\leq n\) and \(F_N=\{x_N\}\). Thus \(F_m\not=F_N\) for all \(m\leq n\) since
\(x_{N}\not\in\{a_{i,j}:i,j< N\}\). Thus \(\plII\) may defeat the predetermined
strategy \(A_n\) by playing \(F_n\) each round.

Since \(\plI\notprewin G_{<2}(\mc A,\mc B)\) immediately implies
\(\plI\notprewin G_{fin}(\mc A,\mc B)\), we assume the latter. Given \(A_n\in\mc A\)
for \(n<\omega\), we note this defines a (non-winning) predetermined 
strategy for \(\plI\), so \(\plII\) may choose \(F_n\in[A_n]^{<\aleph_0}\) such that
\(B=\bigcup\{F_n:n<\omega\}\in\mc B\). Since \(B\) is infinite, we note
\(F_n\not=\emptyset\) for infinitely-many \(n<\omega\). Thus \(B\) witnesses
\(\alpha_4(\mc A,\mc B)\) since \(A_n\cap B\supseteq F_n\not=\emptyset\) for
infinitely-many \(n<\omega\).
\end{proof}


\bibliographystyle{plain}
\bibliography{../../bibliography}

\end{document}
