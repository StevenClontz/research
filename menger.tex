\documentclass[11pt]{article}

\pdfpagewidth 8.5in
\pdfpageheight 11in

\setlength\topmargin{0in}
\setlength\headheight{0in}
\setlength\headsep{0.2in}
\setlength\textheight{8in}
\setlength\textwidth{6in}
\setlength\oddsidemargin{0in}
\setlength\evensidemargin{0in}
\setlength\parindent{0.25in}
\setlength\parskip{0.1in} 
 
\usepackage{amssymb}
\usepackage{amsfonts}
\usepackage{amsmath}
\usepackage{mathtools}
\usepackage{amsthm}

\usepackage{enumerate}

      \theoremstyle{plain}
      \newtheorem{theorem}{Theorem}
      \newtheorem{lemma}[theorem]{Lemma}
      \newtheorem{corollary}[theorem]{Corollary}
      \newtheorem{proposition}[theorem]{Proposition}
      \newtheorem{conjecture}[theorem]{Conjecture}
      \newtheorem{question}[theorem]{Question}
      \newtheorem{example}[theorem]{Example}
      
      \theoremstyle{definition}
      \newtheorem{definition}[theorem]{Definition}
      
      \theoremstyle{remark}
      \newtheorem{remark}[theorem]{Remark}


% Strategy uparrow shortcuts
\newcommand{\win}{\uparrow}
\newcommand{\prewin}{\uparrow_{\text{pre}}}
\newcommand{\markwin}{\uparrow_{\text{mark}}}
\newcommand{\tactwin}{\uparrow_{\text{tact}}}
\newcommand{\ktactwin}[1]{\uparrow_{#1\text{-tact}}}
\newcommand{\kmarkwin}[1]{\uparrow_{#1\text{-mark}}}
\newcommand{\codewin}{\uparrow_{\text{code}}}
\newcommand{\limitwin}{\uparrow_{\text{limit}}}

\newcommand{\oneptcomp}[1]{#1\cup\{\infty\}}

\newcommand{\congame}[2]{Con_{O,P}(#1,#2)}
\newcommand{\clusgame}[2]{Clus_{O,P}(#1,#2)}

\newcommand{\lfkpgame}[1]{LF_{K,P}(#1)}
\newcommand{\lfklgame}[1]{LF_{K,L}(#1)}

\newcommand{\pfgame}[1]{PF_{F,C}(#1)}

\newcommand{\mengame}[1]{Cov_{C,S}(#1)}

\newcommand{\sigmaprodr}[1]{\Sigma\mathbb{R}^{#1}}
\newcommand{\sigmaprodtwo}[1]{\Sigma2^{#1}}

\newcommand{\concat}{^\frown}
\newcommand{\rest}{\restriction}

\newcommand{\close}[1]{\overline{#1}}

\newcommand{\<}{\langle}
\renewcommand{\>}{\rangle}

\newcommand{\mc}[1]{\mathcal{#1}}

\newcommand{\ds}{\displaystyle}

\begin{document}

  \begin{definition}
    $X$ is \textbf{Menger} if for all open covers $\mathcal{U}_0,\mathcal{U}_1,\dots$ there exist finite subcollections $\mathcal{F}_n \subseteq \mathcal{U}_n$ such that $\bigcup_{n<\omega} \mathcal{F}_n$ is a cover of $X$.
  \end{definition}

  \begin{proposition}
    $\sigma$-compact $\Rightarrow$ Menger $\Rightarrow$ Lindelof
  \end{proposition}

  \begin{definition}
    In the two-player game $\mengame{X}$ player $C$ chooses open covers $\mathcal{U}_n$ of $X$, followed by player $S$ choosing a finite subcollection $\mathcal{F}_n\subseteq\mathcal{U}_n$. $S$ wins if $\bigcup_{n<\omega} \mathcal{F}_n$ is a cover of $X$.
  \end{definition}

  \begin{theorem}
    $X$ is Menger if and only if $S \win \mengame{X}$.
  \end{theorem}

  \begin{proof}
    First, suppose $X$ wasn't Menger. Then there would exist open covers $\mathcal{U}_0,\mathcal{U}_1,\dots$ of $X$ such that for any choice of finite subcollections $\mathcal{F}_n\subseteq\mathcal{U}_n$, $\bigcup_{n<\omega} \mathcal{F}_n$ isn't a cover of $X$. Thus $C\prewin\mengame{X} \Rightarrow S \not\win\mengame{X}$.

    The other direction is based upon Gruenhage's topological game presentation. Assume $X$ is Menger, and consider a strategy for $C$ in $\mengame{X}$.

    Since $X$ is Lindelof, we can assume $C$ plays only countable covers of $X$. Then, since $S$ is choosing finite subsets, we may assume $S$ chooses some initial segement of the countable cover. In turn, we can assume $C$ plays an increasing open cover $\{U_0,U_1,\dots\}$ where $U_n\subseteq U_{n+1}$. And in that case, it's suffient to assume $S$ simply chooses a singleton subset of each cover. And finally, since choices made by $S$ are already covered, we can assume that every open set in a cover played by $C$ covers the sets chosen by $S$ previously.

    As a result, we have the following figure of a tree of plays which I need to draw:

    (Insert figure here.)

    Note that for $a,b,c\in\omega^{<\omega}$ and $m\leq n$, we know $U_{a\concat m}\subseteq U_{a\concat n}$ and $U_a \subseteq U_{a\concat b}$. Also, if $a=b\concat n \concat c$, then \[U_{a\concat m} = U_{b\concat n \concat c \concat m} \supseteq U_{b\concat n} \supseteq U_{b\concat m}\] (for example: $U_{48397}\supseteq U_{4837} \supseteq U_{47}$).

    We may observe that if $S$ can find an $f:\omega\to\omega$ such that $\bigcup_{n<\omega}U_{f\rest n} = X$, she can use $\{U_{f\rest 0}\}, \{U_{f\rest 1}\}, \dots$ to counter $C$'s strategy.

    Let $\ds V_k^n = \bigcap_{a\in\omega^{\leq n}} U_{a\concat k}$. We claim that (1) $V_k^n$ is open, (2) $\mathcal{V}^n=\{V_0^n,V_1^n,\dots\}$ is increasing, and (3) $\mathcal{V}^n$ is a cover. Proofs:

    \begin{enumerate}
      \item
      $\ds V_k^n = \bigcap_{a\in\omega^{\leq n}} U_{a\concat k} = \bigcap_{a\in k^{\leq n}} U_{a\concat k}$

      \item
      $\ds V_{k+1}^n = \bigcap_{a\in{k+1}^{\leq n}} U_{a\concat (k+1)} \supseteq \bigcap_{a\in{k+1}^{\leq n}} U_{a\concat k} \supseteq \bigcap_{a\in{k}^{\leq n}} U_{a\concat k} = V_k^n$

      \item
      Let $x\in X$ and pick $l<\omega$ such that $x\in U_l$. Then for $a\in l^{\leq n}$, choose $l_a$ such that $x\in U_{a\concat l_a}$. Then let $k>l_a$ for all $a$. It follows that \[x \in \bigcap_{a\in k^{\leq n}} U_{a\concat l_a} \subseteq \bigcap_{a\in k^{\leq n}} U_{a\concat k} = V_k^n\] THIS ISN'T QUITE RIGHT SHOULD FIX THIS
    \end{enumerate}

    Finally, apply Menger to $\mathcal{V}^n$, resulting in the cover $\{V_{f(0)}^0, V_{f(1)}^1, \dots\}$, noting \[X = \bigcup_{n<\omega}V_{f(n)}^n \subseteq \bigcup_{n<\omega} U_{(f\rest n)\concat f(n)} = \bigcup_{n<\omega} U_{f\rest n+1}\]
  \end{proof}

\end{document}