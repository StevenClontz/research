 \documentclass[11pt]{article}

\pdfpagewidth 8.5in
\pdfpageheight 11in

\setlength\topmargin{0in}
\setlength\headheight{0in}
\setlength\headsep{0.2in}
\setlength\textheight{8in}
\setlength\textwidth{6in}
\setlength\oddsidemargin{0in}
\setlength\evensidemargin{0in}
\setlength\parindent{0.25in}
\setlength\parskip{0.1in} 
 
\usepackage{amssymb}
\usepackage{amsfonts}
\usepackage{amsmath}
\usepackage{mathtools}
\usepackage{amsthm}

\usepackage{enumerate}

      \theoremstyle{plain}
      \newtheorem{theorem}{Theorem}
      \newtheorem{lemma}[theorem]{Lemma}
      \newtheorem{corollary}[theorem]{Corollary}
      \newtheorem{proposition}[theorem]{Proposition}
      \newtheorem{conjecture}[theorem]{Conjecture}
      \newtheorem{question}[theorem]{Question}
      \newtheorem{example}[theorem]{Example}
      
      \theoremstyle{definition}
      \newtheorem{definition}[theorem]{Definition}
      
      \theoremstyle{remark}
      \newtheorem{remark}[theorem]{Remark}


% Strategy uparrow shortcuts
\newcommand{\win}{\uparrow}
\newcommand{\prewin}{\uparrow_{\text{pre}}}
\newcommand{\markwin}{\uparrow_{\text{mark}}}
\newcommand{\tactwin}{\uparrow_{\text{tact}}}
\newcommand{\ktactwin}[1]{\uparrow_{#1\text{-tact}}}
\newcommand{\kmarkwin}[1]{\uparrow_{#1\text{-mark}}}
\newcommand{\codewin}{\uparrow_{\text{code}}}
\newcommand{\limitwin}{\uparrow_{\text{limit}}}

\newcommand{\oneptcomp}[1]{#1\cup\{\infty\}}

\newcommand{\congame}[2]{Con_{O,P}(#1,#2)}
\newcommand{\clusgame}[2]{Clus_{O,P}(#1,#2)}

\newcommand{\lfkpgame}[1]{LF_{K,P}(#1)}
\newcommand{\lfklgame}[1]{LF_{K,L}(#1)}

\newcommand{\pfgame}[1]{PF_{F,C}(#1)}

\newcommand{\mengame}[1]{Cov_{C,S}(#1)}

\newcommand{\sigmaprodr}[1]{\Sigma\mathbb{R}^{#1}}
\newcommand{\sigmaprodtwo}[1]{\Sigma2^{#1}}

\newcommand{\concat}{^\frown}
\newcommand{\rest}{\restriction}

\newcommand{\close}[1]{\overline{#1}}

\newcommand{\<}{\langle}
\renewcommand{\>}{\rangle}

\newcommand{\mc}[1]{\mathcal{#1}}

\newcommand{\ds}{\displaystyle}

\begin{document}

% \begin{example}
% Let $L\subseteq \prod_{i=1}^\infty [0,1]$ be the inverse limit space with bonding functions all equal to
% \[
%   f(x) = \left\{
%      \begin{array}{lr}
%        2x & : x \leq 0.5 \\
%        2-2x & : x \geq 0.5
%      \end{array}
%    \right.
% \]
% Then the following hold:
%   \begin{enumerate}
%     \item The subspaces $C_t = \{\alpha\in L : \alpha(1)=t\}$ are each homeomorphic to the Cantor Set.
%     \item All proper subcontinuua are homeomorphic to the unit interval $[0,1]$
%     \item All proper subcontinuua are nowhere dense in the space.
%   \end{enumerate}
% \end{example}

% \begin{proof}
% For (1), consider that for each number $t\in(0,1)$, there are exactly two preimages under $f$ inside $(0,1)$, giving a natural corespondance with the Cantor tree with branches given by points in the Cantor set $2^\omega$. The arguments for $t=0,1$ are similar.

% For (2), we first claim that all basic open sets are of the form $L\cap\prod_{n=1}^\infty B_n$ where $B_n=[0,1]$ for all $n\not=N$. By definition all basic open sets are of the form $L\cap\prod_{n=1}^\infty A_n$ where $A_n=[0,1]$ for all $n\not=N_1,\dots,N_m$. It's easily seen that if $N=N_m$ and $B_N = \bigcap_{i=1}^m (f^{-1})^{(N_m-N_i)}(A_{N_i})$ and $B_n=[0,1]$ otherwise, then $L\cap\prod_{n=1}^\infty A_n = L\cap\prod_{n=1}^\infty B_n$.

% Using this fact, we can argue that each proper subcontinuum $K$ is of the form $L \cap \prod_{n=1}^\infty [a_n,b_n]$ where $0\leq a_N < b_N < 1$ for some $N$. 

% First note that every maximal basic open set missed by $K$ must be of the form $L\cap\prod_{n=1}^\infty B_n$ here $B_n=[0,1]$ for all $n\not=N$ and $[0,1] \setminus B_N$ is connected, that is, $B_N=[0,a)\cup(b,1]$. (If not, then the disconnection of $[0,1]\setminus B_N$ yields a disconnection for $K$.) Thus $K = \prod_{n=1}^\infty [0,1]\setminus ([0,a_n)\cup(b_n,1]) = \prod_{n=1}^\infty [a_n,b_n]$.

% We can see that $a_n<b_n$ always; otherwise the continuum is a single point. Suppose $b_n = 1$ for all $n$, then either $a_n=0$ for all $n$ (contradiction since $K$ is a proper subcontinuum), or $a_N>0$, and $b_{N+1}\leq 1-\frac{a_N}{2}$ (contradition).

% Finally, considering $[a_N,b_N]$ where $0\leq a_N<b_n<1$, we note that as only a single sequence $\alpha$ in $K$ may satisfy $\alpha(N)=t$ for each $t\in [a_N,b_N]$, the projection from $K$ onto the $N^{th}$ coordinate is a homeomorphism onto $[a_N,b_N]$.

% For (3), it is easy to show that for any sequence $\alpha\in L \cap \prod_{n=1}^\infty [a_n,b_n]$ where $0\leq a_N < b_N < 1$ for some $N$, an open neighborhood of $\alpha$ would contain infinte sequences $\beta$ with $\alpha(N)=\beta(N)$ and $\beta(p)>b_p$ for some number $p$.

% \end{proof}

Let $X_n$ be a continuum for each $n<\omega$, and $f_n:X_{n+1}\to X_n$ be a continuous function for each $n<\omega$, and $\ds L=\lim_{\leftarrow}(X_n,f_n)$ be the inverse limit space induced by the spaces $X_n$ and bonding maps $f_n$.

\begin{lemma}\label{basis}
  Let 
    \[
      \mc{B}_N = 
      \{L\cap \prod_{n<\omega} B_n : B_n=X_n \text{ for all } n\not= N \text{ and } B_N \text{ is open in } X_N\}
    \]

  Then $\mc{B} = \bigcup_{N<\omega} \mc{B}_N$ is a basis for $L$.
\end{lemma}

\begin{proof}
  Let $\alpha$ be a member of a basic open set induced by the product topology:
    \[
      L \cap \prod_{n<\omega} A_n
    \]
  where $A_n=X_n$ for all $n\not=N_i$ where $i<m$ and $N_i<N_{i+1}$, and $A_{N_i}$ is open in $X_{N_i}$.

  Let $g_{a\to b}: 2^{X_a}\to 2^{X_b}$ for $a\leq b$ be defined by
  \[
    g_{a\to a} = id_{X_a}
  \]
  \[
    g_{a\to (b+1)} =
    \underbrace{ f^{-1}_{b} \circ \dots \circ f^{-1}_{a+1} \circ f^{-1}_a}_{b-a+1 \text{ times}},
  \]

  Let $N=N_{m-1}$ and
    \[
      B_N = \bigcap_{i<m} g_{N_i\to N} (A_{N_i})
    \]
  noting that $B_N$ is open. Note that $\alpha(N_i)\in A_i$ for all $i<m$ implies $\alpha(N)\in B_N$, and thus $\alpha\in L \cap \prod_{n<\omega} B_n$ where $B_n=X_n$ for all $n\not=N$.

  Finally, let $\beta\in L \cap \prod_{n<\omega} B_n$. Since $\beta(N)\in B_N = \bigcap_{i<m} g_{N_i\to N} (A_{N_i})$, we may easily see that $\beta(N_i)\in A_{N_i}$ for each $i<m$ and thus $\beta\in L\cap\prod_{n<\omega} A_n$.
\end{proof}

\begin{lemma}\label{subcontinuua}
  For each subcontinuum $K\subseteq L$, there are minimal subcontinuua $K_n\subseteq X_n$ such that
    \[
      K = L \cap \prod_{n<\omega} K_n
    \]
\end{lemma}

\begin{proof}
  For each $N<\omega$, let $\mc{B}'_N$ contain all basic open sets in $\mc{B}_N$ whose intersection with $K$ is empty. Then let 
    \[
      K_N = X_N \setminus \bigcup_{B\in\mc{B}'_N} \pi_N(B)
    \]

  $K_N$ is a closed subset of a compact space, and is trivially compact. It is also connected: suppose $K_N\subseteq G\cup H$ with $G,H$ disjoint open in $X_N$. Then $K\subseteq \pi_N^{-1}(G)\cup\pi_N^{-1}(H)$ with $\pi_N^{-1}(G)$,$\pi_N^{-1}(H)$ disjoint open, disconnecting $K$ and showing the contradiction.

  Let $\alpha\in K$. Then as $\alpha\not\in \bigcup_{B\in\mc{B}'_N} B$, we know $\alpha(N)\not\in \bigcup_{B\in\mc{B}'_N} \pi_N(B)$ for any $N<\omega$, so $\alpha \in L \cap \prod_{n<\omega} K_n$.

  Let $\alpha\in L\setminus K$. Then $\alpha\in B \in \mc{B}_N$ for some $N$, and thus $\alpha(N)\in \bigcup_{B\in\mc{B}'_N} \pi_N(B)$. This shows $\alpha(N)\not\in K_N$ and thus $\alpha \not\in L \cap \prod_{n<\omega} K_n$.

  This shows $K = L \cap \prod_{n<\omega} K_n$. To see that minimal candidates for $K_n$ exist, observe that that if
    \[
      K = L \cap \prod_{n<\omega} K_{n,\lambda}
    \]
  for all $\lambda$ in some indexing set $I$, then if $K^*_n = \bigcap_{\lambda\in I} K_{n,\lambda}$ we may see
    \[
      K = L \cap \prod_{n<\omega} K^*_n
    \]
  and thus $K^*_n$ is the minimal subcontinuum for each $n$. ($K^*_n$ is obviously compact, and observe that if it weren't connected, $K$ wouldn't be connected either.)
\end{proof}

\begin{example}
Let $L$ be the inverse limit space induced by $X_n=[0,1]$ and $f_n=f$ where
\[
  f(x) = \left\{
     \begin{array}{lr}
       2x & : x \leq 0.5 \\
       2-2x & : x \geq 0.5
     \end{array}
   \right.
\]
Then the following hold:
  \begin{enumerate}
    \item The subspaces $C_t = \{\alpha\in L : \alpha(0)=t\}$ are each homeomorphic to the Cantor Set.
    \item All proper subcontinuua $K$ are homeomorphic to the unit interval.
    \item All proper subcontinuua $K$ are nowhere dense in the space.
  \end{enumerate}
\end{example}

\begin{proof}
The reader may easily prove the first item by considering the Cantor tree produced by the branching sequences with a fixed initial coordinate.

By Lemma \ref{subcontinuua}, we may write any proper subcontinuum $K$ as
  \[
    K = L \cap \prod_{n<\omega} [a_n,b_n] 
  \]
for $0\leq a_n \leq b_n \leq 1$ with $[a_n,b_n]$ minimal.

It's easily seen that $a_n < b_n$ must actually be strict (otherwise $K$ is a single point).

We proceed to show that if each $[a_n,b_n]$ is minimal, then there must exist some $N$ such that $0\leq a_N < b_N < 1$.

If $a_n=0$ and $b_n=1$ always, then $K=L$ and is not a proper subcontinuum, so either:
  \begin{itemize}
    \item We assume $0< a_N < b_N \leq 1$ and observe by the minimality of $[a_{N+1},b_{N+1}]$ and $f(1-\frac{a_N}{2})=a_N$ that $[a_{N+1},b_{N+1}]\subseteq[a_{N+1},1-\frac{a_N}{2}]$, which implies $0\leq a_{N+1} < b_{N+1} \leq 1 - \frac{a_N}{2} < 1$.
    \item We have $0\leq a_N < b_N < 1$ for free.
  \end{itemize}
Then one of the following occurs:
  \begin{itemize}
    \item Suppose $b_{N+1}=1$; then $a_{N+2}>0$. In this case, the reader may check that each sequence in $K$ has a unique value in the $(N+2)^{th}$ coordinate - if not, then the sequences must ``branch'' in some $M>N+2$ coordinate, but either $b_M<\frac{1}{2}$ or $a_M>\frac{1}{2}$ prevents such branching.
    \item It may be that $0=a_n$ for all $n<\omega$. Then the reader may again check that each sequence in $K$ has a unique value in the $(N+2)^{th}$ coordinate where $0=a_{N+2}<b_{N+2}<1$, as $M>N+2 \Rightarrow b_M<\frac{1}{2}$ prevents branching.
  \end{itemize}

It may be easily verified then that the projection $\pi_{N+2}$ is a homeomorphism from $K$ to $[a_{N+2},b_{N+2}]$.
\end{proof}

\end{document}