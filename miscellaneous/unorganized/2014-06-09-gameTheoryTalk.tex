% $Header: /home/vedranm/bitbucket/beamer/solutions/generic-talks/generic-ornate-15min-45min.en.tex,v 90e850259b8b 2007/01/28 20:48:30 tantau $

\documentclass{beamer}

% This file is a solution template for:

% - Giving a talk on some subject.
% - The talk is between 15min and 45min long.
% - Style is ornate.



% Copyright 2004 by Till Tantau <tantau@users.sourceforge.net>.
%
% In principle, this file can be redistributed and/or modified under
% the terms of the GNU Public License, version 2.
%
% However, this file is supposed to be a template to be modified
% for your own needs. For this reason, if you use this file as a
% template and not specifically distribute it as part of a another
% package/program, I grant the extra permission to freely copy and
% modify this file as you see fit and even to delete this copyright
% notice.


\mode<presentation>
{
  \usetheme{Warsaw}
  % or ...

  \setbeamercovered{transparent}
  % or whatever (possibly just delete it)
}

      \theoremstyle{theorem}
      \newtheorem{proposition}[theorem]{Proposition}
      \theoremstyle{definition}
      \newtheorem{game}[theorem]{Game}
      \newtheorem{question}[theorem]{Question}

\usepackage[english]{babel}
% or whatever

\usepackage[latin1]{inputenc}
% or whatever

\usepackage{times}
\usepackage[T1]{fontenc}
% Or whatever. Note that the encoding and the font should match. If T1
% does not look nice, try deleting the line with the fontenc.


\usepackage{marvosym} % For \Smiley
\usepackage{verbatim} % for \verbatiminput

\title
{Finite and Infinite Games}

\subtitle
{at Lamar University} % (optional)

\author%[Author, Another] % (optional, use only with lots of authors)
{Steven~Clontz~\\http://stevenclontz.com}%\inst{1} \and S.~Another\inst{2}}
% - Use the \inst{?} command only if the authors have different
%   affiliation.

\institute[Auburn University] % (optional, but mostly needed)
{
  %\inst{1}%
  Department of Mathematics and Statistics\\
  Auburn University}
  %\and
  %\inst{2}%
  %Department of Theoretical Philosophy\\
  %University of Elsewhere}
% - Use the \inst command only if there are several affiliations.
% - Keep it simple, no one is interested in your street address.

\date[14-06-09] % (optional)
{June 9, 2014}


% If you have a file called "university-logo-filename.xxx", where xxx
% is a graphic format that can be processed by latex or pdflatex,
% resp., then you can add a logo as follows:

 % \pgfdeclareimage[height=1cm]{university-logo}{auburn_logo.png}
 % \logo{\pgfuseimage{university-logo}}



% Delete this, if you do not want the table of contents to pop up at
% the beginning of each subsection:
%\AtBeginSubsection[]
%{
%  \begin{frame}<beamer>{Outline}
%    \tableofcontents[currentsection,currentsubsection]
%  \end{frame}
%}


% If you wish to uncover everything in a step-wise fashion, uncomment
% the following command:

%\beamerdefaultoverlayspecification{<+->}

% My game notational definitions
\newcommand{\win}{\uparrow}
\newcommand{\prewin}{\uparrow_{\text{pre}}}
\newcommand{\markwin}{\uparrow_{\text{mark}}}
\newcommand{\tactwin}{\uparrow_{\text{tact}}}
\newcommand{\ktactwin}[1]{\uparrow_{#1\text{-tact}}}
\newcommand{\kmarkwin}[1]{\uparrow_{#1\text{-mark}}}
\newcommand{\codewin}{\uparrow_{\text{code}}}
\newcommand{\limitwin}{\uparrow_{\text{limit}}}
\newcommand{\oneptcomp}[1]{#1^*}
\newcommand{\oneptlind}[1]{#1^\dagger}
\newcommand{\congame}[2]{Con_{\pl O\pl P}(#1,#2)}
\newcommand{\clusgame}[2]{Clus_{\pl O\pl P}(#1,#2)}
\newcommand{\lfkpgame}[1]{LF_{\pl K\pl P}(#1)}
\newcommand{\lfklgame}[1]{LF_{\pl K\pl L}(#1)}
\newcommand{\pfgame}[1]{PF_{\pl F\pl C}(#1)}
\newcommand{\mengame}[1]{Cov_{\pl C\pl F}(#1)}
\newcommand{\rothgame}[1]{Cov_{\pl C\pl S}(#1)}
\newcommand{\altrothgame}[1]{Cov_{\pl P\pl O}(#1)}
\newcommand{\fillgame}[1]{Fill^{\subseteq}_{\pl M\pl N}(#1)}
\newcommand{\sfillgame}[1]{Fill^{\subsetneq}_{\pl M\pl N}(#1)}
\newcommand{\kfillgame}[1]{Fill^{\subseteq}_{\pl C\pl F}(#1)}
\newcommand{\ksfillgame}[1]{Fill^{\subsetneq}_{\pl C\pl F}(#1)}
\newcommand{\recallgame}[2]{Rec^{#1}_{\pl F\pl S}(#2)}
\newcommand{\sigmaprodr}[1]{\Sigma\mathbb{R}^{#1}}
\newcommand{\sigmaprodtwo}[1]{\Sigma2^{#1}}
\newcommand{\concat}{^\frown}
\newcommand{\rest}{\restriction}
\newcommand{\cl}[1]{\overline{#1}}
\newcommand{\pow}[1]{\mc{P}(#1)}
\newcommand{\<}{\langle}
\renewcommand{\>}{\rangle}
\newcommand{\al}[1]{{#1}^*}
\newcommand{\mc}[1]{\mathcal{#1}}
\newcommand{\po}{\mathbb{P}}
\newcommand{\pok}{\po_\kappa}
\newcommand{\Lim}{\mathrm{Lim}}
\newcommand{\Suc}{\mathrm{Suc}}
\newcommand{\ds}{\displaystyle}
\newcommand{\alcomp}{\al\parallel}
\newcommand{\rank}{\textrm{rank}}
\newcommand{\dom}{\textrm{dom}}
\newcommand{\scish}{almost-$\sigma$-(relatively compact)}

\usepackage{mathrsfs}
\newcommand{\pl}[1]{\mathscr{#1}}

\newcommand{\vpause}{\pause\vspace{1em}}

\newcommand{\term}[1]{\textbf{#1}}




\begin{document}
% \renewcommand{\pause}{}

\begin{frame}
  \titlepage
\end{frame}

% \begin{frame}{Table of Contents}
%   \tableofcontents
%   % You might wish to add the option [pausesections]
% \end{frame}

\section{Before We Begin...}

\begin{frame}{Grad School!}
  \begin{itemize}
    \item I entered graduate school at Auburn in 2008. Planned on doing
          two Masters in Math and Math Ed. Quickly switched to just Math,
          and I'm finishing up my dissertation on topological games
          to defend this fall or spring.
    \pause
    \item Kind of stumbled into my program. Worked for me, but I feel lucky!
    \pause
    \item I highly recommend graduate school in math - but keep your mind on
          your end career goal (Academia? Industry?)
  \end{itemize}
\end{frame}

\begin{frame}{Preparing for Grad School}
  \begin{itemize}
    \item Think about what you want to study. You're no longer a ``math major'',
          you'll be a graph theorist, or topologist, or algebraist.
    \pause
    \item Find professors that do what you want to study. You'll be picking
          an advisor to work with for a while. (Or, consider getting a Masters
          before PhD to get your feet wet.)
    \pause
    \item Make sure there's Graduate Assistantships available. At Auburn,
          being a GA waives tuition and gives you a paycheck.
          \textbf{Don't go into debt for grad school!}
    \pause
    \item Check website for other requirements. (Math GRE? Required letters?
          Deadlines?)
  \end{itemize}
\end{frame}

\begin{frame}{What's being a Grad Student like?}
  \begin{itemize}
    \item Classes... at first. Hopefully, only classes you're interested in!
    \pause
    \item More free time... to study or research.
    \pause
    \item Get involved! Connect with other grad students outside your
          department. Serving grad student gov't is worthwhile too.
    \pause
    \item Go to conferences. Students often can get funding to cover costs,
          you get to travel, and learn what's happening at other schools.
    \pause
    \item Teach and give talks. Being able to communicate your work is as
          important as actually doing it.
  \end{itemize}
\end{frame}


\section{Introduction}

\subsection{Abstract}

\begin{frame}{Abstract}%{Subtitles are optional.}
    \small
    Two player games of perfect information such as chess and checkers have
    been played for centuries.

    \vpause

    Such games may be mathematically modeled as a tuple $G=\<M,W\>$, where $M$
    represents the moves of the game, and $W$ represents the playthroughs
    (sequences of choices in $M$) which result in a victory for the first
    player.

    \vpause

    We will investigate the classic result that all finite length games are
    determined: that is, exactly one player has a strategy which guarantees
    victory in the game regardless of the moves of her opponent.

    \vpause

    In addition, we will learn how infinite length games are used in fields
    such as set theory and topology by using a game to prove that the real
    numbers are uncountable.
\end{frame}

\subsection{Anatomy of a Game}

\begin{frame}{Heads up}
  This talk is about \term{sequential} or \term{combinatorial} games
  of perfect information.

  \vpause

  Game theory is a broad subject, including classic games like the
  \term{Prisoner's Delimma} where two players make a \term{simultaneous}
  choice, or \term{Yahtzee} where the players face randomness from
  dice rolls.

  \vpause

  However, we're going to look at games in the family of \term{Tic Tac Toe}
  or \term{Chess}, where two players take turns making moves with full knowledge
  of their options and the history of their previous moves.
\end{frame}

\begin{frame}{Some Definitions}
  \begin{definition}
    Let $\omega=\{0,1,2,\dots\}$, and let $B^A$ be the set of all functions
    with domain $A$ and range $B$.

    Then $X^\omega$ contains all functions from $\{0,1,2,\dots\}$ to $X$, or
    equivalently, all sequences of the form $\<x_0,x_1,x_2,\dots\>$ with
    $x_i\in X$.
  \end{definition}
  \pause
  \begin{definition}
    A \term{game} $G$ is a tuple $\<M,W\>$ where $W\subseteq M^\omega$.
    $M$ represents the set of possible moves of the game, and $W$ contains
    certain sequences of moves $\<a_0,b_0,a_1,b_1,\dots\>$ called
    \term{victories} (for the first player).
  \end{definition}
\end{frame}

\begin{frame}
  In this context, all games have two players, and there are no ties.

  \vpause

  If the players of the game are $\pl A$ and $\pl B$, then a \term{playthrough}
  of the game is some sequence in $M^\omega$:
  \[
    p=\<a_0,b_0,a_1,b_1,a_2,b_2,\dots\>
  \]

  \pause

  If $p$ is in $W$, then the first player $\pl A$ has won the game; otherwise,
  the second player $\pl B$ has won the game.
\end{frame}

\begin{frame}{Example}
  As an example, let $\<M,W\>$ be the game \term{Sylver Coinage} with players
  $\pl A$, $\pl B$ where $M=\{2,3,4,\dots\}$.

  \vpause

  Defining $W$ directly as a set is usually obnoxious, so we'll define it
  implicitly by setting this rule: no player can choose a number which is
  the sum of previously chosen numbers, perhaps with repetition.

  \vpause

  Thus, if $4$ and $7$ have been chosen previously, then $25=4+7+7+7$ is not a
  legal move.

  \vpause

  Thus a sequence $\<a_0,b_0,a_1,b_1,\dots\>$ will be in $W$ if it shows the
  second player $\pl B$ breaking the rules before the first player $\pl A$.
\end{frame}

\begin{frame}
  For example, consider the playthrough beginning with
  \[
    \<4,11,6,5,7,3,2,\dots\>
  \]
  \begin{itemize}
    \item $\pl A$ chose $4$
    \item $\pl B$ chose $11$
    \item $\pl A$ chose $6$ (legal moves remaining:
          $\{2,3,5,7,9,13\}$)
    \item $\pl B$ chose $5$ (legal moves remaining:
          $\{2,3,7\}$)
    \item $\pl A$ chose $7$ (legal moves remaining:
          $\{2,3\}$)
    \item $\pl B$ chose $3$ (legal moves remaining: $\{2\}$)
    \item $\pl A$ chose $2$ (legal moves remaining: $\emptyset$)
    \item $\pl B$ chose something illegal and lost.
  \end{itemize}
\end{frame}

\begin{frame}
  This game, invented by John Conway, is an example of a \term{finite game},
  since eventually one of the players are forced to break the given rules.
  (A puzzle I'll leave for you to work on!)

  \vpause

  We've just seen that every sequence of the form $\<4,11,6,5,7,3,2,\dots\>$
  is in $W$, since $\pl A$ wins those playthroughs of the game.
  (Actually, another puzzle: show that all sequences of the form
  $\<4,11,6,5,7,\dots\>$ are in $W$.)

  \vpause

  An artifact of this game model is that all playthroughs are infinite
  sequences. After $\pl B$ makes an illegal move, there's no point to keep
  playing in reality, but the sequences in $W$ stretch on in every possible
  combination...
\end{frame}

\subsection{Strategies and Determinacy}

\begin{frame}{Strategies}
  \begin{definition}
    A \term{strategy} is a function $\sigma$ which turns a finite sequence
    of moves in $M$ into a new move in $M$.
  \end{definition}

  \pause

  Put another way, a strategy is a fixed rule which tells a player what move
  to make during each round \textit{in response} to all the previous moves
  of the game.
\end{frame}

\begin{frame}{Attacks}
  \begin{definition}
    An \term{attack} is sequence of moves in $M$.
  \end{definition}

  \pause

  Put another way, an attack is a fixed rule which tells a player what move
  to make during each round \textit{ignoring} all the previous moves of
  the game.
\end{frame}

\begin{frame}{Winning Strategies}
  \begin{definition}
    The \term{result} of a game for which $\pl A$ uses the strategy $\sigma$
    $\pl B$ uses the attack $\<b_0,b_1,\dots\>$ is the playthrough of the
    game $\<\sigma(\emptyset),b_0,\sigma(b_0),b_1,\sigma(b_0,b_1),\dots\>$
  \end{definition}

  (Or the similar definition when $\pl B$ has a strategy and $\pl A$
  has an attack.)

  \pause

  \begin{definition}
    If $\sigma$ is a strategy for $\pl A$ such that the result of the game for
    every possible attack by $\pl B$ is in $W$, then $\sigma$ is a
    \term{winning strategy}.
  \end{definition}

  (Or the similar definition when $\pl B$ has a strategy, and all results
  are not in $W$.)
\end{frame}

\begin{frame}
  \begin{definition}
    If one of the players has a winning strategy for a game, then that
    game is said to be \term{determined}.
  \end{definition}

  Obviously, both players can't have a guaranteed way to win the same
  game, but is it possible that neither player can guarantee a way to win?
  That is, for every fixed strategy by either player, could the opponent
  always have some chance of getting lucky and beating it?
\end{frame}

\section{Determinacy of Finite Games}

\subsection{Borel Determinacy Theorem}

\begin{frame}{Borel Determinacy Theorem}
  We could use a very strong topological and set-theoretic result to prove
  that finite games are determined.

  \begin{theorem}
    If $M$ is given the discrete topology, and $M^\omega$ is given the usual
    product topology, then the game $G=\<M,W\>$ is determined whenever $W$
    is a Borel subset of the space $M^\omega$.
  \end{theorem}

  \pause

  With a little topology, you can show that if $G$ is finite, then $W$ is
  an open set, which implies it's Borel. Thus finite games \textit{are}
  determined: one of the players has a winning strategy.

  \vpause

  Of course, all that requires a few semesters of graduate topology to grok,
  so maybe there's a better way...
\end{frame}

\subsection{Decision Tree Proof}

\begin{frame}{Decision Trees}
  Finite games can be modeled as a decision tree.

  \centerline{\includegraphics[width=3in]{decisionTree1.pdf}}

  The above tree models a game where $\pl A$ and $\pl B$ alternate choosing
  ``left'' or ``right'' moves to descend the tree. A player wins if they
  move into a terminal node of the tree, since the opponent cannot move farther.
\end{frame}

\begin{frame}
  \centerline{\includegraphics[width=3in]{decisionTree2.pdf}}

  \vspace{1em}

  We can label the tree by first showing the states where $\pl A$ (blue) and
  $\pl B$ (red) have already won the game.
\end{frame}

\begin{frame}
  \centerline{\includegraphics[width=3in]{decisionTree3.pdf}}

  \vspace{1em}

  Then, we can move back and label the spaces where the active player is able
  to move to a vertex of their color.
\end{frame}

\begin{frame}
  \centerline{\includegraphics[width=3in]{decisionTree4.pdf}}

  \vspace{1em}

  Eventually, we label the entire tree based on when the active player has
  the option to move into their color or not.
\end{frame}

\begin{frame}
  \centerline{\includegraphics[width=3in]{decisionTree5.pdf}}

  \vspace{1em}

  Since the top vertex is blue, $\pl A$ has a winning strategy: always make
  the choice which leads to another blue vertex.
\end{frame}

\section{Infinite Games}

\subsection{Infinite Games}

\begin{frame}{Infinite Games}
  Why do we care so much about determinacy? Why did we define game playthroughs
  to be infinite sequences?

  \vpause

  These topics come into play when considering infinite games.

  \vpause

  \begin{definition}
    A game $G$ is \term{infinite} if there exists a playthrough such that it
    is still possible for either player to win during every round of the game.
  \end{definition}
\end{frame}

\begin{frame}{How are they played?}
  Even though a game could never actually be played, we can still construct
  strategies (functions) and attacks (sequences), and we can compute the
  result of a game given a strategy and attack.

  \vpause

  Put another way, for every infinite game, there is a finite analog of the
  game which lasts exactly one round: one player chooses a strategy, followed
  by the opponent choosing an attack based upon it. The result of the infinite
  game is computed, and that determines the result of the single-round finite
  game.
\end{frame}

\begin{frame}{Determinacy}
  Like many theorems about infinite mathematical objects, whether infinite
  games are determined depend on your set-theoretic axioms. Mathematicians who
  work in foundations often use the Zermelo-Fraenkel (ZF) set theory, which
  isn't powerful enough to write proofs on the subject.

  \vpause

  The \term{Axiom of Determinacy} states that all games which
  involve (countably) infinite moves are determined: one of the players always
  can construct a winning strategy.

  \vpause

  But the more commonly used \term{Axiom of Choice} can be used to construct
  a game where if either player fixes a strategy, the other player can always
  create a counter-attack which defeats it. (See the Banach-Mazur game and
  Bernstein subsets of the real numbers.)
\end{frame}

\subsection{Applications}

\begin{frame}{Example Game}
  Let $\<M,W\>$ be \term{Convergence Game $A$} where $M$ is the set of real
  numbers $\mathbb{R}$, and $A$ is a subset of the real numbers.

  \vpause

  Players $\pl A$, $\pl B$ must follow the rule that every real number chosen is
  strictly between the latest numbers chosen by $\pl A$ and $\pl B$.

  \vpause

  The start of a playthrough could be
    \[
      \left\<
        5,
        \underset{(5<12)}{12},
        \underset{(5<2\pi<12)}{2\pi},
        \underset{(2\pi<7<12)}{7},
        \underset{(2\pi<6.5<7)}{6.5},
        \dots
      \right\>
    \]
\end{frame}

\begin{frame}{A Winning Condition}
  Since there's always infinitely many numbers between every
  two real numbers, $\pl A$ and $\pl B$ always have legal moves to choose from.

  \vpause

  That's why we must add a winning condition: if both players always make legal
  moves, then $\pl A$ wins if the numbers she chose form a sequence converging
  to a number in the set $A$, and $\pl B$ wins otherwise.

  \vpause

  For example, $\pl A$ won the earlier playthrough if the sequence
  $\<5,2\pi,6.5,\dots\>$ converges to a number in $A$.
\end{frame}

\begin{frame}{Who wins?}
  \begin{theorem}
    $\pl B$ has a winning strategy in Convergence Game $A=\{a_0,a_1,\dots\}$
    (that is, when $A$ is a ``countable'' set).
  \end{theorem}
\end{frame}

\begin{frame}
  \begin{proof}
    $\pl B$'s strategy is to take the list of numbers $\{a_0,a_1,\dots\}$,
    and every turn, $\pl B$ chooses the number furthest to the left of the list
    which is legal to play.

    \vpause

    Then, at the ``end'' of the game, assuming that $\pl A$ also followed the
    rules, every number in the list $\{a_0,a_1,\dots\}$ is either to the left
    of one of $\pl A$'s points, or to the right of one of $\pl B$'s points.

    \vpause

    Thus, $\pl A$'s points cannot converge to any of the numbers in
    $\{a_0,a_1,\dots\}$ no matter what attack she attempts.
  \end{proof}
\end{frame}

\begin{frame}{The Application}
  Using this game, we get a classic set theory result due to Cantor:

  \vpause

  \begin{theorem}
    The real numbers $\mathbb{R}$ cannot be written in a list
    $\{r_0,r_1,\dots\}$ (they are ``uncountable'').
  \end{theorem}

  \begin{proof}
    Every increasing bounded above sequence converges to a real number
    (see Cal II). Thus $\pl A$ always wins the Convergence Game $A=\mathbb{R}$.

    But since $\pl B$ wins if $A$ is a countable set which can be written like
    $\{a_0,a_1,\dots\}$, we know that $\mathbb{R}$ cannot be countable.
  \end{proof}
\end{frame}

\section*{}

\begin{frame}
Questions? Thanks for having me!
\end{frame}


\end{document}


