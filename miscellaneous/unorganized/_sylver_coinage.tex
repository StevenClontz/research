
But a two-player game need not have a bounded number of rounds, as is
demonstrated by this game due to John Conway:
\cite{winning_ways}

\begin{game}
  Let $\coingame$ denote the game \term{Sylver Coinage} with players 
  $\pl A$ and $\pl B$. During round $n$, $\pl A$ chooses an integer $a_n>0$,
  followed by $\pl B$ choosing another integer $b_n>0$, with the rule that
  neither player may choose an integer which is the sum of nonnegative integer 
  multiples of previously chosen integers. $\pl A$ wins the game if $\pl B$
  ever chooses the number $1$, and $\pl B$ wins otherwise.
\end{game}

After some thought, the reader may confirm that there will be no 
legal moves except $1$ remaining after a finite number of rounds. 
Specifically, if $\pl A$
chooses $a_0$ in the initial round, then all following legal moves must be
of the form $i \mod\, a_0$ for $0<i<a_0$. $i \mod\, a_0$ may be repeated,
but all following repetitions must be strictly less than the previous
iterations, which may only be done finitely many times. Thus after a finite 
number of moves, there will only be finitely many legal moves remaining. 
\footnote{
  A more involved
  argument due to Sylvester (for whom the game is named)
  shows that the number of legal moves 
  is finite as soon as two coprime numbers have been played. \cite{sylvester}
}

Sylver Coinage was defined as an example of an ``unboundedly unboundedly 
bounded'' game, the maximum length of which is not decided until at least
the second move.