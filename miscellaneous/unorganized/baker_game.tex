\documentclass[11pt]{article}

\pdfpagewidth 8.5in
\pdfpageheight 11in

\setlength\topmargin{0in}
\setlength\headheight{0in}
\setlength\headsep{0.2in}
\setlength\textheight{8in}
\setlength\textwidth{6in}
\setlength\oddsidemargin{0in}
\setlength\evensidemargin{0in}
\setlength\parindent{0.25in}
\setlength\parskip{0.1in} 
 
\usepackage{amssymb}
\usepackage{amsfonts}
\usepackage{amsmath}
\usepackage{mathtools}
\usepackage{amsthm}

\usepackage{enumerate}

      \theoremstyle{plain}
      \newtheorem{theorem}{Theorem}
      \newtheorem{lemma}[theorem]{Lemma}
      \newtheorem{corollary}[theorem]{Corollary}
      \newtheorem{proposition}[theorem]{Proposition}
      \newtheorem{conjecture}[theorem]{Conjecture}
      \newtheorem{question}[theorem]{Question}
      \newtheorem{example}[theorem]{Example}
      
      \theoremstyle{definition}
      \newtheorem{definition}[theorem]{Definition}
      
      \theoremstyle{remark}
      \newtheorem{remark}[theorem]{Remark}


% Strategy uparrow shortcuts
\newcommand{\win}{\uparrow}
\newcommand{\prewin}{\uparrow_{\text{pre}}}
\newcommand{\markwin}{\uparrow_{\text{mark}}}
\newcommand{\tactwin}{\uparrow_{\text{tact}}}
\newcommand{\ktactwin}[1]{\uparrow_{#1\text{-tact}}}
\newcommand{\kmarkwin}[1]{\uparrow_{#1\text{-mark}}}
\newcommand{\codewin}{\uparrow_{\text{code}}}
\newcommand{\limitwin}{\uparrow_{\text{limit}}}

\newcommand{\oneptcomp}[1]{#1\cup\{\infty\}}

\newcommand{\congame}[2]{Con_{O,P}(#1,#2)}
\newcommand{\clusgame}[2]{Clus_{O,P}(#1,#2)}

\newcommand{\lfkpgame}[1]{LF_{K,P}(#1)}
\newcommand{\lfklgame}[1]{LF_{K,L}(#1)}

\newcommand{\pfgame}[1]{PF_{F,C}(#1)}

\newcommand{\mengame}[1]{Cov_{C,S}(#1)}

\newcommand{\sigmaprodr}[1]{\Sigma\mathbb{R}^{#1}}
\newcommand{\sigmaprodtwo}[1]{\Sigma2^{#1}}

\newcommand{\concat}{^\frown}
\newcommand{\rest}{\restriction}

\newcommand{\close}[1]{\overline{#1}}

\newcommand{\<}{\langle}
\renewcommand{\>}{\rangle}

\newcommand{\mc}[1]{\mathcal{#1}}

\newcommand{\ds}{\displaystyle}

\usepackage{hyperref}

\begin{document}

\textbf{Notes on an example due to Baker}

In \cite{baker}, the author defines a game played on the unit interval.

\begin{game}
  Let $G(X,J)$ denote a game with players $\pl A$ and $\pl B$.

  In round 0, $\pl A$ chooses a number $a_0$ such that $0\leq a_0\leq 1$, 
  followed by $\pl B$ choosing a number $b_0$ such that $a_0<b_0\leq 1$.

  In round $n+1$, $\pl A$ chooses a number $a_{n+1}$ such that 
  $a_n<a_{n+1}< b_n$, followed by $\pl B$ choosing a number $b_{n+1}$ such
  that $a_{n+1}< b_{n+1}<b_n$.

  $\pl A$ wins the game if $\lim_{n\to\infty}a_n\in X$, and $\pl B$ wins
  otherwise.
\end{game}

The game is strongly related to one formulation of the Banach-Mazur game
played upon the unit interval, which has been extensively studied
\cite{telgarsky}.

\begin{game}
  Let $M(X,J)$ denote the Banach-Mazur interval game with players $\pl A$ and
  $\pl B$.

  In round 0, $\pl A$ chooses a closed interval $I_0\subseteq J=[0,1]$,
  followed by $\pl B$ choosing a closed interval $J_0\subseteq I_0$.

  In round $n+1$, $\pl A$ chooses a closed interval $I_{n+1}\subseteq J_n$,
  followed by $\pl B$ choosing a closed interval $J_{n+1}\subseteq I_{n+1}$.
\end{game}

The author of \cite{baker} asks if there is a set $X$ such that $G(X,J)$ is
indetermined: neither $\pl A$ nor $\pl B$ have a winning strategy.

We show that such a set would also make $MB(X,J)$ indeteremined.

\begin{theorem}
  $\pl A\win MB(X,J) \Rightarrow \pl A\win G(X,J)$ and
  $\pl B\win MB(X,J) \Rightarrow \pl B\win G(X,J)$.
  (Thus if $MB(X,J)$ is determined, then $G(X,J)$ is determined.)
\end{theorem}

\begin{proof}
  First let $\sigma$ witness $\pl A\win MB(X,J)$. We define the strategy $\tau$
  for $\pl A$ in $G(X,J)$ like so:
    \[
      a_0 = \tau(\emptyset) = \inf(\sigma(\emptyset))
    \]
    \[
      a_{n+1} = \tau(\<b_0,\dots,b_n\>) 
        = 
      \inf\left(\sigma\left(\left\<
        \left[\frac{2a_0+b_0}{3},\frac{a_0+2b_0}{3}\right],
        \dots,
        \left[\frac{2a_n+b_n}{3},\frac{a_n+2b_n}{3}\right]
      \right\>\right)\right)
    \]

  It is easily seen that
    $
      \bigcap_{n<\omega}
      [\frac{2a_n+b_n}{3},\frac{a_n+2b_n}{3}]
        =
      \{x\}
    $
  and $x\in X$ since $\sigma$ is a winning strategy.
  Thus $\lim_{n\to\infty}a_n=x$ and $\pl A\win G(X,J)$.

  If $\sigma$ now witnesses $\pl B\win MB(X,J)$, then a similar argument shows
  that
    \[
      b_n = \tau(\<b_0,\dots,b_n\>) 
        = 
      \inf\left(\sigma\left(\left\<
        \left[\frac{2a_0+b_0}{3},\frac{a_0+2b_0}{3}\right],
        \dots,
        \left[\frac{2a_n+b_n}{3},\frac{a_n+2b_n}{3}\right]
      \right\>\right)\right)
    \]
  defines a winning strategy for $\pl B$ in $G(X,J)$.
\end{proof}

\begin{corollary}
  If $X$ is Baire, then $G(X,J)$ is determined.
\end{corollary}

\begin{proof}
  If $X$ is a Baire subset of a Polish space, then $MB(X,J)$ is determined.
\end{proof}

\newpage

\begin{thebibliography}{9}

\bibitem{baker}
  Baker, M. H.,
  \emph{Uncountable sets and an infinite real number game}.
  \url{http://arxiv.org/pdf/math/0606253.pdf}
  2006.

\bibitem{telgarsky}
  Telgarksy, R.,
  \emph{Topological games: On the 50th anniversary of the Banach-Mazur game}.
  \url{http://www.telgarsky.com/1987-RMJM-Telgarsky-Topological-Games.pdf}
  1987.

\end{thebibliography}

\end{document}


