%!TEX root = dissertation.tex
% ^ leave for LaTeXTools build functionality

\chapter{Introduction}

Topological games have been studied since the 1930s to characterize
properties of topological spaces. Generally speaking, a topological game $G(X)$
is a two player game defined for each topological space $X$ such that a
topological property $P$ is characterized by a player $\pl A$
having a so-called ``winning'' strategy for $G(X)$ which cannot be countered
by the opponent, denoted $\pl A\win G(X)$.

The study of limited information strategies in topological games involves
the following observation. If the existence of a winning strategy for a player
in $G(X)$ characterizes property $P$, then the existence of a winning strategy
which doesn't require perfect information about the history of the game
characterizes a (perhaps non-strictly) stronger property $Q$.

This document is organized into five chapters in addition to this introductory
chapter. The second provides preliminary definitions and conventions used
throughout the paper, and the remaining chapters each consider different
topological games from the literature and extend results on these games
by considering limited information strategies.

The first game considered is Gruenhage's convergence game $\gruConGame{X}{x}$,
originally called the $W$-game and introduced in \cite{MR0428275} to
answer a question of Phil Zenor, and later used to characterize the $W$-space
property generalizing first-countability. With regards to limited information
and the one-point compactification of an uncountable cardinal, denoted
$\oneptcomp\kappa$, Peter Nyikos
noted that the first player $\pl O$ has a winning tactical strategy considering
only the most recent move of the opponent, denoted
$\pl O\tactwin\gruConGame{\oneptcomp\kappa}{\infty}$. If the game is
slightly altered to $\gruConGameHard{X}{x}$, perfect information strategies are
preserved. However, Nyikos demonstrated that $\pl O$ lacks a winning Markov
strategy considering only the most recent move of the opponent and the round
number, denoted $\pl O\notmarkwin\gruConGameHard{\oneptcomp\kappa}{\infty}$.
This result has been extended as follows:

\begin{thm*}
  $\pl O\notkmarkwin{k}\gruConGameHard{\oneptcomp\kappa}{\infty}$ for
  $\kappa>\omega$, $k<\omega$.
  That is, $\pl O$ cannot force a win in the game with a $k$-Markov strategy
  which only considers $k$ previous moves of the opponent and the round number.
\end{thm*}

The game is made easier for $\pl O$ by weakening the game to
$\gruClusGameHard{X}{x}$, exchanging the convergence requirement with
clustering. In this scenario, the author has shown the size of the
uncountable cardinal $\kappa$ matters.

\begin{thm*}
  $\pl O\markwin\gruClusGameHard{\oneptcomp\kappa}{\infty}$ if and only if
  $\kappa\leq\omega_1$.
\end{thm*}

Based upon this game is a relatively new game from the literature,
Bell's ``proximal'' game $\bellConGame{X}$. Peter Nyikos noted that
Corson compact spaces satisfy $\pl D\win\bellConGame{X}$, and asked if this
in fact was a characterization of Corson compacts. The author
answered this question with Gary Gruenhage in $\cite{MR3227201}$.

\begin{thm*}
  Among compact Hausdorff spaces, $X$ is Corson compact if and only if
  $\pl D\win\bellConGame{X}$.
\end{thm*}

Many perfect information results on $\bellConGame{X}$ may be easily extended to
limited information analogs, but others are more elusive. Bell showed
in \cite{MR3239205} that $\pl D\win\bellConGame{X}$ extends to arbitrary
sigma-products, but the proof makes non-trivial use of perfect information.
Thus for $k$-Markov strategies, the result has only been shown to hold for
countable products.

\begin{thm*}
  For $k<\omega$, if $\pl D\kmarkwin{k}\bellConGame{X_i}$ for all $i<\omega$,
  then $\pl D\kmarkwin{k}\bellConGame{\prod_{i<\omega} X_i}$.
\end{thm*}

Another game due to Gruenhage is the compact-point game $\gruKPGame{X}$
which may be used to characterize metacompactness and $\sigma$-metacompactness
among locally compact spaces using tactical and Markov strategies. Using
predetermined strategies which only consider the round number:

\begin{thm*}
  For locally compact spaces, $\pl K\prewin\gruKPGame{X}$ if and only if
  $X$ is hemicompact.
\end{thm*}

The compact-compact variation $\gruKLGame{X}$ may be used to characterize
paracompactness among locally compact spaces
using perfect information strategies.
However, the existence of winning predetermined strategies for $\gruKLGame{X}$
coincides
with $\gruKPGame{X}$ when considering locally compact or even compactly generated
spaces. By investigating the subspace $\omega\cup\{\mc F\}$ of the Stone-Cech
compactification $\beta\omega$ consisting of a single free ultrafilter,
a distinction between the two games is revealed.

\begin{thm*}
  There exists a free ultrafilter $\mc F$ such that
  $\pl K \prewin\gruKPGame{\omega\cup\{\mc F\}}$, but
  $\pl K \notprewin\gruKLGame{\omega\cup\{\mc F\}}$ for any free
  ultrafilter $\mc F$.
\end{thm*}
\begin{thm*}
  Assuming $CH$, there exists a free ultrafilter $\mc S$ such that
  $\pl K \notprewin\gruKPGame{\omega\cup\{\mc S\}}$.
\end{thm*}

A winning $k+1$-Markov strategy for $\gruConGame{X}{x}$ may be improved to
a winning Markov strategy; however, it remains open whether this same
result holds true for $\gruKPGame{X}$. The chapter on $\gruKPGame{X}$ concludes
with a class of spaces which may seem to provide a counterexample at first,
however:

\begin{thm*}
  The space $\pmb X$ satisfies $\pl K\win\gruKPGame{\pmb X}$, but
  $\pl K\notkmarkwin{k}\gruKPGame{\pmb X}$ for all $k<\omega$.
\end{thm*}

The final chapter is based upon one of the oldest topological games:
the Menger game $\menGame{X}$, characterizing Menger's covering property.
In this game, $k$-Markov strategies need only consider at most
two moves of the opponent.

\begin{thm*}
  $\pl F\kmarkwin{k+2}\menGame{X}$ if and only if
  $\pl F\kmarkwin{2}\menGame{X}$.
\end{thm*}

For a one-point Lindel\"of-ication $\oneptlind\kappa$ of discrete $\kappa$,
the topological game $\menGame{\oneptlind\kappa}$ is equivalent to a
set-theoretic game $\cloFillIntGame{\kappa}$ with respect to $\pl F$'s $k$-Markov
strategies. Using this game, one may see the following.

This game has similarities to a game introduced by
Scheepers \cite{MR1129143}, who also introduced a set-theoretic axiom
$\alcompS\kappa$ to study it. This axiom may also be applied to study
$\menGame{\oneptlind\kappa}$.

\begin{thm*}
  Assume $\alcompS{\kappa}$. Then
  $\pl F\kmarkwin{2}\menGame{\oneptlind\kappa}$.
\end{thm*}

Scheepers observed that $\alcompS{\omega_1}$ is a theorem of $ZFC$.

\begin{thm*}
  $\pl F\kmarkwin{2}\menGame{\oneptlind\omega_1}$ but
  $\pl F\notmarkwin\menGame{\oneptlind\omega_1}$.
\end{thm*}

Telgarsky \cite{MR753073} and Scheepers \cite{MR1273523} provided different
proofs to show that among metrizable spaces, $\pl F\markwin\menGame{X}$
characterizes $\sigma$-compactness. These results may be extended by considering
Markov strategies. Recall that for Lindel\"of spaces, metrizability is
characterized by regularity and second-countability.

\begin{thm*}
  For regular spaces, $\pl F\markwin\menGame{X}$ if and only if
  $X$ is $\sigma$-compact.
\end{thm*}

\begin{thm*}
  For second-countable spaces, $\pl F\markwin\menGame{X}$ if and only if
  $\pl F\win\menGame{X}$.
\end{thm*}

The topological property characterized by $\pl F\kmarkwin{2}\menGame{X}$
seems to be heretofore unstudied. The final chapter concludes by
introducing the sufficient robustly Menger property, and uses it to study the
space $R_\omega$ finer than the usual Euclidean real line.

\begin{thm*}
  Assume $\alcompS{2^\omega}$. Then $\pl F\kmarkwin{2}\menGame{R_\omega}$.
\end{thm*}