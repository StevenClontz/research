%!TEX root = dissertation.tex
% ^ leave for LaTeXTools build functionality

\chapter{Toplogical Games and Strategies\\ of Perfect and Limited Information}

The goal of this paper is to explore the applications of limited information
strategies in existing topological games. (TODO: History of combinatorical
and topological games?)

\section{Games}

Intuitively, the games studied in this paper are two-player games for which
each player takes turns making a choice from a set of possible moves. At
the conclusion of the game, the choices made by both players are examined,
and one of the players is declared the winner of that playthrough, with
no ties allowed.

Games may be modeled mathematically in various ways (TODO: cite digraph model)
but we will find it convenient to think of them in terms defined by
Gale and Stewart. (TODO: cite)

\begin{defn}
  A \term{game} is a tuple $\<M,W\>$ such that 
  $W\subseteq M^{\omega}$. $M$ is set of \term{moves} for
  the game, and $M^{\omega}$ is the set of all possible 
  \term{playthroughs} of the game.

  $W$ is the set of \term{winning playthroughs} or \term{victories} for the 
  first player, and $M^{\omega}\setminus W$ is the set of victories for the 
  second player. ($W$ is often called the \term{payoff set} for the
  first player.)
\end{defn}

To illustrate this definition, we may formally model the well-known game 
Tic-Tac-Toe. For convenince, we may label the game board as a ``magic square''
of the numbers $0$ through $8$ such that the combinations of three integers
which sum to $12$ are exactly the rows, columns, and diagonals of the
grid. 

(TODO: image) 

Then writing $X$ or $O$ on
the game board is equivalent to choosing a number $0$ through $8$, and 
getting a ``tic-tac-toe'' is equivalent to having chosen three numbers which 
sum to $12$.

\begin{game}
  Let $\tttgame$ denote the game \term{Tic-Tac-Toe}, with players $\pl X$ 
  and $\pl O$.

  The moves $M$ for the game are the first nine non-negative 
  integers $\{0,\dots,8\}$. The victories $W$ for $\pl X$ 
  consist of sequences $\<X_0,O_0,X_1,O_1,\dots\>$ such that either of the 
  following hold:
    \begin{itemize}
      \item There exists some $k<4$ such that $\<X_0,O_0,\dots,X_k\>$ contains
            no repeated integers, but $\<X_0,O_0,\dots,X_k,O_k\>$ does.

            ($\pl O$ drew her mark on a space already occupied.)
      \item There exists $i<j<k<5$ such that $\<X_0,O_0,\dots,X_4\>$ contains 
            no repeated integers and $X_i+X_j+X_k=12$, and for all
            $i'<j'<k'<k$ it follows that $O_{i'}+O_{j'}+O_{k'}\not=12$.

            ($\pl X$ did not draw her mark on a space that was already occupied,
            and was able to secure a ``tic-tac-toe'' before $\pl O$.)
            \footnote{
              The observant reader will note that in this formulation of 
              Tic-Tac-Toe, player $\pl O$ wins ``cat's games'' where neither 
              player secures a ``tic-tac-toe'' (since no ties are allowed).
            }
    \end{itemize}
\end{game}

Due to the tedium of defining games explicitly from the definition, we 
typically define games by declaring the \term{rules} that each player must 
follow (e.g. players may not make a mark on a space which was already marked)
and the \term{winning condition} for the first player (e.g. $\pl X$ secures
a ``tic-tac-toe'' before $\pl O$). Then a playthrough is in $W$ if either 
the first player made only \term{legal moves} which observed the game's rules
and the playthrough satisifies the winning condition, or the second player
made an \term{illegal move} which contradicted the game's rules.

Often, players should not be able to choose from the same set of moves
(e.g. player Black cannot move White's pieces in chess), in which case $M$ may
be partitioned into two sets $M_0,M_1$ and the rules for the game may be defined
such that each player may only choose from one of those sets 
(e.g. Black will automatically lose if she moves one of White's pieces).

An artifact of this game model is that games never technically end, since 
playthroughs of the game are infinite sequences. Nonetheless, Tic-Tac-Toe
is always decided by $\pl X$'s fifth move, and in chess, there is also an
upper bound for the length of a game due to the finite number of ways pieces
may be positioned on the chessboard (and the ``threefold rule''
which declares the game finished if the same position is repeated three times).

But a two-player game need not have a bounded number of rounds, as is
demonstrated by this game due to John Conway:
\cite{winning_ways}

\begin{game}
  Let $\coingame$ denote the game \term{Sylver Coinage} with players 
  $\pl A$ and $\pl B$. During round $n$, $\pl A$ chooses an integer $a_n>0$,
  followed by $\pl B$ choosing another integer $b_n>0$, with the rule that
  neither player may choose an integer which is the sum of nonnegative integer 
  multiples of previously chosen integers. $\pl A$ wins the game if $\pl B$
  ever chooses the number $1$, and $\pl B$ wins otherwise.
\end{game}

After some thought, the reader may confirm that there will be no 
legal moves except $1$ remaining after a finite number of rounds. 
Specifically, if $\pl A$
chooses $a_0$ in the initial round, then all following legal moves must be
of the form $i \mod\, a_0$ for $0<i<a_0$. $i \mod\, a_0$ may be repeated,
but all following repetitions must be strictly less than the previous
iterations, which may only be done finitely many times. Thus after a finite 
number of moves, there will only be finitely many legal moves remaining. 
\footnote{
  A more involved
  argument due to Sylvester (for whom the game is named)
  shows that the number of legal moves 
  is finite as soon as two coprime numbers have been played. \cite{sylvester}
}

Sylver Coinage was defined as an example of an ``unboundedly unboundedly 
bounded'' game, the maximum length of which is not decided until at least
the second move.


\subsection{Infinite and Topological Games}

(TODO: History of infinite games in topology / descriptive set theory)

\begin{defn}
  A game is said to be an \term{infinite game} if there exists a playthrough
  $p\in M^\omega$ of the game such that for all $n<\omega$,
    $ 
      \{q : q\geq p\rest n \}
    $
  is not a subset of either $W$ or $M^\omega\setminus W$.
\end{defn}

Put another way, an infinite game need not be decided after a finite number
of rounds. Games which are not infinite are of course called
\term{finite games}. 

As an illustration, we may consider an example due to 
Baker \cite{baker}. 

\begin{game}
  Let $\bakergame{A}$ denote a game with players $\pl A$ and $\pl B$,
  defined for each subset $A\subset \mathbb{R}$.
  In round $0$, $\pl A$ chooses a number $a_0$, followed by $\pl B$ choosing
  a number $b_0$ such that $a_0<b_0$.
  In round $n+1$, $\pl A$ chooses a number $a_{n+1}$ such that 
  $a_n<a_{n+1}<b_n$, followed by $\pl B$ choosing a number $b_{n+1}$ such that
  $a_{n+1}<b_{n+1}<b_n$.

  $\pl A$ wins the game if the sequence $\<a_n:n<\omega\>$ converges to a 
  point in $A$, and $\pl B$ wins otherwise.
\end{game}

Certainly, $\pl A$ and $\pl B$ will never be in
a position without (infinitely many) legal moves available, so this is
an infinite game. While the game could never be ``completed'' in reality,
the winning condition considers the infinite sequence of moves made by the
players and declares a victor at the ``end'' of the game.

As a simple example, if $A=\mathbb{R}$, then $W=M^\omega$. That is,
all playthroughs of the game are victories for $\pl A$, since every
bounded increasing sequence converges to some real number. 

\begin{defn}
  A \term{topological game} is a game defined in terms of an arbitrary
  topological space.
\end{defn}

Topological games are usually infinite games. One of the earliest examples
of a topological game is the Banach Mazur game, proposed by Stanislaw Mazur
as Problem 43 in Stefan Banach's Scottish Book (1935). We give a more
general definition here.

\begin{game}
  Let $\bmgame{X}$ denote the \term{Banach-Mazur game} with players $\pl E$,
  $\pl N$ defined for each topological space $X$.
  In round $0$, $\pl E$ chooses a nonempty open set $E_0\subseteq X$, followed
  by $\pl N$ choosing a nonempty open subset $N_0\subseteq E_0$.
  In round $n+1$, $\pl E$ chooses a nonempty open subset $E_{n+1}\subseteq N_n$, 
  followed by $\pl N$ choosing a nonempty open subset 
  $N_{n+1}\subseteq E_{n+1}$.

  $\pl E$ wins the game if $\bigcap_{n<\omega} E_n = \emptyset$, 
  and $\pl N$ wins otherwise.
\end{game}

For example, if $X$ is a locally compact Hausdorff space, $\pl N$ can ``force'' 
a win by choosing $N_0$ such that $\cl{N_0}$ is compact, and choosing 
$N_{n+1}$ such that 
$N_{n+1}\subseteq\cl{N_{n+1}}\subseteq O_{n+1}\subseteq N_n$ 
(possible since $N_n$ is a compact Hausdorff $\Rightarrow$ normal space). 
Since $\bigcap_{n<\omega} E_n = \bigcap_{n<\omega} N_n$ is the decreasing 
intersection of compact sets, it cannot be empty.

This concept of when (and how) a player can ``force'' a win in certain
topological games is the focus of this manuscript.



\section{Strategies}

We shall make the notion of forcing a win in a game rigorous by introducing
``strategies'' and ``attacks'' for games.

\begin{defn}
  A \term{strategy} for a game $G=\<M,W\>$ is a function 
  from $M^{<\omega}$ to $M$.
\end{defn}

\begin{defn}
  An \term{attack} for a game $G=\<M,W\>$ is a function 
  from $\omega$ to $M$.
\end{defn}

Intuitively, a strategy is a rule for one of the players on how to play
the game based upon the previous moves of her opponent, while an attack is
a fixed strike by an opponent.

\begin{defn}
  The \term{result} of a game for a strategy $\sigma$ for the first player and 
  attack $\<a_0,a_1,\dots\>$ for  the second player is the playthrough
    \[
      \<\sigma(\emptyset),a_0,\sigma(\<a_0\>), a_1, \sigma(\<a_0,a_1\>),\dots\>
    \]
  Likewise, if $\sigma$ is a strategy for the second player, and 
  $\<a_0,a_1,\dots\>$ is an attack by the first player, then the result is
  the playthrough
    \[
      \<a_0,\sigma(\<a_0\>), a_1, \sigma(\<a_0,a_1\>),\dots\>
    \]
\end{defn}

We now may rigorously define the notion of ``forcing'' a win in a game.

\begin{defn}
  A strategy $\sigma$ is a \term{winning strategy} for a player if for
  every attack by the opponent, the result of the game is a victory 
  for that player.

  If a winning strategy exists for player $\pl A$ in the game $G$, then we
  write $\pl A \win G$. Otherwise, we write $\pl A \not\win G$.
\end{defn}

Of course, a strategy $\sigma$ is not a winning strategy for a player if there 
exists some \term{counter-attack} by the opponent for which the result is 
a victory for the opponent. Typically this counter-attack is defined in
terms of the strategy $\sigma$; else, the counter-attack is itself a
winning strategy depending on only the round number (which we will
investigate further in a later section).

\begin{defn}
  A game $G$ with players $\pl A$, $\pl B$ is said to be \term{determined}
  if either $\pl A \win G$ or $\pl B \win G$.
\end{defn}

(TODO: cite Gale and Stewart)
% http://en.wikipedia.org/wiki/Borel_determinacy_theorem

\begin{thm}
  If the move set $M$ for a game $G=\<M,W\>$ is given the discrete topology
  and $W$ is either an open or closed subset of $M^\omega$ with the
  usual product topology, then $G$ is determined.
\end{thm}

(TODO: outline proof?)

We get this important corollary:

\begin{cor}
  Finite games are determined.
\end{cor}

\begin{proof} 
  For every victory $p\in W$ for the first player, there was some 
  round $n<\omega$ for which that partial playthrough was guaranteed to
  result in a victory for the first player. 

  Put another way, there exists $n<\omega$ such that
    $ 
      \{q : q\geq p\rest n \}
    $
  (a neighborhood of $p$) is contained in $W$. Thus $W$ is open.
\end{proof}

(TODO: find combinatorial proof and cite it)

(TODO: Cover Martin's Borel determinacy theorem?)

\subsection{Applications of Strategies}

\subsection{Limited Information Strategies}


\section{Examples of Topological Games}