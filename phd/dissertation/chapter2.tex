%!TEX root = dissertation.tex
% ^ leave for LaTeXTools build functionality

\chapter{Toplogical Games and Strategies\\ of Perfect and Limited Information}

The goal of this paper is to explore the applications of limited information
strategies in existing topological games. (TODO: History of combinatorical
and topological games?)

\section{Games}

Intuitively, the games studied in this paper are two-player games for which
each player takes turns making a choice from a set of possible moves. At
the conclusion of the game, the choices made by both players are examined,
and one of the players is declared the winner of that playthrough, with
no ties allowed.

Games may be modeled mathematically in various ways (TODO: cite digraph model)
but we will find it convenient to think of them as follows.

\begin{defn}
  A \term{game} is a tuple $\<M,W\>$ such that 
  $W\subseteq (M^{\omega})^2$. $M$ is set of \term{moves} for
  the game, and $(M^{\omega})^2$ is the set of all possible 
  \term{playthroughs} of the game.

  $W$ is the set of \term{winning} playthroughs for the 
  first player, and $(M^{\omega})^2\setminus W$ is the set of winning
  playthroughs for the second player.
\end{defn}

To illustrate this definition, we may formally model the well-known game 
Tic-Tac-Toe. Imagine that the game board is labeled as a ``magic square''
of the numbers $0$ through $8$. (TODO: image) Then writing $X$ or $O$ on
the game board is equivalent to choosing a number $0$ through $8$, and 
getting a ``tic-tac-toe'' is equivalent to having chosen three numbers which 
sum to $12$.

\begin{game}
  Let $\tttgame$ denote the game \term{Tic-Tac-Toe}, with players $\pl X$ 
  and $\pl O$.

  The moves $M$ for the game are the first nine non-negative 
  integers $\{0,\dots,8\}$. The winning playthroughs $W$ for $\pl X$ 
  consist of pairs of sequences $\<X,O\>$ such that $X\geq\<X_0,\dots,X_4\>$,
  $O\geq\<O_0,\dots,O_3\>$, and any of the following hold:
    \begin{itemize}
      \item There exists some $k<4$ such that $\<X_0,O_0,\dots,X_k\>$ contains
            no repeated integers, but $\<X_0,O_0,\dots,X_k,O_k\>$ does.

            ($\pl O$ drew her mark on a space already occupied.)
      \item There exists $i<j<k<5$ such that $\<X_0,O_0,\dots,X_4\>$ contains 
            no repeated integers and $X_i+X_j+X_k=12$, and for all
            $i'<j'<k'<k$ it follows that $O_{i'}+O_{j'}+O_{k'}\not=12$.

            ($\pl X$ did not draw her mark on a space that was already occupied,
            and was able to secure a ``tic-tac-toe'' before $\pl O$.)
            \footnote{
              The observant reader will note that in this formulation of 
              Tic-Tac-Toe, player $\pl O$ wins ``cat's games'' where neither 
              player secures a ``tic-tac-toe'' (since no ties are allowed).
            }
    \end{itemize}
\end{game}

Due to the tedium of defining games explicitly from the definition, we 
typically define games by declaring the \term{rules} that each player must 
follow (e.g. players may not make a mark on a space which was already marked)
and the \term{winning condition} for the first player (e.g. $\pl X$ secures
a ``tic-tac-toe'' before $\pl O$). Then a playthrough is in $W$ if either 
the first player made only \term{legal moves} which observed the game's rules
and the playthrough satisifies the winning condition, or the second player
made an \term{illegal move} which contradicted the game's rules.

Often, players should not be able to choose from the same set of moves
(e.g. player Black cannot move White's pieces in chess), in which case $M$ may
be partitioned into two sets $M_0,M_1$ and the rules for the game may be defined
such that each player may only choose from one of those sets 
(e.g. Black will automatically lose if she moves one of White's pieces).

An artifact of this game model is that games never technically end, since 
playthroughs of the game are infinite sequences. Nonetheless, Tic-Tac-Toe
is always decided by $\pl X$'s fifth move, and in chess, there is also an
upper bound for the length of a game due to the finite number of ways pieces
may be positioned on the chessboard (and the ``threefold rule''
which declares the game finished if the same position is repeated three times).

But a two-player game need not have a bounded number of rounds, as is
demonstrated by this game due to John Conway:

\begin{game}
  The game ...
\end{game}



\subsection{Infinite Games}

\subsection{Topological Games}


\section{Strategies}

\subsection{Perfect Information}

\subsection{Limited Information}


\section{Examples of Topological Games}