%!TEX root = dissertation.tex
% ^ leave for LaTeXTools build functionality

\chapter{Bell's Game}

A very recent development related to Gruenhage's convergence and
clustering games comes from
Jocelyn Bell, used to study the uniform box product of uniform spaces.

\section{Uniform Spaces}

\begin{defn}
  A \term{uniformity} on a set $X$ is a filter $\mb D$ of subsets of $X^2$,
  known as \term{entourages}, such that for each entourage $D\in\mb D$:
  \begin{itemize}
    \item $\Delta=\{\<x,x\>:x\in X\}\subseteq D$
    \item There exists $\frac{1}{2}D\in\mb D$ such that
      \[
        \frac{1}{2}D\circ\frac{1}{2}D
          =
        \left\{\<x,z\>:\exists y\in X\left(\<x,y\>,\<y,z\>\in\frac{1}{2}D\right)\right\}
          \subseteq
        D
      \]
    \item $D^{-1}=\{\<y,x\>:\<x,y\>\in D\}\in \mb D$
  \end{itemize}
\end{defn}

A set $X$ with a uniformity is called a \term{uniform sapce}.
As $\mb D$ is a filter, we also have that $D\cap E\in \mb D$ for all
$E\in\mb D$, and $F\in \mb D$ for all $F\supseteq D$. Note that if $\mb E$ is
a filter base satisfying the conditions for a uniformity, then we say
$\mb E$ is a \term{uniformity base} which may be extended to a uniformity by
closing it under the superset operation.

A uniformity is a generalization of a metric.

\begin{defn}
  For an entourage $D\in\mb D$ and a point $x\in X$, the
  \term{$D$-ball around $x$} is the set $D[x]=\{y:\<x,y\>\in D\}$.
\end{defn}

\begin{defn}
  If $d$ is a metric for the space $X$, then the \term{metric uniformity} for
  $X$ is generated by the uniformity base $\{D_\epsilon:\epsilon>0\}$
  where $D_\epsilon = \{\<x,y\>:d(x,y)<\epsilon\}$.
\end{defn}

Like metrics, uniformities induce natural topological structures.

\begin{defn}
  The \term{uniform topology} for a space $X$ with uniformity $\mb D$ is given
  by letting $U$ be open if for each $x\in U$, there exists $D\in\mathbb{D}$
  such that $D[x]\subseteq U$.
\end{defn}

\begin{thm}
  The uniform topology for a space $X$ with uniformity $\mb D$ is the coarsest
  topology such that $D[x]$ is a neighborhood (not necessarily open)
  of $x$ for each $x\in X$ and $D\in\mathbb{D}$.
\end{thm}

\begin{proof}
  Let $\mc T$ be a topology such that $D[x]$ is a neighborhood
  of $x$ for each $x\in X$, and let $U$ be open in the uniform topology.
  For each $x\in U$, there exists $D_x\in\mathbb{D}$ such that
  $D_x[x]\subseteq U$, and since $D_x[x]$ is a neighborhood of $x$, there
  is $U_x\in \mc T$ such that $x\in U_x\subseteq D_x[x]$. Since
  $U=\bigcup_{x\in U}U_x\in \mc T$, the result follows as $\mc T$ contains the
  uniform topology.
\end{proof}

The uniform topology
for a metric uniformity is simply the usual metric topology, and $D_\epsilon(x)$
is the usual metric $\epsilon$-ball around $x$.

We require a few known results on uniform spaces.

\begin{thm}
  Every uniform topology is $T_{3\frac{1}{2}}$.
\end{thm}

\begin{defn}
  A topology on $X$ is \term{uniformizable} if there exists
  a uniformity whose uniform topology equals the topology on $X$.
\end{defn}

\begin{thm}
  Every $T_{3\frac{1}{2}}$ topology is uniformizable.
\end{thm}

\begin{defn}
  The \term{universal uniformity} for a uniformizable topology is the uniformity
  finer than all uniformities which induce the given topology.
\end{defn}

\begin{thm}
  Every uniformizable topology is induced by its universal uniformity.
\end{thm}

\begin{defn}
  For a uniformizable space $X$, a \term{symmetric open entourage} $D$ is an
  entourage of the universal uniformity such that
  \begin{itemize}
    \item $D$ is open in the product topology on $X^2$
    \item $D$ is symmetric; that is, $D=D^{-1}$.
  \end{itemize}
\end{defn}

\begin{thm}
  For every entourage $D$ in the universal uniformity on $X$, there exists a
  symmetric open entourage $U\subseteq D$.
\end{thm}

If $D$ is a symmetric open entourage, then we assume $\frac{1}{2}D$ is
also a symmetric open entourage.

\section{A game on uniformizable spaces}

\begin{game}
  Let $\bellUniGame{X}$ denote the \term{Bell uniform space game} with players
  $\pl D$, $\pl P$ which
  proceeds as follows for a space $X$ with uniformity $\mb D$. In round $0$,
  $\pl D$ chooses an entourage $D_0$, followed by $\pl P$ choosing a point
  $p_0\in X$. In round $n+1$, $\pl D$ chooses an entourage $D_{n+1}$, followed
  by $\pl P$ choosing a point $p_{n+1}\in D_n[p_n]$.

  $\pl D$ wins in the case that either $\<p_0,p_1,\dots\>$ converges in $X$,
  or $\bigcap_{n<\omega}D_n[p_n] = \emptyset$. $\pl P$ wins otherwise.
\end{game}

This original formulation was used in \cite{MR3239205} to prove that the
$\sigma$-product of spaces for which $\pl D$ has a winning strategy is
collectionwise normal, as well as to show the collectionwise normality
of certain uniform box products.

Considering the universal uniformity, we may
consider a purely topological variations of the game.

\begin{game}
  Let $\bellConHardGame{X}$ denote the \term{hard Bell game} with players $\pl D$,
  $\pl P$ which proceeds as follows for a topological space $X$. In round $0$,
  $\pl D$ chooses a symmetric open entourage $D_0$, followed by $\pl P$
  choosing a point $p_0\in X$. In round $n+1$, $\pl D$ chooses a symmetric
  open entourage $D_{n+1}$, followed
  by $\pl P$ choosing a point $p_{n+1}\in D_n[p_n]$.

  $\pl D$ wins in the case that either $\<p_0,p_1,\dots\>$ converges in $X$,
  or $\bigcap_{n<\omega}D_n[p_n] = \emptyset$. $\pl P$ wins otherwise.
\end{game}

\begin{game}
  Let $\bellConGame{X}$ denote the \term{Bell game} with players $\pl D$,
  $\pl P$ which proceeds analogously to $\bellConHardGame{X}$, except that
  $\pl P$ must ensure $p_{n+1}\in E_n[p_n]$,
  and $\pl D$ wins when $\<p_0,p_1,\dots\>$ converges in $X$
  or $\bigcap_{n<\omega}E_n[p_n] = \emptyset$, where
  $E_n=\bigcap_{m\leq n}D_n$.
\end{game}

With regards to $\pl D$'s perfect information winning strategies, all three
are equivalent.

\begin{thm}
  $\pl D\win\bellConGame{X}$ if and only if $\pl D\win\bellConHardGame{X}$ if and
  only if $\pl D$ has a winning strategy for the proximal game for some
  uniformity inducing the topology on $X$.
\end{thm}

\begin{proof}
  TODO
\end{proof}

The hard and normal variants $\bellConHardGame{X}$, $\bellConGame{X}$ mirror the
$W$ game variants $\gruConGameHard{X}{x}$, $\gruConGame{X}{x}$. In fact, Bell
showed a strong connection in \cite{MR3239205}.

\begin{thm}
  If $\pl D\win\bellConGame{X}$, then $X$ is a $W$-space.
\end{thm}

% \begin{thm}
% For all $x\in X$:
%   \begin{itemize}
%     \item
%       $\pl D\win \bellConGame{X} \Rightarrow \pl O \win \gruConGame{X}{x}$
%     \item
%       $\pl D\ktactwin{2k} \bellConGame{X} \Rightarrow \pl O \ktactwin{k} \gruConGame{X}{x}$
%     \item
%       $\pl D\kmarkwin{2k} \bellConGame{X} \Rightarrow \pl O \kmarkwin{k} \gruConGame{X}{x}$
%   \end{itemize}
% \end{thm}

% \begin{thm}
%   Let $X\cup\{\infty\}$ be a uniformizable space such that $X$ is discrete. Then
%   \begin{itemize}
%     \item
%       $\pl O\win \gruConGame{X\cup\{\infty\}}{\infty} \Leftrightarrow \pl D \win \bellConGame{X\cup\{\infty\}}$
%     \item
%       $\pl O\ktactwin{k} \gruConGame{X\cup\{\infty\}}{\infty} \Leftrightarrow \pl D \ktactwin{k} \bellConGame{X\cup\{\infty\}}$
%     \item
%       $\pl O\kmarkwin{k} \gruConGame{X\cup\{\infty\}}{\infty} \Leftrightarrow \pl D \kmarkwin{k} \bellConGame{X\cup\{\infty\}}$
%   \end{itemize}
% \end{thm}

% \begin{prop} For any $x\in X$ and $k\geq 1$,
%   \begin{itemize}
%     \item
%       $\pl O\ktactwin{k}\gruConGame{X}{x} \Leftrightarrow \pl O\tactwin\gruConGame{X}{x}$
%     \item
%       $\pl O\kmarkwin{k}\gruConGame{X}{x} \Leftrightarrow \pl O\markwin\gruConGame{X}{x}$
%   \end{itemize}
% \end{prop}

% \begin{cor}
%   Let $X\cup\{\infty\}$ be a uniformizable space such that $X$ is discrete, and $k\geq 1$. Then
%   \begin{itemize}
%     \item
%       $\pl D\ktactwin{k}\bellConGame{X\cup\{\infty\}} \Leftrightarrow O\tactwin\bellConGame{X\cup\{\infty\}}$
%     \item
%       $\pl D\kmarkwin{k}\bellConGame{X\cup\{\infty\}} \Leftrightarrow O\markwin\bellConGame{X\cup\{\infty\}}$
%   \end{itemize}
% \end{cor}

% \begin{prop} For any uniform space $X$,
%   \begin{itemize}
%     \item
%       $\pl D\ktactwin{k}\bellConGame{X} \Leftrightarrow \pl D\ktactwin{2}\bellConGame{X}$
%     \item
%       $\pl D\kmarkwin{k}\bellConGame{X} \Leftrightarrow \pl D\kmarkwin{2}\bellConGame{X}$
%   \end{itemize}
% \end{prop}

% \begin{thm}
%   For any uniformly locally compact space $X$,
%       $\pl D\win \bellConGame{X} \Leftrightarrow \pl D\win \bellClusGame{X}$
% \end{thm}

% \begin{thm}
%   For any uniformly locally compact proximal space $X$, $\pl O\win \gruConGame{X}{H}$ for all compact $H\subseteq X$.
% \end{thm}

% \begin{cor}
%   A compact uniform space $X$ is Corson compact if and only if it is proximal.
% \end{cor}

% \begin{thm}
%   $\pl O \prewin \gruConGame{X}{H}$ if and only if there exists a countable base
%   around $H$.
% \end{thm}


% \begin{cor}
%   $X$ is first countable if and only if $\pl O \prewin \gruConGame{X}{x}$ for
%   all $x\in X$
% \end{cor}

% \begin{cor}
%   $\pl D \prewin \bellConGame{X}$ implies $X$ is first countable.
% \end{cor}

% \begin{cor}
%   If $X$ is scattered compact and $\pl O \prewin\gruConGame{X}{x}$ for all
%   $x\in X$ (or $\pl D \prewin\bellConGame{X}$), then $X$ is metrizable.
% \end{cor}


% \begin{thm}
%   If $H$ is a closed subset of $X$, then
%     $
%       \pl D \limitwin \bellConGame{X}
%         \Rightarrow
%       \pl D \limitwin \bellConGame{H}
%     $
%   where $\limitwin$ is any of $\win$, $\ktactwin{k}$, or $\kmarkwin{k}$.
% \end{thm}


% \begin{thm}
%   If $\pl D\limitwin \bellConGame{X_i}$ for $i<\omega$, then
%   $\pl D\limitwin \bellConGame{\prod_{i<\omega}X_i}$, where $\limitwin$ is either
%   $\win$ or $\kmarkwin{k}$.
% \end{thm}

% (TODO: I expect I should be able to do some clever things
% assuming $S(\kappa,\omega,\omega)$ to get a similar result for sigma
% products of dimension $\kappa$.)

% \begin{lem}
%   $\pl O\prewin\gruClusGame{X}{S}$ if and only if $\pl O\prewin\gruConGame{X}{S}$.
% \end{lem}

% \begin{thm}
%   For any predetermined absolutely proximal space $X$,
%   $\pl O\prewin \gruConGame{X}{H}$ for all compact $H\subseteq X$.
% \end{thm}


% \begin{ex}
%   Let $X=I\times 2$ be the Alexandrov double interval. Then
%   $\pl D \not\prewin \bellConGame{X}$, but $\pl D \markwin \bellConGame{X}$.
% \end{ex}

% \begin{thm}
%   For any uniformly locally compact space $X$,
%       $\pl D\prewin \bellConGame{X} \Leftrightarrow \pl D\prewin \bellClusGame{X}$
% \end{thm}

% \begin{prop}
% If $\pl D \prewin \bellConGame{X}$, then $X$ has a $G_\delta$ diagonal.
% \end{prop}


% \begin{ex}
% The Sorgenfrey line $S$ has a $G_\delta$ diagonal but $\pl P\win\bellConGame{S}$.
% \end{ex}

% \begin{cor}
%   For $X$ with uniformity $\mathbb{D}$ inducing the compact Hausdorff topology
%   $\tau$, the following are equivalent:
%     \begin{enumerate}[(a)]
%       \item $\pl D \prewin \bellConGame{X}$
%       \item $\pl D \prewin \bellClusGame{X}$
%       \item $X$ has a $G_\delta$ diagonal
%       \item $\mathbb{D}$ is metrizable
%       \item $\tau$ is metrizable
%     \end{enumerate}
% \end{cor}


% \begin{thm}
%   A uniformly locally compact space with a $G_\delta$ diagonal is metrizable.
% \end{thm}

% \begin{cor}
%   If $X$ is uniformly locally compact, then $\pl D \prewin \bellConGame{X}$
%   implies $X$'s topology is metrizable.
% \end{cor}

% \begin{ex}
%   Let $R$ be the Michael Line. Then $\pl P\win \bellConGame{X}$.
% \end{ex}

% \begin{proof}
%   During round $0$, $\pl P$ may choose $m(0)=0$ and $p(0)=1$,
%   and during round $n+1$,
%   $\pl P$ may choose $m(n+1)>m(n)$ and
%   $p(n+1)=p(n)+\frac{1}{10^{m(n+1)}}$ such that $p$ is a legal attack.

%   It follows that $p$ ``converges'' to
%   $x=\sum_{n<\omega}\frac{1}{10^{m(n)}}$, except $x$ is an irrational
%   number composed of $1$s separated by strings of $0$s of strictly
%   increasing size.
% \end{proof}

% \begin{ex}
%   Let $\omega_1$ be given a ladder topology:
%     \begin{itemize}
%       \item All successor ordinals are isolated.
%       \item Strictly increasing sequences (ladders) $L_\alpha:\omega\to\alpha$
%             are defined for each limit ordinal $\alpha$ such that $L_\alpha$ converges to $\alpha$ in the order topology, and each limit
%             $\alpha$ is given neighborhoods of the form
%             $L(\alpha,m)=\{\alpha\}\cup\{L_\alpha(n):n\geq m\}$.
%       \item $\omega_1=\bigcup_{\alpha\in\omega_1^L} L(\alpha,0)$
%     \end{itemize}

%   Let
%     \[
%       A(\alpha,n)
%         =
%       [L(\alpha,0)\setminus L(\alpha,n)]^1
%         \cup
%       \{\oneptcomp\omega_1 \setminus (L(\alpha,0)\setminus L(\alpha,n))\}
%     \]
%     \[
%       B(\alpha)
%         =
%       \{L(\alpha,0), \oneptcomp\omega_1 \setminus L(\alpha,0)\}
%     \]

%   Finite refinements of $A(\alpha,n)$ and $B(\alpha)$ give partitions
%   witnessing a uniformization of the ladder topology.

%   Then $\bellConGame{\oneptcomp\omega_1}$ is indetermined.
% \end{ex}

% (TODO: finish proof)