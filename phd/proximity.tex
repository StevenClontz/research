\documentclass[11pt]{article}

\pdfpagewidth 8.5in
\pdfpageheight 11in

\setlength\topmargin{0in}
\setlength\headheight{0in}
\setlength\headsep{0.2in}
\setlength\textheight{8in}
\setlength\textwidth{6in}
\setlength\oddsidemargin{0in}
\setlength\evensidemargin{0in}
\setlength\parindent{0.25in}
\setlength\parskip{0.1in} 
 
\usepackage{amssymb}
\usepackage{amsfonts}
\usepackage{amsmath}
\usepackage{mathtools}
\usepackage{amsthm}

\usepackage{enumerate}

      \theoremstyle{plain}
      \newtheorem{theorem}{Theorem}
      \newtheorem{lemma}[theorem]{Lemma}
      \newtheorem{corollary}[theorem]{Corollary}
      \newtheorem{proposition}[theorem]{Proposition}
      \newtheorem{conjecture}[theorem]{Conjecture}
      \newtheorem{question}[theorem]{Question}
      \newtheorem{example}[theorem]{Example}
      
      \theoremstyle{definition}
      \newtheorem{definition}[theorem]{Definition}
      
      \theoremstyle{remark}
      \newtheorem{remark}[theorem]{Remark}


% Strategy uparrow shortcuts
\newcommand{\win}{\uparrow}
\newcommand{\prewin}{\uparrow_{\text{pre}}}
\newcommand{\markwin}{\uparrow_{\text{mark}}}
\newcommand{\tactwin}{\uparrow_{\text{tact}}}
\newcommand{\ktactwin}[1]{\uparrow_{#1\text{-tact}}}
\newcommand{\kmarkwin}[1]{\uparrow_{#1\text{-mark}}}
\newcommand{\codewin}{\uparrow_{\text{code}}}
\newcommand{\limitwin}{\uparrow_{\text{limit}}}

\newcommand{\oneptcomp}[1]{#1\cup\{\infty\}}

\newcommand{\congame}[2]{Con_{O,P}(#1,#2)}
\newcommand{\clusgame}[2]{Clus_{O,P}(#1,#2)}

\newcommand{\lfkpgame}[1]{LF_{K,P}(#1)}
\newcommand{\lfklgame}[1]{LF_{K,L}(#1)}

\newcommand{\pfgame}[1]{PF_{F,C}(#1)}

\newcommand{\mengame}[1]{Cov_{C,S}(#1)}

\newcommand{\sigmaprodr}[1]{\Sigma\mathbb{R}^{#1}}
\newcommand{\sigmaprodtwo}[1]{\Sigma2^{#1}}

\newcommand{\concat}{^\frown}
\newcommand{\rest}{\restriction}

\newcommand{\close}[1]{\overline{#1}}

\newcommand{\<}{\langle}
\renewcommand{\>}{\rangle}

\newcommand{\mc}[1]{\mathcal{#1}}

\newcommand{\ds}{\displaystyle}

\begin{document}

\begin{definition}
  For any partition $\mc R$ of a space $X$ and $x\in X$, let $\mc R[x]$ be such that $x\in\mc R[x]\in\mc R$.

  For partitions $\mc R_0,\dots,\mc R_n$, let $\mc H_n=\bigwedge_{m\leq n} \mc R_m$ be the coarsest partition which refines each $\mc R_m$.

  For partitions $\mc R,\mc S$ let $\mc R\otimes \mc S = \{r\times s: r\in R, s\in S\}$.
\end{definition}

\begin{proposition}
  $x\in \mc R[y] \Leftrightarrow y\in \mc R[x]$.

  $\mc H_n[x]=\left(\bigwedge_{m\leq n} \mc R_m\right)[x]=\bigcap_{m\leq n}\mc R_m[x]$.
\end{proposition}

\begin{definition}
  For zero-dimensional $X$, the proximity game $\proxgame{X}$ proceeds as follows: in round $n$, $\pl R$ chooses a clopen partition $\mc R_n$ of $X$, followed by $\pl P$ choosing a point $p_n\in X$.

  Player $\pl R$ wins if either $\bigcap_{n<\omega} \mc H_n[p_n] = \emptyset$ or $p_n$ converges.
\end{definition}

\begin{proposition}
  This game is perfect-information equivalent to the analogous game studied by Bell, requiring $\pl P$'s play $p_{n+1}$ to be in $\mc H_n[p_n]$ in rounds $n+1$, and requiring $\pl O$ choose refinements.
\end{proposition}

\begin{proof}
  Allowing $\pl P$ to play $p_{n+1}\not\in \mc H_n[p_n]\Rightarrow \mc H_n[p_{n+1}]\not=\mc H_n[p_n]$ does not introduce any new winning plays for $\pl P$ as for any such move, $\bigcap_{m<\omega}\mc H_n[p_n]\subseteq \mc H_{n+1}[p_{n+1}]\cap \mc H_n[p_n]\subseteq \mc H_n[p_{n+1}]\cap \mc H_n[p_n]=\emptyset$.

  Allowing $\pl R$ to play non-refining clopen partitions does not introduce any new winning plays for $\pl R$ as the winning condition relies on the refinement of all $\mc R_n$ anyway.
\end{proof}

\begin{definition}
  A space $X$ is \textbf{proximal} iff $X$ is zero-dimensional and $\pl R\win\proxgame{X}$.
\end{definition}

\begin{definition}
  A space $X$ is \textbf{Mark\"ov proximal} iff $X$ is zero-dimensional and $\pl R\markwin\proxgame{X}$.
\end{definition}

\begin{definition}
  For any space $X$ and a point $x\in X$, the \textbf{$W$-convergence-game} $\congame{X}{x}$ proceeds as follows: in round $n$, $\pl O$ chooses a neighborhood $U_n$ of $x$, followed by $\pl P$ choosing a point $p_n\in X$.

  For open sets $U_0,\dots,U_n$, let $V_n=\bigcap_{m\leq n} U_m$. Player $\pl O$ wins if either $p_n\not\in V_n$ for some $n<\omega$, or if $p_n$ converges.
\end{definition}

\begin{definition}
  A space $X$ is a \textbf{$W$-space} iff $\pl O\win\congame{X}{x}$ for all $x\in X$.
\end{definition}

    % \begin{theorem}
    %    All proximal spaces are $W$-spaces.
    % \end{theorem}

    % \begin{proof}
    %   Let $\sigma$ witness $\pl R\win\proxgame{X}$, and assume without loss of generality that $\mc R_n=\mc H_n$. For each $x\in X$, we define $\tau_x$ for $\pl O$ in $\congame{X}{x}$ to be $\tau_x()=\sigma(x)[x]$ and $\tau_x(p_0,\dots,p_n)=\sigma(x,p_0,x,\dots,p_n,x)[x]$ (so that $\tau_x(p_0,\dots,p_{n-1})=U_n=V_n$). Let $p_0,p_1,\dots$ be an attack against $\tau_x$ by $\pl P$ in $\congame{x}{X}$ such that $p_n\in V_n$ for each $n<\omega$.

    %   We consider the attack $x,p_0,x,p_1,\dots$ against $\sigma$ by $\pl P$ in $\proxgame{X}$. Since the $(2n)$th move by $\pl P$ is $x$, obviously $x\in \mc H_{2n}[x]$ for $\pl C$'s $(2n)$th move $\mc H_{2n}$.  

    %   The $(2n+1)$th move by $\pl C$ is $\mc H_{2n+1}=\sigma(x,p_0,\dots,p_{n-1},x)$ and the $(2n+1)$th move by $\pl P$ is $p_n$, so since 
    %     \[
    %       p_n \in 
    %       V_n =
    %       \tau_x(p_0,\dots,p_{m-1})=
    %       \sigma(x,p_0,x,\dots,p_{m-1},x)[x]=
    %       \mc H_{2n+1}[x]
    %     \]
    %   it follows that $x\in \mc H_{2n+1}[p_n]$.

    %   This shows $x\in \bigcap_{n<\omega} \mc H_n[p_n]\not= \emptyset$ and as $\sigma$ is a winning strategy for $\pl R$ in $\proxgame{X}$, it follows that $x,p_0,x,p_1,\dots$ converges. This implies $p_0,p_1,\dots$ converges to $x$, proving that $\tau_x$ is a winning strategy for $\pl O$ in $\congame{X}{x}$.
    % \end{proof}



\begin{definition}
  For each finite tuple $(m_0,\dots,m_{n-1})$, we define the \textbf{$k$-tactical fog-of-war}
    \[
      T_k(m_0,\dots,m_{n-1})=(m_{n-k},\dots,m_{n-1})
    \]
  and the \textbf{$k$-Mark\"ov fog-of-war}
    \[
      M_k(m_0,\dots,m_{n-1})=(m_{n-k},\dots,m_{n-1},n)
    \]

  So $P\ktactwin{k}G$ if and only if there exists a winning strategy for $P$ of the form $\sigma\circ T_k$, and $P\kmarkwin{k}G$ if and only if there exists a winning strategy of the form $\sigma\circ M_k$.
\end{definition}

\begin{theorem}
For all $x\in X$:
  \begin{itemize}
    \item
      $\pl R\win \proxgame{X} \Rightarrow \pl O \win \congame{X}{x}$
    \item
      $\pl R\prewin \proxgame{X} \Rightarrow \pl O \prewin \congame{X}{x}$
    \item
      $\pl R\ktactwin{2k} \proxgame{X} \Rightarrow \pl O \ktactwin{k} \congame{X}{x}$
    \item
      $\pl R\kmarkwin{2k} \proxgame{X} \Rightarrow \pl O \kmarkwin{k} \congame{X}{x}$
  \end{itemize}
\end{theorem}

\begin{proof}
Let $\sigma$ witness $\pl R \ktactwin{2k}\proxgame{X}$ (resp. $\pl R \kmarkwin{2k}\proxgame{X}$, $\pl R\win\proxgame{X}$). We define the $k$-tactical (resp. $k$-Mark\"ov, perfect info) strategy $\tau$ such that
  \[
    \tau\circ L_k(p_0,\dots,p_{n-1})
      =
    \sigma\circ L_{2k}(x,p_0,\dots,x,p_{n-1})[x]
      \cap
    \sigma\circ L_{2k}(x,p_0,\dots,x,p_{n-1},x)[x]
  \]
where $L_{2k}$ is the $2k$-tactical fog-of-war (resp. $2k$-Mark\"ov fog-of-war, identity) and $L_{k}$ is the $k$-tactical fog-of-war (resp. $k$-Mark\"ov fog-of-war, identity).

Let $p_0,p_1,\dots$ attack $\tau$ such that $p_n\in V_n=\bigcap_{m\leq n}\tau\circ L_k(p_0,\dots,p_{m-1})$ for all $n<\omega$. Consider the attack $q_0,q_1,\dots$ against the winning strategy $\sigma$ such that $q_{2n}=x$ and $q_{2n+1}=p_n$. 

Certainly, $x\in \mc H_{2n}[x]= \mc H_{2n}[q_{2n}]$ for any $n<\omega$. Note also for any $n<\omega$ that 
    \[
      p_n \in 
      V_n =
      \bigcap_{m\leq n}\tau\circ L_k(p_0,\dots,p_{m-1})
    \]
    \[
      =
      \bigcap_{m\leq n}\left(
        \sigma\circ L_{2k}(x,p_0,\dots,x,p_{m-1})[x]\cap
        \sigma\circ L_{2k}(x,p_0,\dots,x,p_{m-1},x)[x]
      \right)
    \]
    \[
      =
      \bigcap_{m\leq n}\left(
        \sigma\circ L_{2k}(q_0,q_1,\dots,q_{2m-2},q_{2m-1})[x]\cap
        \sigma\circ L_{2k}(q_0,q_1,\dots,q_{2m-2},q_{2m-1},q_{2m})[x]
      \right)
    \]
    \[
      \bigcap_{m\leq n}\mc R_{2m}[x]\cap R_{2m+1}[x]=
      \mc H_{2n+1}[x]
    \]
so $x\in\mc H_{2n+1}[p_n]=\mc H_{2n+1}[q_{2n+1}]$. Thus $x\in \bigcap_{n<\omega} \mc H_n[q_n]$, and since $\sigma$ is a winning strategy, the attack $q_0,q_1,\dots$ converges, and must converge to $x$. Thus $p_0,p_1,\dots$ converges to $x$, and $\tau$ is also a winning strategy.
\end{proof}

\begin{corollary}
For all $x\in X$:
  \begin{itemize}
    \item
      $\pl R\ktactwin{k} \proxgame{X} \Rightarrow \pl O \ktactwin{k} \congame{X}{x}$
    \item
      $\pl R\kmarkwin{k} \proxgame{X} \Rightarrow \pl O \kmarkwin{k} \congame{X}{x}$
  \end{itemize}
\end{corollary}

\begin{corollary}
   All proximal spaces are $W$-spaces.
\end{corollary}

\begin{definition}
  In the one-point compactification $\oneptcomp\kappa=\kappa\cup\{\infty\}$ of discrete $\kappa$, define the clopen partition $\mc C(F) = [F]^1\cup\{\oneptcomp{\kappa}\setminus F\}$.
\end{definition}

\begin{theorem}
  $\pl R\codewin\proxgame{\oneptcomp{\kappa}}$
\end{theorem}

\begin{proof}
  Use the coding strategy $\sigma()=\mc C(\emptyset)=\{\oneptcomp\kappa\}$, $\sigma(\mc C(F),\alpha)=\mc C(F\cup\{\alpha\})$ for $\alpha<\kappa$ and $\sigma(\mc C(F),\infty)=\mc C(F)$. Note $\mc R_n=\mc H_n$. For any attack $p_0,p_1,\dots$ against $\sigma$ such that $\bigcap_{n<\omega} \mc H_n[p_n]\not=\emptyset$, suppose 
    \begin{itemize}
      \item $\infty\in\bigcap_{n<\omega} \mc H_n[p_n]$. Then $p_n\in \oneptcomp{\kappa}\setminus\{p_m:m<n\}$ shows that the non-$\infty$ $p_n$ are all distinct. If co-finite $p_n=\infty$, we have $p_n\to\infty$. Otherwise, there are infinite distinct $p_n$, and since neighorhoods of $\infty$ are co-finite, we have $p_n\to\infty$.
      \item $\infty\not\in\ \mc H_N[p_N]$ for some $N<\omega$, so $\alpha\in\bigcap_{n<\omega} \mc H_n[p_n]$ for some $\alpha<\kappa$. Then $\mc H_n[p_n]=\{\alpha\}$ for all $n\geq N$, and thus $p_n\to\alpha$.
    \end{itemize}
  Thus $\sigma$ is a winning coding strategy.
\end{proof}

\begin{theorem}
$\pl O\win \congame{\oneptcomp\kappa}{\infty} \Rightarrow \pl R \win \proxgame{\oneptcomp\kappa}$

$\pl O\prewin \congame{\oneptcomp\kappa}{\infty} \Rightarrow \pl R \prewin \proxgame{\oneptcomp\kappa}$

$\pl O\ktactwin{k} \congame{\oneptcomp\kappa}{\infty} \Rightarrow \pl R \ktactwin{k} \proxgame{\oneptcomp\kappa}$

$\pl O\kmarkwin{k} \congame{\oneptcomp\kappa}{\infty} \Rightarrow \pl R \kmarkwin{k} \proxgame{\oneptcomp\kappa}$
\end{theorem}

\begin{proof}
  Let $\sigma\circ L$ be a winning strategy where $L$ is the identify (resp. a $k$-tactical fog-of-war, a $k$-Mark\"ov fog-of-war).

  Define $\tau\circ L$ such that
    \[
      \tau\circ L(p_0,\dots,p_{n-1})=\mc R(\oneptcomp\kappa\setminus(\sigma\circ L(p_0,\dots,p_{n-1})))
    \]

  For any attack $p_0,p_1,\dots$ against $\tau$ such that $\bigcap_{n<\omega}\mc H_n[p_n]\not=\emptyset$, suppose 
    \begin{itemize}
      \item $\mc H_n[p_n]=\mc H_n[\infty]=\bigcap_{m\leq n}\sigma\circ L(p_0,\dots,p_{m-1})=\bigcap_{m\leq n}U_m=V_n$ for all $n<\omega$. Since $\sigma$ is a winning strategy, the $p_n$ converge at $\infty$.
      \item $\mc H_N[p_N]\not=\mc H_N[\infty]$ for some $N<\omega$. Then $\mc H_N[p_N]=\{p_N\}$, and since $\bigcap_{n<\omega}\mc H_n[p_n]\not=\emptyset$, we have $\mc H_n[p_n]=\mc H_N[p_N]=\{p_N\}\Rightarrow p_n=p_N$ for all $n\geq N$, and the $p_n$ converge at $p_N$.
    \end{itemize}
\end{proof}

\begin{corollary}
$\pl O\win \congame{\oneptcomp\kappa}{\infty} \Leftrightarrow \pl R \win \proxgame{\oneptcomp\kappa}$

$\pl O\prewin \congame{\oneptcomp\kappa}{\infty} \Leftrightarrow \pl R \prewin \proxgame{\oneptcomp\kappa}$

$\pl O\ktactwin{k} \congame{\oneptcomp\kappa}{\infty} \Leftrightarrow \pl R \ktactwin{k} \proxgame{\oneptcomp\kappa}$

$\pl O\kmarkwin{k} \congame{\oneptcomp\kappa}{\infty} \Leftrightarrow \pl R \kmarkwin{k} \proxgame{\oneptcomp\kappa}$
\end{corollary}

\begin{corollary}
$O\prewin \proxgame{\oneptcomp{\omega}}$.

$O\tactwin\proxgame{\oneptcomp{\omega}}$.

$O\not\kmarkwin{k} \proxgame{\oneptcomp{\kappa}}$ for $\kappa\geq\omega_1$.
\end{corollary}

\begin{proof}
Results hold for $\pl O$ and $\congame{\oneptcomp{\kappa}}{\infty}$.
\end{proof}

\begin{definition}
  The \textbf{almost-proximal game} $\aproxgame{X}$ is analogous to $\proxgame{X}$ except that the points $p_n$ need only cluster for $\pl R$ to win the game.
\end{definition}

\begin{definition}
  The \textbf{$W$-clustering game} $\clusgame{X}{x}$ is analogous to $\congame{X}{x}$ except that the points $p_n$ need only cluster at $x$ for $\pl O$ to win the game.
\end{definition}

\begin{proposition}
$\pl O\win \clusgame{\oneptcomp\kappa}{\infty} \Rightarrow \pl R \win \aproxgame{\oneptcomp\kappa}$

$\pl O\prewin \clusgame{\oneptcomp\kappa}{\infty} \Rightarrow \pl R \prewin \aproxgame{\oneptcomp\kappa}$

$\pl O\ktactwin{k} \clusgame{\oneptcomp\kappa}{\infty} \Rightarrow \pl R \ktactwin{k} \aproxgame{\oneptcomp\kappa}$

$\pl O\kmarkwin{k} \clusgame{\oneptcomp\kappa}{\infty} \Rightarrow \pl R \kmarkwin{k} \aproxgame{\oneptcomp\kappa}$
\end{proposition}

\begin{proof}
  Same proof as before, replacing ``converge'' with ``cluster''.
\end{proof}

\begin{corollary}
$\pl R\markwin \aproxgame{\oneptcomp{\omega_1}}$.
\end{corollary}

\begin{proof}
Holds for $\pl O$ and $\clusgame{\oneptcomp{\omega_1}}{\infty}$.
\end{proof}

\begin{proposition}
If $\sigma\circ L$ is a winning strategy for $\pl R$ in $\proxgame{X}$ (resp. $\aproxgame{X}$) where $L$ is the identity (or a $k$-tactical fog-of-war or a $k$-Mark\"ov fog-of-war), and $C$ is a closed subspace of $X$, then
  \[
    \tau\circ L(p_0,\dots,p_{n-1}) = C\cap \sigma\circ L(p_0,\dots,p_{n-1})
  \]
defines a winning strategy $\tau\circ L$ for $\pl R$ in $\proxgame{X}$ (resp. $\aproxgame{X}$).
\end{proposition}

\begin{proof}
For any attack $p_0,p_1,\dots$ against $\tau\circ L$ in $\proxgame{C}$ (resp. $\aproxgame{C}$), note $p_0,p_1,\dots$ is also an attack against $\sigma\circ L$ in $\proxgame{X}$ (resp. $\aproxgame{X}$). 

If $\pl R$ wins in $\proxgame{X}$ (resp. $\aproxgame{X}$) by $\mc H^\sigma_n[p_n]=\emptyset$, then note that $\mc H^\tau_n[p_n]\subseteq\mc H^\sigma_n[p_n]=\emptyset$. 

If If $\pl R$ wins in $\proxgame{X}$ (resp. $\aproxgame{X}$) because the $p_n$ converge (resp. cluster), then they converge (resp. cluster) in the closed set $C$. 

Either way, $\tau\circ L$ defeats the arbitrary attack and is thus a winning strategy.
\end{proof}

\begin{proposition}
If for any $i<m<\omega$, $\sigma_i\circ L$ is a winning strategy for $\pl R$ in $\proxgame{X_i}$ (resp. $\aproxgame{X_i}$) where $L$ is the identity (or a $k$-tactical fog-of-war or a $k$-Mark\"ov fog-of-war), then
  \[
    \tau\circ L(p_0,\dots,p_{n-1}) = \bigotimes_{i<m}\sigma_i\circ L(p_0(i),\dots,p_{n-1}(i))
  \]
defines a winning strategy $\tau\circ L$ for $\pl R$ in $\proxgame{\prod_{i<m}X_i}$ (resp. $\aproxgame{\prod_{i<m}X_i}$).
\end{proposition}

\begin{proof}
For any attack $p_0,p_1,\dots$ against $\tau\circ L$ in $\proxgame{\prod_{i<m}X_i}$ (resp. $\aproxgame{\prod_{i<m}X_i}$), note that for any $i<m$, $p_0(i),p_1(i),\dots$ is an attack against $\sigma_i\circ L$ in $\proxgame{X_i}$ (resp. $\aproxgame{X}$).

If for some $i<m$, $\pl R$ defeats the attack $p_0(i),p_1(i),\dots$ because $\bigcap_{n<\omega}\mc H^i_n[p_n(i)]=\emptyset$, then we see immediately that $\bigcap_{n<\omega} \mc H_n[p_n]=\emptyset$ and $\tau$ defeats the attack $p_0,p_1,\dots$.

Otherwise for all $i<m$, we have $p_n(i)$ converging (resp. clustering) at some $x_i\in X$. It follows then that $p_0,p_1,\dots$ converges (resp. clusters) at $x=\<x_i:i<m\>$ and $\tau$ defeats the attack $p_0,p_1,\dots$.
\end{proof}

\begin{definition}
For $H\subseteq X$, the \textbf{$W$-subset-convergence-game} $\congame{X}{H}$ is analogous to $\congame{X}{x}$: $\pl O$ chooses open neighborhoods of $H$ and tries to force $p_n\to H$.
\end{definition}

\begin{theorem}
  For all compact $H\subseteq X$, $\pl R \win\proxgame{X}$ implies $\pl O \win\congame{X}{H}$.
\end{theorem}

\begin{proof}
  Adapted from G's proof.

  Let $\sigma$ witness $\pl R \win\proxgame{X}$, assuming $\sigma(p)$ refines $\sigma(q)$ whenever $q\subseteq p$.

  For certain finite sequences of points $p\in X^{<\omega}$, we define a tree of finite sequences $\<T(p),\subseteq\>$ as follows:

  \begin{itemize}
    \item $T(\emptyset)$ contains the empty sequence, and for each of the finite nonempty
      \[
        V\in\{U\cap H: U\in\sigma(\emptyset)\}
      \]
      choose a unique $h_V\in V$ and include $\<h_V\>$ in $T(\emptyset)$.
    \item Assume that whenever $T(p)$ is defined, it satisfies the following:
      \begin{itemize}
        \item $T(p)$ is finite
        \item $p'\subseteq p \Rightarrow T(p')\subseteq T(p)$
        \item If $\<h_0,q_0,\dots,h_n\>\in T(p)$ then $\<q_0,\dots,q_{n-1}\>$ is a subsequence of $p$ and $q_i\in\sigma(h_0,q_0,\dots,h_{i-1},q_{i-1})[h_i]$ for all $i<n$
        \item For each sequence $t\concat\<h,q\>\in T(p)$ and for each of the finite nonempty
          \[
            V\in\{U\cap H\cap \sigma(t)[h]: U\in \sigma(t\concat\<h,q\>)\}
          \]
          there is a unique $h_V\in V$ such that $t\concat\<h,q,h_V\>\in T(p)$.
        \item $\{\sigma(t)[h]:t\concat\<h\>\text{ is maximal in }T(p)\}$ partitions $\st{\bigwedge_{s\in T(p)}\sigma(s)}{H}$.
      \end{itemize}

    \item Then when $T(p)$ is defined, we define $T(p\concat\<q\>)$ for each $q\in \st{\bigwedge_{s\in T(p)}\sigma(s)}{H}$ as follows:
      \begin{itemize}
        \item Assume $T(p)\subseteq T(p\concat\<q\>)$.
        \item Find the maximal $t_q\concat\<h_q\>$ in $T(p)$ such that $q\in\sigma(t_q)[h_q]$. Include $t_q\concat\<h_q,q\>$ in $T(p\concat\<q\>)$.
        \item For each of the finite nonempty
          \[
            V\in\mc V(t_q,h_q,q)=\{U\cap H\cap \sigma(t_q\concat\<h_q,q\>)[h]: U\in \sigma(t_q\concat\<h_q,q\>)\}
          \]
          choose a unique $h_V\in V$ and include $t_q\concat\<h_q,q,h_V\>$ in $T(p\concat\<q\>)$.
        \item Note that 
          \[
            \{\sigma(t)[h]:t\concat\<h\>\text{ is maximal in }T(p), h\not= h_q\}
          \]
          partitions 
          \[
            \st{\bigwedge_{s\in T(p)}\sigma(s)}{H}\setminus\sigma(t_q)[h_q] = \st{\bigwedge_{s\in T(p\concat\<q\>)}\sigma(s)}{H}\setminus\sigma(t_q)[h_q]
          \]
          and that
          \[
            \{\sigma(t_q\concat\<h_q,q\>)[h_V]: \mc V\in V(t_q,h_q,q)\}
          \]
          partitions 
          \[
            \st{\bigwedge_{V\in \mc V(t_q,h_q,q)}\sigma(t_q\concat\<h_q,q,h_V\>)}{H}\cap \sigma(t_q)[h_q] 
              = 
            \st{\bigwedge_{s\in T(p\concat\<q\>)}\sigma(s)}{H}\cap\sigma(t_q)[h_q]
          \]
          so our definition satisfies the recursion hypotheses.
      \end{itemize}
  \end{itemize}

  We may define a strategy $\tau$ for $\pl O$ in $\congame{X,H}$ as follows. Let $\tau(\emptyset)=\st{\bigwedge_{s\in T(\emptyset)}\sigma(s)}{H}$. If $T(p)$ is defined and $q\in \st{\bigwedge_{s\in T(p)}\sigma(s)}{H}$, then let $\tau(p\concat\<q\>)=\st{\bigwedge_{s\in T(p\concat\<q\>)}\sigma(s)}{H}$ (and $\tau(p\concat\<q\>)=X$ otherwise).

  Let $p\in X^\omega$ attack $\tau$ such that $p(n)\in\tau(p\rest n)$ always. It follows that $T(p\rest n)$ is defined for all $n<\omega$, so let $T_p=\bigcup_{n<\omega}T(p\rest n)$. By definition, it is evident that $T_p$ is an infinite tree with finite levels, so choose an infinite branch $p'=\<h_0,q_0,\dots\>$.

  Since $p'$ is an attack on $\sigma$, and $p'(n+1)\in\sigma(p\rest n+1)[p(n)]$ always, it follows that $p'$ converges. Since $p(2n)=h_n\in H$, $p'$ converges in $H$, and so does its subsequence $p''=\<q_0,q_1,\dots\>$, which is also a subsequence of $p$.

  We've shown $p$ clusters in $H$, and since $\tau(p\rest n+1)\subseteq \tau(p)$, it follows analogously to a result of G that $p$ converges in $H$.
\end{proof}

\begin{corollary}
If $X$ is compact and $\pl R\win\proxgame{X}$, then $\pl O\win\congame{X^2}{\Delta}$, and thus $X$ is Corson compact.
\end{corollary}

\begin{proof}
Note $\pl R\win\proxgame{X^2}$ and $\Delta$ is a compact subset of $X^2$, so $\pl O\win\congame{X^2}{\Delta}$. By a result of G, $X$ is Corson compact.
\end{proof}











\end{document}