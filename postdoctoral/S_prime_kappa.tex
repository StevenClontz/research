\documentclass[11pt]{article}

\usepackage{../clontzStyle}
\usepackage{../clontzDefinitions}

\usepackage{tikz}
\usetikzlibrary{matrix}

\begin{document}

  (joint work with Alan Dow)

  \begin{definition}
    Two functions \(f,g\) are almost compatible if
    \(\{a\in\dom f\cap\dom g:f(a)\not=g(a)\}\) is finite.
  \end{definition}

  \begin{definition}
    \(S'(\theta)\) states that there exists a cofinal family
    \(\mc S\subseteq[\theta]^{\omega}\) and a collection of pairwise
    almost compatible
    finite-to-one functions \(\{f_S\in\omega^S:S\in\mc S\}\)
  \end{definition}

  \begin{definition}
    \(S(\theta)\) strengthens \(S'(\theta)\) by requiring the collection
    to contain one-to-one functions.
  \end{definition}

  We wish to show that Scheeper's original \(S(\theta)\) is strictly stronger
  than \(S'(\theta)\).

  \begin{definition}
    A topological space is said to be \(\omega\)-bounded if each countable
    subset of the space has compact closure.
  \end{definition}


  % TODO: read through Alan's proof and fit to rest of document
          \begin{theorem}
           For each $n\in \omega$, there is a locally countable,
            $\omega$-bounded topology on $\omega_n$. Note
            that this means that the closure of any set has
            the same cardinality and weight as the set.
          \end{theorem}


          To prove the theorem, we must actually prove a stronger lemma.

          \begin{lemma}
          Assume that $X$ has cardinality at most $\omega_n$ (for
          any $n\in \omega$),
          and is locally countable, locally compact, and the closure
          of each set has the same cardinality as the set.
          Then $X$ has an $\omega$-bounded extension with the
          same properties.
          \end{lemma}

            \begin{proof}
            We prove this by induction on $n$.
            In fact, we make our inductive statement that  if $\tilde X$
            is the extension of $X$, then  $\tilde X \setminus X$ also
            has cardinality $\omega_n$.
             If $n=0$,
              then we can just take the free union of two copies of $X$
              and then the one-point compactification.
             So suppose $n>0$ and
            that $X$ is such a topology on the ordinal $\omega_n$.
            For each $\alpha<\omega_n$, the closure of the initial
            segment $\alpha$ is bounded by some $\gamma_\alpha$.
            Also, because $X$ is locally countable, $\gamma_\alpha$
            can be chosen so that $\alpha$ is contained  in the interior
            of $\gamma_\alpha$. There is a cub  $C\subset \omega_n$
            with the property that for each $\delta\in C$ and $\alpha<\delta$,
             $\gamma_\alpha$ is also less than $\delta$. This implies
             that for each $\delta\in C$, the initial segment $\delta$ is
             open, and if $\delta$ has uncountable cofinality, then
              $\delta$ is clopen.

              The proof will be easier to visualize if we now
              identify the points of $X$ with the point set
               $\omega_n\times \{0\}$ and we will add the points
                $\omega_n\times \{1\}$ to create the extension.
             By induction on $\lambda \in C$ we define a topology
             on $\omega_n\times\{0\}\cup \lambda\times \{1\}$ so that
             $\omega_n \times \{0\}$ is an open subset. We also ensure,
             by induction, for each $\alpha<\lambda$,
             the closure of $\alpha\times 2$ is an
              $\omega$-bounded  subset of $\lambda\times 2$.

          In the case that $n=1$, then  choose any sequence
           $\lambda_n : n\in \omega$ increasing cofinal in $\lambda$.
           If $\lambda $ is a limit in $C$, then we simply take the
           topology we have constructed so far on $\lambda\times 2$
           and there's nothing more needs to be done. Otherwise
          we may assume that $\lambda_0$ is the predecessor
          of $\lambda\in C$ and
           we set $Y_\lambda $ to equal the countable set
            $\overline{\lambda}\setminus \lambda$. For convenience,
             and with no loss, we assume that $\lambda$ itself is a limit
             of limits.
           And we have a topology on
           $$\lambda_0\times 2 \cup (\lambda \cup Y_\lambda)\times\{0\}~.$$
            Recursively
            choose clopen sets $U_n$ in this topology
             so that $\lambda_0\times 2\subset U_0$,
             $U_{n}\cup \lambda_{n+1}\times \{0\}$ is contained
              $ U_{n+1}$ while $U_{n+1}$ is disjoint from $Y_\lambda$.
              It is easy to see that we can have all the points
              in $(\lambda\setminus \{\lambda_n : n\in\omega\})\times \{1\}$
              be isolated, and arrange that $(\lambda_n,1) $ is the
              point at infinity in the one-point compactification
              $U_n\cup (\lambda_n\times \{1\})$.

                Now we handle the case $n>1$ and we can shrink $C$
                and now assume
                that $C$ is the closure of $\{\lambda\in C : \mbox{cf}(\lambda)>\omega\}$.
                We again proceed by induction on $\lambda\in C$.  If
                $\lambda$ is a limit in $C$, then there is nothing to do: we
                simply have defined an appropriate topology on
                $\omega_n\times  \{0\}\cup \lambda\times \{1\}$ so that
                for each $\mu\in C\cap \lambda$ with $\mbox{cf}(\mu)>\omega$,
                 $\mu\times 2$ is a clopen $\omega$-bounded subspace.
                In case $\lambda$ is not a limit of $C$, then $\lambda$
                has uncountable cofinality and a predecessor $\mu\in C$.
                We therefore have that $\lambda\times\{0\}$ is clopen
                in $\omega_n\times \{0\}$.
                 We apply the induction hypothesis
                to the space $\lambda\times \{0\} \cup \mu\times 2$ to choose
                the topology on $\lambda\times 2$.


            \end{proof}

  \begin{definition}
    A Kurepa family \(\mc K\subseteq[\theta]^{\omega}\) on \(\theta\)
    satisfies that
    \(\mc K\rest A=\{K\cap A:K\in\mc K\}\) is countable
    for each \(A\in[\theta]^\omega\).
  \end{definition}

  \begin{corollary}
    There exists a Kurepa family cofinal in \([\omega_k]^\omega\)
    for each \(k<\omega\).
  \end{corollary}

  \begin{proof}
    This is actually a corollary of an observation of Todorcevic communicated
    by Dow in [TODO cite Gen Prog in Top I]:
    if every Kurepa family of size at most \(\theta\)
    extends to a cofinal Kurepa family, then the same is true of \(\theta^+\).
    So the result follows as
    every Kurepa family \(\mc K\) of size \(\omega\) extends to
    the cofinal Kurepa family \([\bigcup\mc K]^\omega\).

    We may alternatively
    obtain the result from the previous topological argument by using the family
    \(\mc K\) of compact sets in the constructed topology on
    \(\omega_k\) as our witness. Of course, every Lindel\"of set in
    a locally countable space is countable. Thus \(\mc K\)
    is cofinal in \([\omega_k]^\omega\)
    since for every countable set \(A\), \(\cl A\) is compact and countable.
    It is Kurepa since for every countable set \(A\), let (TODO)
  \end{proof}

  \begin{theorem}
    \(S'(\theta)\) holds whenever there exists a cofinal Kurepa family on \(\theta\).
  \end{theorem}

  \begin{proof}
    Let \(k<\omega\), and \(\mc K=\{K_\alpha:\alpha<\kappa\}\)
    be a cofinal Kurepa family on \(\theta\).
    We should define \(f_\alpha:K_\alpha\to\omega\) for each \(\alpha<\kappa\).

    Suppose we've defined pairwise almost compatible
    \(\{f_\beta:\beta<\alpha\}\). To define
    \(f_\alpha\), we first recall that \(\mc K\rest K_\alpha\) is countable,
    so we may choose \(\beta_n<\alpha\) for \(n<\omega\) such that
    \(
      \{K_\beta:\beta<\alpha\}\rest K_\alpha \setminus \{\emptyset\}
        =
      \{K_\alpha\cap K_{\beta_n}:n<\omega\}
    \).
    Let \(K_\alpha=\{\delta_{i,j}:i\leq\omega,j<w_i\}\) where
    \(w_i\leq\omega\) for each \(i\leq\omega\),
    \(
      K_\alpha\cap \left(K_{\beta_n}\setminus\bigcup_{m<n}K_{\beta_m}\right)
        =
      \{\delta_{n,j}:j<w_n\}
    \),
    and
    \(
      K_\alpha\setminus\bigcup_{n<\omega}K_{\beta_n}
        =
      \{
        \delta_{\omega,j}:j<w_\omega
      \}
    \).
    Then let \(f_\alpha(\delta_{n,j})=\max(n,f_{\beta_n}(\delta_{n,j}))\) for
    \(n<\omega\) and \(f_\alpha(\delta_{\omega,j})=j\) otherwise.

    We should show that \(f_\alpha\) is finite-to-one. Let \(n<\omega\).
    We need only worry about \(\delta_{m,j}\) for \(m\leq n\) since
    \(f_\alpha(\delta_{m,j})\geq m\). Since each
    \(f_{\beta_m}\) is finite-to-one, \(f_{\beta_m}(\delta_{m,j})\leq n\)
    for only finitely many \(j\). Thus \(f_\alpha\) maps to
    \(n\) only finitely often.

    We now want to demonstrate that \(f_\alpha\sim f_{\beta_n}\) for all
    \(n<\omega\). We again need only concern ourselves with \(\delta_{m,j}\) for
    \(m\leq n\) since otherwise \(\delta_{m,j}\not\in K_{\beta_n}\).
    For \(m=n\), we have
    \(f_\alpha(\delta_{n,j})=\max(n,f_{\beta_n}(\delta_{n,j}))\) which differs
    from \(f_{\beta_n}(\delta_{n,j})\) for only the finitely many \(j\) which
    are mapped below \(n\) by \(f_{\beta_n}\).
    For \(m<n\) and \(\delta_{m,j}\in K_{\beta_n}\),we have
    \(f_\alpha(\delta_{m,j})=\max(m,f_{\beta_m}(\delta_{m,j}))\) which can
    only differ
    from \(f_{\beta_n}(\delta_{m,j})\) for only the finitely many \(j\) which
    are mapped below \(m\) by \(f_{\beta_m}\) or the finitely many \(j\)
    for which the
    almost compatible \(f_{\beta_n}\sim f_{\beta_m}\) differ.
  \end{proof}

  \begin{corollary}
    \(S'(\omega_k)\) holds for all \(k<\omega\).
  \end{corollary}

  As noted in [TODO cite Dow],
  Jensen's one-gap two-cardinal theorem under \(V=L\) [TODO cite] can be used
  to show that there exist cofinal Kurepa families on every cardinal.

  \begin{corollary}[\(V=L\)]
    \(S'(\theta)\) holds for all cardinals.
  \end{corollary}

  In particular, \(S(\omega_2)\) fails under \(CH\), showing the two are
  distinct. Actually, \(CH\) is not required to have \(S(\omega_2)\) fail.


  % TODO: read through Alan's proof and fit to rest of document
        We are going to need a technical lemma (available in Kunen).
        \bigskip

        \begin{lemma}
        Assume that $G\subset \operatorname{Fn}(\omega_2,2)$ is a generic filter,
         and let $\mu\in \omega_2$. Then the final model $V[G]$ can be
         regarded as forcing  with $\operatorname{Fn}(\omega_2\setminus \mu,
         2)$ over the model $V[G_\mu]$.
        In addition, for each $\operatorname{Fn}(\omega_2,2)$-name $\dot A$
        of a subset of $\omega$ (treat as a subset of $\omega\times
        \operatorname{Fn}(\omega_2,2) $),
        there is a canonical name $\dot A(G_\mu)$ where,
        $$\dot A(G_\mu) = \{ (n,p\restriction [\mu,\omega_2))  :
         (n,p)\in \dot A\ \ \mbox{and} \ \ p\restriction \mu\in G_\mu\}$$
        and we get that the valuation of $\dot A(G_\mu)$ by the tail
        of the generic, $G_{\omega_2\setminus \mu}$, is the same as
        the valuation of $\dot A$ by the full generic.
        \end{lemma}


        \begin{theorem}
          If we add $\omega_2$ Cohen reals to a model of CH we get
        that Scheepers' $S(\omega_2)$ (still) fails.
        \end{theorem}

        \begin{proof}
        The forcing poset is $\operatorname{Fn}(\omega_2, 2)$.
        Let $\{ \dot f_{A} : A\in [\omega_2]^\omega\}$ be a family of
        names such that $\dot f_{A}$ is a one-to-one function from $A$ into
        $\omega$. It suffices to only consider sets $A$ from the ground
        model.
        \bigskip

        Put all the lemma  stuff in an elementary submodel $M$ of the universe
        (technically of $H(\kappa)$,  or of $V_\kappa$,
         for some large $\kappa$). Standard methods says that we can assume that
         $|M|=\omega_1 =\mathfrak c$ and that $M^\omega\subset M$ (which means
         that every countable subset of $M$ is a member of $M$).
        \bigskip

        Let $\lambda = M\cap \omega_2$ (same as the supremum of $M\cap
        \omega_2$). Consider the name  $\dot f_{[\lambda,\lambda+\omega)}$.
        What is such a name?  We can assume that it is a set of pairs
         of the form $( (\lambda+k,m), p)$ where $p\in \mathop{Fn}(\omega_2,
         2)$ and, of course, $k,m\in \omega$. This is (almost) equivalent to
         saying
         that $p$ forces that $\dot f_{[\lambda,\lambda+\omega)}(\lambda+k) =
         m$. We don't take all such $p$, in fact for each $k,m$ it is enough
        to take a
         countable set of such $p$ to get an equivalent name
         (Kunen calls it a nice name if we take, for each $k,m$ an
        antichain that is maximal among such conditions).
        Given any such name $\dot f$, let $\operatorname{supp}(\dot f)$
        denote the union of the domains of conditions $p$ appearing in the
        name.



         Also let $Y$ equal $\operatorname{supp}(\dot
         f_{[\lambda,\lambda+\omega)})\setminus \lambda$.
         Let $\delta$ denote the order type of
         $Y$ and let the 2-parameter notation
         $\varphi_{\mu,\lambda}$ be the order-preserving function from
        $\mu\cup Y$ onto  the ordinal $\mu+\delta$.  This lifts canonically to
        an order-preserving bijection $\varphi_{\mu,\lambda}:
        \operatorname{Fn}(\mu\cup Y,2) \mapsto
        \operatorname{Fn}(\mu+\delta,2)$.
        Similarly, we make sense of the name
         $\varphi_{\mu,\lambda}(\dot f_{[\lambda,\lambda+\omega)})$, call it $F_M$.
        Here simply, for each tuple $(~(k,m)~, p)\in \dot
        f_{[\lambda,\lambda+\omega)}$,
         we have that $(~(k,m)~,\varphi_{\mu,\lambda}(p))$ is in $F_M$.
        Again, let $\varphi_{\mu,\lambda}(\dot f_{[\lambda,\lambda+\omega)})$
        be interpreted in the above sense as giving $F_M$ (which is an element
        of $M$).  Note that we do not regard $\delta$ as fixed here, but
        rather simply depending on the $\operatorname{supp}(\dot
        f_{[\lambda,\lambda+\omega)})$ described above. Other values
        replacing $\lambda>\mu$ will result in their own set $Y$
        and canonical map $\varphi_{\mu,\lambda}$; but one thing we do have to
        assume (or arrange) for other values $\alpha$ replacing $\lambda$
        is that $\operatorname{supp}(\dot f_{[\alpha,\alpha+\omega)})$
        intersected with
        $\alpha$ is contained in $\mu$.

        Now the object $F_M$ is an
        element of $M$, and $M$ believes this statement is true:
        $$
        (\forall \beta\in\omega_2)~ (\exists \beta<\lambda\in \omega_2)~~~
        \operatorname{supp}(\dot f_{[\lambda,\lambda+\omega)})
        \cap \lambda\subset \mu \ \ \mbox{and}\ \
        F_M = \varphi_{\mu,\lambda}(\dot f_{[\lambda,\lambda+\omega)})$$


        But now, this  means that,
        not only is there  an $\alpha\in M$,
        $ F_M = \varphi_{\mu,\alpha}(\dot f_{[\alpha,\alpha+\omega)})$
        but also that there is an increasing sequence
         $\{\alpha_\xi : \xi  \in \omega_1\}\subset \lambda$ of
         such $\alpha$'s satisfying that, for each $\xi$
        we have that $\operatorname{supp}(\dot
        f_{[\alpha_\xi,\alpha_\xi+\omega)})$ is contained in $\alpha_{\xi+1}$.

        Choose such a sequence.
         This means that if we let $A = \bigcup_{n>0} [\alpha_n,\alpha_n+\omega)$ we
         have the name $\dot f_A$ in $M$. This then means that all
         the $( (\beta,m) , p)$ appearing in $\dot f_A$ have the property
         that $\mathop{dom}(p)$ is contained in $M$.
        There is, within $M$, a name $\dot g$ satisfying that
         $\dot f_A(\alpha_n+k) = \dot f_{[\alpha_n,\alpha_n+\omega)}(\alpha_n+k)$
        for all $k >\dot g(n)$.

        \bigskip

        We now apply the above Lemma using $\mu = \mu_0$ and we are now
        working in the extension $V[G_\mu]$. We will abuse the notation
        and use $\dot f_{[\alpha_n,\alpha_n+\omega)}$ instead of
          $\dot f_{[\alpha_n,\alpha_n+\omega)}(G_{\mu})$ as defined in
          the Lemma.
        We work for a contradiction. Something special has now happened,
         namely,  the supports of the names $\{
        \dot f_{[\alpha_n, \alpha_n+\omega)} :  0< n<\omega\}$ are pairwise disjoint
        and also disjoint from the support of the name
         $\dot f_{[\lambda,\lambda+\omega)}$ (under the same convention about
         $G_\mu$.  And not only that, these names are pairwise isomorphic (in
         the way that they all map to $F_M$).


        \bigskip

        Since $A$ is disjoint from $[\lambda,\lambda+\omega)$,
        there must be an integer $\ell$
        together with a condition
         $q\in \mathop{Fn}(\omega_2\setminus \mu,2)$ satisfying that for all
         $n>\ell$ , $q$ forces that

        \centerline{
         ``if $k>\dot g(n)$ (since  $\alpha_n+k\in A$) then
         $\dot f_{[\alpha_n,\alpha_n+\omega)}(\alpha_n+k) \neq
        \dot f_{[\lambda,\lambda+\omega)}(\lambda +k)$''.}

        Choose $n$ large enough so that $\mathop{dom}(q) \cap [\alpha_n,
        \mu_{n+1})$ is
        empty.
         Choose $q_1<q\restriction \lambda$ (in $M$) so that
        $$ \varphi_{\mu,\alpha_n}(q_1\restriction \operatorname{supp}(
         \dot f_{[\alpha_n,\alpha_n+\omega)}) =
        \varphi_{\mu,\lambda}(q\restriction \operatorname{supp}(
         \dot f_{[\lambda,\lambda+\omega)}) $$
        and then (again in $M$) choose $q_2 < q_1$
        so that it both forces a value
         $L$ on $\ell+\dot g(n)$
        and subsequently forces a value $m$ on
         $\dot f_{[\alpha_n, \alpha_n+\omega)}(\alpha_n+L+1)$.
        But now, again calculate
        $$ q_3 = \varphi_{\mu,\lambda}^{-1} \circ \varphi_{\mu,\alpha_n}
        (q_2\restriction \operatorname{supp}(\dot
        f_{[\alpha_n,\alpha_n+\omega)}))$$
        and, by the isomorphisms, we have that $q_3$ forces that
         $\dot f_{[\lambda,\lambda+\omega)}(\lambda+L+1) = m$.

        Technically (or with more care)  all of this is taking place in the
        poset $\operatorname{Fn}(\omega_2\setminus \mu,2)$ and this means
        that $q_3$ and $q$ are all compatible with each other.


        Follow
        the bouncing ball: it suffices to consider
         $q(\beta)=e$ and to assume that $q_3(\beta)$ is defined.
        Since $q_3(\beta)$ is defined, we have that
        there is a  $\beta'\in\mathop{dom}(q_2)$ such that $
         \varphi_{\mu,\lambda}(\beta) = \varphi_{\mu,\alpha_n}(\beta')$,
        and that $q_3(\beta) = q_2(\beta')$.
        But, by definition of $q_1$, $\beta'\in\mathop{dom}(q_1)$
        and even  that $q_1(\beta') = q(\beta)$.
         Then, since $q_2<q_1$, we have that $q_2(\beta')=q_1(\beta') =
         q(\beta)$. This completes the circle that $q_3(\beta) = q(\beta)$.
        \bigskip

        Finally, our contradiction is that $q_3\cup q_2\cup q$
        forces that
         $k=L+1$ violates the quoted statement above.
        \end{proof}

  On the other hand, it's also consistent that \(S'(\theta)\) can fail.

  \begin{theorem}
    There's a model where \(S'(\omega_\omega)\)
    fails? (TODO: get Alan to send this argument.)
  \end{theorem}

  \begin{question}
    Is \(S'(\theta)\) equivalent to having a Kurepa family on \(\theta\)?
  \end{question}

  \section*{Applications!}

  \begin{figure}[h]
\begin{center}
\begin{tikzpicture}
  \matrix (m) [matrix of math nodes,row sep=3em,column sep=1em,minimum width=2em]
  {
    &
    \pl F \kmarkwin2\menGame{\oneptlind\kappa} &
    \pl F \kmarkwin2\schFillIntGame\kappa \\

    S'(\kappa) &
    \pl F \kmarkwin2\schFillInitialGame\kappa &
    \pl F \kmarkwin2\schFillGame\kappa \\

    &
    &
    \pl F \ktactwin2\schFillStrictGame\kappa \\
  };
  \path[>=latex,->]
    (m-2-1) edge (m-1-2)
    (m-2-1) edge (m-2-2)
    (m-2-1) edge (m-3-3)
    (m-1-2) edge (m-2-2)
    (m-1-3) edge (m-2-3)
    (m-3-3) edge (m-2-3);
  \path[>=latex,<->]
    (m-1-2) edge (m-1-3)
    (m-2-2) edge (m-2-3);
\end{tikzpicture}
\end{center}
\caption{Diagram of Scheeper/Menger game implications with \(S'(\kappa)\)}
\label{GamesDiagram2}
\end{figure}

  \begin{theorem}
    Figure \ref{GamesDiagram2} holds. (Proven in [TODO cite])
    (Actually, TODO double-check that it works with just S', particularly
    the strict game)
  \end{theorem}

  It was left open if these implications can be reversed. The answer is
  consistently no.

  \begin{theorem}
    Let \(\alpha\) be the limit of increasing ordinals \(\beta_n\) for \(n<\omega\).
    If \(\pl F \kmarkwin2\schFillIntGame{\omega_{\beta_n}}\) for all
    \(n<\omega\), then \(\pl F\kmarkwin2\schFillIntGame{\omega_\alpha}\).
  \end{theorem}

  \begin{proof}
    Let \(\sigma_n\) be a winning \(2\)-mark for \(\pl F\) in
    \(\schFillIntGame{\omega_{\beta_n}}\). Define the \(2\)-mark \(\sigma\)
    for \(\pl F\) in \(\schFillIntGame{\omega_\alpha}\) as follows:
    \[
      \sigma(\<C\>,0)
        =
      \sigma_0(\<C\cap\omega_{\beta_0}\>,0)
    \]
    \[
      \sigma(\<C,D\>,n+1)
        =
      \sigma_{n+1}(\<D\cap\omega_{\beta_{n+1}}\>,0)
        \cup
      \bigcup_{m\leq n}
      \sigma_m(\<C\cap\omega_{\beta_m},D\cap\omega_{\beta_m}\>,n-m+1)
    \]

    Let \(\<C_0,C_1,\dots\>\) be an attack by
    \(\pl C\) in \(\schFillIntGame{\omega_\alpha}\), and
    \(\alpha\in\bigcap_{n<\omega}C_n\).
    Choose \(N<\omega\) with \(\alpha<\omega_{\beta_{N+1}}\). Consider the
    attack
    \(\<C_{N+1}\cap\omega_{\beta_{N+1}},C_{N+2}\cap\omega_{\beta_{N+1}},\dots\>\)
    by \(\pl C\) in \(\schFillIntGame{\omega_{\beta_{N+1}}}\). Since
    \(\sigma_{N+1}\) is a winning strategy and
    \(\alpha\in\bigcap_{n<\omega}C_{N+n+1}\cap\omega_{\beta_{N+1}}\), either
    \(\alpha\in\sigma_{N+1}(\<C_{N+1}\cap\omega_{\beta_{N+1}}\>,0)\) and thus
    \(\alpha\in\sigma(\<C_N,C_{N+1}\>,N+1)\), or
    \(
      \alpha
        \in
      \sigma_{N+1}(
        \<C_{N+M+1}\cap\omega_{\beta_{N+1}},C_{N+M+2}\cap\omega_{\beta_{N+1}}\>
        ,M+1
      )
    \)
    for some \(M<\omega\) and thus
    \(
      \alpha
        \in
      \sigma(
        \<C_{N+M+1},C_{N+M+2}\>,
        N+M+2
      )
    \). Thus \(\sigma\) is a winning strategy.
  \end{proof}

  \begin{theorem}
    Let \(\alpha\) be the limit of increasing ordinals \(\beta_n\) for \(n<\omega\).
    If \(\pl F \kmarkwin2\schFillInitialGame{\omega_{\beta_n}}\) for all
    \(n<\omega\), then \(\pl F\kmarkwin2\schFillInitialGame{\omega_\alpha}\).
  \end{theorem}

  \begin{proof}
    Let \(\sigma_n\) be a winning \(2\)-mark for \(\pl F\) in
    \(\schFillInitialGame{\omega_{\beta_n}}\). Define the \(2\)-mark \(\sigma\)
    for \(\pl F\) in \(\schFillInitialGame{\omega_\alpha}\) as follows:
    \[
      \sigma(\<C\>,0)
        =
      \sigma_0(\<C\cap\omega_{\beta_0}\>,0)
    \]
    \[
      \sigma(\<C,D\>,n+1)
        =
      \sigma_{n+1}(\<D\cap\omega_{\beta_{n+1}}\>,0)
        \cup
      \bigcup_{m\leq n}
      \sigma_m(\<C\cap\omega_{\beta_m},D\cap\omega_{\beta_m}\>,n-m+1)
    \]

    Let \(\<C_0,C_1,\dots\>\) be an attack by
    \(\pl C\) in \(\schFillInitialGame{\omega_\alpha}\), and
    \(\alpha\in C_0\).
    Choose \(N<\omega\) with \(\alpha<\omega_{\beta_{N+1}}\). Consider the
    attack
    \(\<C_{N+1}\cap\omega_{\beta_{N+1}},C_{N+2}\cap\omega_{\beta_{N+1}},\dots\>\)
    by \(\pl C\) in \(\schFillInitialGame{\omega_{\beta_{N+1}}}\). Since
    \(\sigma_{N+1}\) is a winning strategy and
    \(\alpha\in C_{N+1}\cap\omega_{\beta_{N+1}}\), either
    \(\alpha\in\sigma_{N+1}(\<C_{N+1}\cap\omega_{\beta_{N+1}}\>,0)\) and thus
    \(\alpha\in\sigma(\<C_N,C_{N+1}\>,N+1)\), or
    \(
      \alpha
        \in
      \sigma_{N+1}(
        \<C_{N+M+1}\cap\omega_{\beta_{N+1}},C_{N+M+2}\cap\omega_{\beta_{N+1}}\>
        ,M+1
      )
    \)
    for some \(M<\omega\) and thus
    \(
      \alpha
        \in
      \sigma(
        \<C_{N+M+1},C_{N+M+2}\>,
        N+M+2
      )
    \). Thus \(\sigma\) is a winning strategy.
  \end{proof}

  \begin{corollary}
    It is consistent that \(S'(\omega_\omega)\) fails,
    but as \(S'(\omega_k)\) holds for all \(k<\omega\), we have
    \(\pl F\kmarkwin2\schFillIntGame{\omega_\omega}\) and
    \(\pl F\kmarkwin2\schFillInitialGame{\omega_\omega}\).
  \end{corollary}

  A tricky topological question: does \(\pl F\win\menGame{X}\) imply
  \(\pl F\kmarkwin2\menGame{X}\)?
  (C showed that )
  Under \(V=L\), we cannot hope to find
  a counterexample using \(X=\oneptlind\kappa\) since
  \(S'(\kappa)\) and thus \(\pl F\kmarkwin2\schFillIntGame{\kappa}\)
  always holds.

  \begin{definition}
    Let \(R_\omega\) be the real numbers with the topology of the usual
    open intervals with countably many elements removed.
  \end{definition}

  \begin{theorem}
    \(\pl F\win\menGame{R_\omega}\).
    If there exists a Kurepa family on the reals, then
    \(\pl F\kmarkwin2\menGame{R_\omega}\).
  \end{theorem}

\end{document}