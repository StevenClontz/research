\documentclass[11pt]{article}

\usepackage{../clontzStyle}
\usepackage{../clontzDefinitions}

\begin{document}

  \begin{remark}
    Scheeper's \(S(\kappa)\) requiring injections is stronger than my
    \(S'(\kappa)\) requiring finite-to-one maps. Dow suggests that
    \(S'(\omega_\omega)\) holds in ZFC by the following.
  \end{remark}

  \begin{definition}
    A topological space is said to be \(\omega\)-bounded if each countable
    subset of the space has compact closure.
  \end{definition}

  \begin{theorem}
    For each \(k<\omega\) there exists a topology on \(\omega_k\) which
    is \(\omega\)-bounded and locally countable.
  \end{theorem}

  \begin{proof}
    Assume we've defined an \(\omega\)-bounded topology for \(\omega_k\) such that
    for each \(\gamma\in\omega_k\) there exists a decreasing countable base
    \(\{W_{\gamma,n}:n<\omega\}\) such that each
    \(W_{\gamma,n}\) is countable, compact,
    and contains no ordinals greater than \(\gamma\).
    Note that the usual linear order on \(\omega_1\) satisfies this requirement.

    Let \(\alpha<\omega_{k+1}\), and suppose we've defined compatible topologies
    satisfying the induction hypothesis on
    \(\omega_k\cdot (\beta+1)\) for all \(0\leq\beta<\alpha\).
    If \(\alpha=\beta+1\), then let
    \(\omega_k\cdot(\alpha+1)=\omega_k\cdot(\beta+2)\) be the topological sum
    of the previously defined \(\omega_k\cdot(\beta+1)\) and the
    previously defined
    \(\omega_k\cdot(\beta+2)\setminus\omega_k\cdot(\beta+1)\cong\omega_k\).
    Similarly, if \(cf(\alpha)>\omega\), then let \(\omega_k\cdot(\alpha+1)\)
    be the topological sum of \(\bigcup_{\beta<\alpha}\omega_k\cdot(\beta+1)\)
    and the previously defined
    \(\omega_k\cdot(\alpha+1)\setminus\omega_k\cdot\alpha\cong\omega_k\).
    In either case, \(\omega_k\cdot(\alpha+1)\setminus\omega_k\cdot\alpha\)
    is a clopen copy of \(\omega_k\) under the homeomorphism
    \(\gamma\mapsto\omega_k\cdot\alpha+\gamma\). Then for each
    \(\gamma\in\omega_k\) we may define the witnessing decreasing countable base
    \(\{W_{\omega_k\cdot\alpha+\gamma,n}:n<\omega\}\) for
    \(\omega_k\cdot\alpha+\gamma\) by
    \(
      W_{\omega_k\cdot\alpha+\gamma,n}
        =
      \{\omega_k\cdot\alpha+\delta:\delta\in W_{\gamma,n}\}
    \). Note that for \(cf(\alpha)>\omega\),
    \(\bigcup_{\beta<\alpha}\omega_k\cdot(\beta+1)\) is \(\omega\)-bounded
    since any countable set is contained in some \(\omega_k\cdot(\beta+1)\);
    thus in either case \(\omega_k\cdot(\alpha+1)\) is the topological
    sum of two \(\omega\)-bounded spaces and therefore itself \(\omega\)-bounded.

    The remaining case is where \(\alpha\) is the limit of increasing
    \(\alpha_n\) for \(n<\omega\). Fix a bijection
    \(
      f_\alpha
        :
      \omega_k
        \to
      \omega_k\cdot\alpha
    \). Let
    \(\gamma\in\omega_k\) and define
    \[
      W_{\omega_k\cdot\alpha+\gamma,n}
        =
      \{\omega_k\cdot\alpha+\delta:\delta\in W_{\gamma,n}\}
        \cup
      f_\alpha[W_{\gamma,0}]
        \setminus
      \omega_k\cdot(\alpha_n+1)
    \]
    to be the countable decreasing
    base \(\{W_{\omega_k\cdot\alpha+\gamma,n}:n<\omega\}\)
    for \(\omega_k\cdot\alpha+\gamma\).
    Note that \(\omega_k\cdot(\alpha+1)\setminus\omega_k\cdot\alpha\) is
    then a closed (but not open) copy of \(\omega_k\) under the homeomorphism
    \(\gamma\mapsto\omega_k\cdot\alpha+\gamma\).

    We wish to demonstrate the induction hypothesis for
    \(\omega_k\cdot(\alpha+1)\). Each
    \(W_{\omega_k\cdot\alpha+\gamma,n}\) is countable and contains no
    ordinals greater than \(\omega_k\cdot\alpha+\gamma\). To see that it is
    compact, note that \(\{\omega_k\cdot\alpha+\delta:\delta\in W_{\gamma,n}\}\)
    is a copy of compact \(W_{\gamma,n}\), and for any basic neighborhood
    \(W_{\omega_k\cdot\alpha+\gamma,m}\)
    of \(\omega_k\cdot\alpha+\gamma\), the set
    \(
      f_\alpha[W_{\gamma,0}]
        \cap
      \omega_k\cdot(\alpha_m+1)
    \)
    is a countable subspace of \(\omega_k\cdot(\alpha_m+2)\)




    % If \(\gamma\in\omega_k\cdot(\alpha+1)\setminus\omega_k\cdot\alpha\) where
    % \(cf(\alpha)\not=\omega\), then we immediately see that it is in a clopen
    % copy of \(\omega_k\) giving the result immediately. Otherwise,
    % \(\alpha\) is the limit of increasing
    % \(\alpha_n\) for \(n<\omega\), and we claim that
    % \(
    %   \{\omega_k\cdot\alpha+\delta:\delta\in W\}
    %     \cup
    %   f_\alpha[W]
    %     \setminus
    %   \omega_k\cdot(\alpha_n+1)
    % \) is compact. Since \(W\) is compact in \(\omega_k\) and
    % \(\omega_k\cdot(\alpha+1)\setminus\omega_k\cdot\alpha\) is
    % a closed linear copy of \(\omega_k\), we get that
    % \(\{\omega_k\cdot\alpha+\delta:\delta\in W\}\) is compact.

    % Let \(C\) be a countable subset of \(\omega_k\cdot(\alpha+1)\).
    % In the case that \(\alpha=\beta+1\), we may use the \(\omega\)-boundedness
    % of each part in the clopen partition \(\omega_k\cdot(\beta+1)\) and
    % \(\omega_k\cdot(\beta+2)\setminus\omega_k\cdot(\beta+1)\cong\omega_k\)
    % to conclude that the closure of \(C\) is compact.
    % Similarly, if \(cf(\alpha)>\omega\),
    % then we may use the  \(\omega\)-boundedness
    % of each part in the clopen partition
    % \(\bigcup_{\beta<\alpha}\omega_k\cdot(\beta+1)\) and
    % \(\omega_k\cdot(\alpha+1)\setminus\omega_k\cdot\alpha\cong\omega_k\)
    % to conclude that the closure of \(C\) is compact.

    % The remaining case is again where \(\alpha\) is the limit of increasing
    % \(\alpha_n\) for \(n<\omega\). Then there exists \(\gamma\in\omega_k\)
    % and a countable neighborhood \(W\subseteq[0,\gamma]\) of \(\gamma\)
    % such that
    % \(
    %   C
    %     \subseteq
    %   \{\omega_k\cdot\alpha+\delta:\delta\in W\}
    %     \cup
    %   f_\alpha[W]
    % \).
    % Its closure is compact:
    % the closure operation does not add any ordinals greater than
    % \(\omega_k\cdot\alpha+\gamma\), and any open cover
    % contains another basic open neighborhood of \(\omega_k\cdot\alpha+\gamma\)
    % such as
    % \(
    %   \{\omega_k\cdot\alpha+\delta:\delta\in V\}
    %     \cup
    %   f_\alpha[V]
    %     \setminus
    %   \omega_k\cdot(\alpha_m+1)
    % \)
    % which misses only the compact set
    % \(\{\omega_k\cdot\alpha+\delta:\delta\in W\setminus V\}\) and the
    % closure of the countable set
    % % \(
    % %   f_\alpha[[\omega_k\cdot\alpha,\gamma]]
    % %     \cap
    % %   \omega_k\cdot(\alpha_{\min(m,n)}+1)
    % % \), which is compact by the \(\omega\)-boundedness of
    % % \(\omega_k\cdot(\alpha_{\min(m,n)}+1)\).

    % Finally, since every countable subset of \(\omega_{k+1}\) is contained in
    % some \(\omega_k\cdot(\alpha+1)\), we conclude \(\omega_{k+1}\) is
    % \(\omega\)-bounded.
  \end{proof}

  \begin{theorem*}
    \(S'(\omega_\omega)\).
  \end{theorem*}

\end{document}