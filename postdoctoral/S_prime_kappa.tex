\documentclass[11pt]{article}

\usepackage{../clontzStyle}
\usepackage{../clontzDefinitions}

\begin{document}

  (joint work with Alan Dow)

  \begin{definition}
    Two functions \(f,g\) are almost compatible if
    \(\{a\in\dom f\cap\dom g:f(a)\not=g(a)\}\) is finite.
  \end{definition}

  \begin{definition}
    \(S'(\theta)\) states that there exists a cofinal family
    \(\mc S\subseteq[\theta]^{\omega}\) and a collection of pairwise
    almost compatible
    finite-to-one functions \(\{f_S\in\omega^S:S\in\mc S\}\)
  \end{definition}

  \begin{definition}
    \(S(\theta)\) strengthens \(S'(\theta)\) by requiring the collection
    to contain one-to-one functions.
  \end{definition}

  We wish to show that Scheeper's original \(S(\theta)\) is strictly stronger
  than \(S'(\theta)\).

  \begin{definition}
    A topological space is said to be \(\omega\)-bounded if each countable
    subset of the space has compact closure.
  \end{definition}


  % TODO: read through Alan's proof and fit to rest of document
          \begin{theorem}
           For each $n\in \omega$, there is a locally countable,
            $\omega$-bounded topology on $\omega_n$. Note
            that this means that the closure of any set has
            the same cardinality and weight as the set.
          \end{theorem}


          To prove the theorem, we must actually prove a stronger lemma.

          \begin{lemma}
          Assume that $X$ has cardinality at most $\omega_n$ (for
          any $n\in \omega$),
          and is locally countable, locally compact, and the closure
          of each set has the same cardinality as the set.
          Then $X$ has an $\omega$-bounded extension with the
          same properties.
          \end{lemma}

            \begin{proof}
            We prove this by induction on $n$.
            In fact, we make our inductive statement that  if $\tilde X$
            is the extension of $X$, then  $\tilde X \setminus X$ also
            has cardinality $\omega_n$.
             If $n=0$,
              then we can just take the free union of two copies of $X$
              and then the one-point compactification.
             So suppose $n>0$ and
            that $X$ is such a topology on the ordinal $\omega_n$.
            For each $\alpha<\omega_n$, the closure of the initial
            segment $\alpha$ is bounded by some $\gamma_\alpha$.
            Also, because $X$ is locally countable, $\gamma_\alpha$
            can be chosen so that $\alpha$ is contained  in the interior
            of $\gamma_\alpha$. There is a cub  $C\subset \omega_n$
            with the property that for each $\delta\in C$ and $\alpha<\delta$,
             $\gamma_\alpha$ is also less than $\delta$. This implies
             that for each $\delta\in C$, the initial segment $\delta$ is
             open, and if $\delta$ has uncountable cofinality, then
              $\delta$ is clopen.

              The proof will be easier to visualize if we now
              identify the points of $X$ with the point set
               $\omega_n\times \{0\}$ and we will add the points
                $\omega_n\times \{1\}$ to create the extension.
             By induction on $\lambda \in C$ we define a topology
             on $\omega_n\times\{0\}\cup \lambda\times \{1\}$ so that
             $\omega_n \times \{0\}$ is an open subset. We also ensure,
             by induction, for each $\alpha<\lambda$,
             the closure of $\alpha\times 2$ is an
              $\omega$-bounded  subset of $\lambda\times 2$.

          In the case that $n=1$, then  choose any sequence
           $\lambda_n : n\in \omega$ increasing cofinal in $\lambda$.
           If $\lambda $ is a limit in $C$, then we simply take the
           topology we have constructed so far on $\lambda\times 2$
           and there's nothing more needs to be done. Otherwise
          we may assume that $\lambda_0$ is the predecessor
          of $\lambda\in C$ and
           we set $Y_\lambda $ to equal the countable set
            $\overline{\lambda}\setminus \lambda$. For convenience,
             and with no loss, we assume that $\lambda$ itself is a limit
             of limits.
           And we have a topology on
           $$\lambda_0\times 2 \cup (\lambda \cup Y_\lambda)\times\{0\}~.$$
            Recursively
            choose clopen sets $U_n$ in this topology
             so that $\lambda_0\times 2\subset U_0$,
             $U_{n}\cup \lambda_{n+1}\times \{0\}$ is contained
              $ U_{n+1}$ while $U_{n+1}$ is disjoint from $Y_\lambda$.
              It is easy to see that we can have all the points
              in $(\lambda\setminus \{\lambda_n : n\in\omega\})\times \{1\}$
              be isolated, and arrange that $(\lambda_n,1) $ is the
              point at infinity in the one-point compactification
              $U_n\cup (\lambda_n\times \{1\})$.

                Now we handle the case $n>1$ and we can shrink $C$
                and now assume
                that $C$ is the closure of $\{\lambda\in C : \mbox{cf}(\lambda)>\omega\}$.
                We again proceed by induction on $\lambda\in C$.  If
                $\lambda$ is a limit in $C$, then there is nothing to do: we
                simply have defined an appropriate topology on
                $\omega_n\times  \{0\}\cup \lambda\times \{1\}$ so that
                for each $\mu\in C\cap \lambda$ with $\mbox{cf}(\mu)>\omega$,
                 $\mu\times 2$ is a clopen $\omega$-bounded subspace.
                In case $\lambda$ is not a limit of $C$, then $\lambda$
                has uncountable cofinality and a predecessor $\mu\in C$.
                We therefore have that $\lambda\times\{0\}$ is clopen
                in $\omega_n\times \{0\}$.
                 We apply the induction hypothesis
                to the space $\lambda\times \{0\} \cup \mu\times 2$ to choose
                the topology on $\lambda\times 2$.


            \end{proof}

  \begin{definition}
    A Kurepa family \(\mc K\subseteq[\theta]^{\omega}\) on \(\theta\)
    satisfies that
    \(\mc K\rest A=\{K\cap A:K\in\mc K\}\) is countable
    for each \(A\in[\theta]^\omega\).
  \end{definition}

  \begin{corollary}
    There exists a Kurepa family cofinal in \([\omega_k]^\omega\)
    for each \(k<\omega\).
  \end{corollary}

  \begin{proof}
    This is actually a corollary of an observation of Todorcevic communicated
    by Dow in [TODO cite Gen Prog in Top I]:
    if every Kurepa family of size at most \(\theta\)
    extends to a cofinal Kurepa family, then the same is true of \(\theta^+\).
    So the result follows as
    every Kurepa family \(\mc K\) of size \(\omega\) extends to
    the cofinal Kurepa family \([\bigcup\mc K]^\omega\).

    We may alternatively
    obtain the result from the previous topological argument by using the family
    \(\mc K\) of compact sets in the constructed topology on
    \(\omega_k\) as our witness. Of course, every Lindel\"of set in
    a locally countable space is countable. Thus \(\mc K\)
    is cofinal in \([\omega_k]^\omega\)
    since for every countable set \(A\), \(\cl A\) is compact and countable.
    It is Kurepa since for every countable set \(A\), let (TODO)
  \end{proof}

  \begin{theorem}
    \(S'(\theta)\) holds whenever there exists a cofinal Kurepa family on \(\theta\).
  \end{theorem}

  \begin{proof}
    Let \(k<\omega\), and \(\mc K=\{K_\alpha:\alpha<\kappa\}\)
    be a cofinal Kurepa family on \(\theta\).
    We should define \(f_\alpha:K_\alpha\to\omega\) for each \(\alpha<\kappa\).

    Suppose we've defined pairwise almost compatible
    \(\{f_\beta:\beta<\alpha\}\). To define
    \(f_\alpha\), we first recall that \(\mc K\rest K_\alpha\) is countable,
    so we may choose \(\beta_n<\alpha\) for \(n<\omega\) such that
    \(
      \{K_\beta:\beta<\alpha\}\rest K_\alpha \setminus \{\emptyset\}
        =
      \{K_\alpha\cap K_{\beta_n}:n<\omega\}
    \).
    Let \(K_\alpha=\{\delta_{i,j}:i\leq\omega,j<w_i\}\) where
    \(w_i\leq\omega\) for each \(i\leq\omega\),
    \(
      K_\alpha\cap \left(K_{\beta_n}\setminus\bigcup_{m<n}K_{\beta_m}\right)
        =
      \{\delta_{n,j}:j<w_n\}
    \),
    and
    \(
      K_\alpha\setminus\bigcup_{n<\omega}K_{\beta_n}
        =
      \{
        \delta_{\omega,j}:j<w_\omega
      \}
    \).
    Then let \(f_\alpha(\delta_{n,j})=\max(n,f_{\beta_n}(\delta_{n,j}))\) for
    \(n<\omega\) and \(f_\alpha(\delta_{\omega,j})=j\) otherwise.

    We should show that \(f_\alpha\) is finite-to-one. Let \(n<\omega\).
    We need only worry about \(\delta_{m,j}\) for \(m\leq n\) since
    \(f_\alpha(\delta_{m,j})\geq m\). Since each
    \(f_{\beta_m}\) is finite-to-one, \(f_{\beta_m}(\delta_{m,j})\leq n\)
    for only finitely many \(j\). Thus \(f_\alpha\) maps to
    \(n\) only finitely often.

    We now want to demonstrate that \(f_\alpha\sim f_{\beta_n}\) for all
    \(n<\omega\). We again need only concern ourselves with \(\delta_{m,j}\) for
    \(m\leq n\) since otherwise \(\delta_{m,j}\not\in K_{\beta_n}\).
    For \(m=n\), we have
    \(f_\alpha(\delta_{n,j})=\max(n,f_{\beta_n}(\delta_{n,j}))\) which differs
    from \(f_{\beta_n}(\delta_{n,j})\) for only the finitely many \(j\) which
    are mapped below \(n\) by \(f_{\beta_n}\).
    For \(m<n\) and \(\delta_{m,j}\in K_{\beta_n}\),we have
    \(f_\alpha(\delta_{m,j})=\max(m,f_{\beta_m}(\delta_{m,j}))\) which can
    only differ
    from \(f_{\beta_n}(\delta_{m,j})\) for only the finitely many \(j\) which
    are mapped below \(m\) by \(f_{\beta_m}\) or the finitely many \(j\)
    for which the
    almost compatible \(f_{\beta_n}\sim f_{\beta_m}\) differ.
  \end{proof}

  \begin{corollary}
    \(S'(\omega_k)\) holds for all \(k<\omega\).
  \end{corollary}

  As noted in [TODO cite Dow],
  Jensen's one-gap two-cardinal theorem under \(V=L\) [TODO cite] can be used
  to show that there exist cofinal Kurepa families on every cardinal.

  \begin{corollary}[\(V=L\)]
    \(S'(\theta)\) holds for all cardinals.
  \end{corollary}

  In particular, \(S(\omega_2)\) fails under \(CH\), showing the two are
  unique. Actually, \(CH\) is not required to have \(S(\omega_2)\) fail.


  % TODO: read through Alan's proof and fit to rest of document
  \begin{theorem}
    Adding \(\omega_2\) Cohen reals to a model of \(CH\) forces
    \(\mf c=\omega_2\) and \(\neg S(\omega_2)\).
  \end{theorem}

  \begin{proof}
    TODO add Alan's proof
  \end{proof}

\end{document}