\documentclass[11pt]{article}

\usepackage{amssymb}
\usepackage{amsfonts}
\usepackage{amsmath}
\usepackage{mathtools}
\usepackage{amsthm}

\usepackage[letterpaper,margin=1in]{geometry}

\usepackage{enumerate}

      \theoremstyle{plain}
      \newtheorem{theorem}{Theorem}
      \newtheorem{lemma}[theorem]{Lemma}
      \newtheorem{corollary}[theorem]{Corollary}
      \newtheorem{proposition}[theorem]{Proposition}
      \newtheorem{conjecture}[theorem]{Conjecture}
      \newtheorem{question}[theorem]{Question}
      \newtheorem{claim}[theorem]{Claim}

      \theoremstyle{definition}
      \newtheorem{definition}[theorem]{Definition}
      \newtheorem{example}[theorem]{Example}
      \newtheorem{observation}[theorem]{Observation}
      \newtheorem{game}[theorem]{Game}

      \theoremstyle{remark}
      \newtheorem{remark}[theorem]{Remark}

      \theoremstyle{plain}
      \newtheorem*{theorem*}{Theorem}
      \newtheorem*{lemma*}{Lemma}
      \newtheorem*{corollary*}{Corollary}
      \newtheorem*{proposition*}{Proposition}
      \newtheorem*{conjecture*}{Conjecture}
      \newtheorem*{question*}{Question}
      \newtheorem*{claim*}{Claim}

      \theoremstyle{definition}
      \newtheorem*{definition*}{Definition}
      \newtheorem*{example*}{Example}
      \newtheorem*{observation*}{Observation}
      \newtheorem*{game*}{Game}

      \theoremstyle{remark}
      \newtheorem*{remark*}{Remark}

\title{Two mark in DS game}
\author{Steven Clontz}

\usepackage{../clontzDefinitions}

\newcommand{\bmPoGame}[2]{BM_{po}(#1,#2)}

\begin{document}

\maketitle

Let 
\(
  [f,F,\epsilon]
    =
  \{g\in C_p(X):|g(x)-f(x)|<\epsilon\text{ for all }x\in F\}
\).

\begin{game}
  Let \(G\) be the following game. During round \(n\), player \(\plI\)
  chooses \(\beta_n<\omega_1\), and player \(\plII\) chooses
  \(F_n\in[\omega_1]^{<\aleph_0}\). \(\plII\) wins if
  whenever \(\gamma<\beta_n\) for co-finitely many \(n<\omega\),
  \(\gamma\in F_n\) for infinitely many \(n<\omega\).
\end{game}

\begin{proposition}
  \(\plII \kmarkwin{2} G\).
\end{proposition}

\begin{theorem}
  \(\plI\kmarkwin{2}DS(C_p(\omega_1+1))\)
\end{theorem}
\begin{proof}
  Let \(\sigma\) be a winning 2-mark for \(\plII\) in \(G\).

  Given a point \(f\in C_p(\omega_1+1)\),
  let \(\alpha_f<\omega_1\) satisfy \(f(\beta)=f(\gamma)\) for all
  \(\alpha_f\leq\beta\leq\gamma\leq\omega_1\).

  Let \(\tau(\emptyset,0)=[\mathbf{0},\{\omega_1\},4]\),
  \(\tau(\<f\>,1)=[\mathbf{0};\sigma(\<\alpha_f\>,0)\cup\{\omega_1\};2]\), and
  \[\tau(\<f,g\>,n+2)=[\mathbf{0};\sigma(\<\alpha_f,\alpha_g\>,n+1)\cup\{\omega_1\};2^{-n}].\]

  Let \(\nu\) be a legal attack by \(\plII\) against \(\sigma\).
  For \(\beta\leq\omega_1\), if \(\beta<\alpha_{\nu(n)}\)
  for co-finitely many \(n<\omega\), then
  \(\beta\in\sigma(\<\alpha_{\nu(n)},\alpha_{\nu(n+1)}\>)\) for
  infinitely-many \(n<\omega\), and thus \(0\in\operatorname{cl}\{\nu(n)(\beta):n<\omega\}\).
  Otherwise \(\beta\geq\alpha_{\nu(n)}\) for infinitely many \(n<\omega\),
  and thus \(0\in\operatorname{cl}\{\nu(n)(\beta):n<\omega\}\) as well.
  Thus \(\mathbf{0}\in\operatorname{cl}\{\nu(n):n<\omega\}\).
\end{proof}

  \bibliographystyle{plain}
  \bibliography{../bibliography}


\end{document}
