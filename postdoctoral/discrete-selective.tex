\documentclass[11pt]{article}

\usepackage{amssymb}
\usepackage{amsfonts}
\usepackage{amsmath}
\usepackage{mathtools}
\usepackage{amsthm}

\usepackage[letterpaper,margin=1in]{geometry}

\usepackage{enumerate}

      \theoremstyle{plain}
      \newtheorem{theorem}{Theorem}
      \newtheorem{lemma}[theorem]{Lemma}
      \newtheorem{corollary}[theorem]{Corollary}
      \newtheorem{proposition}[theorem]{Proposition}
      \newtheorem{conjecture}[theorem]{Conjecture}
      \newtheorem{question}[theorem]{Question}
      \newtheorem{claim}[theorem]{Claim}

      \theoremstyle{definition}
      \newtheorem{definition}[theorem]{Definition}
      \newtheorem{example}[theorem]{Example}
      \newtheorem{observation}[theorem]{Observation}
      \newtheorem{game}[theorem]{Game}

      \theoremstyle{remark}
      \newtheorem{remark}[theorem]{Remark}

      \theoremstyle{plain}
      \newtheorem*{theorem*}{Theorem}
      \newtheorem*{lemma*}{Lemma}
      \newtheorem*{corollary*}{Corollary}
      \newtheorem*{proposition*}{Proposition}
      \newtheorem*{conjecture*}{Conjecture}
      \newtheorem*{question*}{Question}
      \newtheorem*{claim*}{Claim}

      \theoremstyle{definition}
      \newtheorem*{definition*}{Definition}
      \newtheorem*{example*}{Example}
      \newtheorem*{observation*}{Observation}
      \newtheorem*{game*}{Game}

      \theoremstyle{remark}
      \newtheorem*{remark*}{Remark}

\title{Two mark in DS game}
\author{Steven Clontz}

\usepackage{../clontzDefinitions}

\newcommand{\bmPoGame}[2]{BM_{po}(#1,#2)}

\begin{document}

\maketitle

Let 
\(
  [f,F,\epsilon]
    =
  \{g\in C_p(X):|g(x)-f(x)|<\epsilon\text{ for all }x\in F\}
\).

\begin{game}
  Let \(G\) be the following game. During round \(n\), player \(\plI\)
  chooses \(\beta_n<\omega_1\), and player \(\plII\) chooses
  \(F_n\in[\omega_1]^{<\aleph_0}\). \(\plII\) wins if
  whenever \(\gamma<\beta_n\) for co-finitely many \(n<\omega\),
  \(\gamma\in F_n\) for infinitely many \(n<\omega\).
\end{game}

For \(f\in\omega^\alpha\), let 
\(f^{\leftarrow}[n]=\{\beta<\alpha:f(\beta)<n\}\).

\begin{proposition}
  \(\plII \kmarkwin{2} G\).
\end{proposition}
\begin{proof}
  Let \(\{f_\alpha\in\omega^\alpha:\alpha<\omega_1\}\) be a collection
  of pairwise almost-compatible finite-to-one functions.
  Define a \(2\)-mark \(\sigma\) for \(\plII\) by
  \[\sigma(\<\alpha\>,0)=\emptyset\] and
  \[
    \sigma(\<\alpha,\beta\>,n+1)\
      =
    f_\beta^\leftarrow[n]
      \cup
    \{\gamma<\alpha\cap\beta:f_\alpha(\gamma)\not=f_\beta(\gamma)\}
  .\]
  
  Let \(\nu\) be an attack by \(\plI\) against \(\sigma\), and let
  \(\gamma<\nu(n)\) for \(N\leq n<\omega\).
  If \(f_{\nu(n)}(\gamma)\not=f_{\nu(n+1)}(\gamma)\) for infinitely-many
  \(N\leq n<\omega\), then \(\gamma\in\sigma(\<\nu(n),\nu(n+1)\>,n+1)\)
  for infinitely-many \(N\leq n<\omega\).
  Otherwise \(f_{\nu(n)}(\gamma)=f_{\nu(n+1)}(\gamma)=M\) for cofinitely-many
  \(N\leq n<\omega\), so \(\gamma\in\sigma(\<\nu(n),\nu(n+1)\>,n+1)\)
  for cofinitely-many \(N\leq n<\omega\). Therefore \(\sigma\) is a winning \(2\)-mark.
\end{proof}

\begin{theorem}
  \(\plI\kmarkwin{2}DS(C_p(\omega_1+1))\)
\end{theorem}
\begin{proof}
  Let \(\sigma\) be a winning 2-mark for \(\plII\) in \(G\).

  Given a point \(f\in C_p(\omega_1+1)\),
  let \(\alpha_f<\omega_1\) satisfy \(f(\beta)=f(\gamma)\) for all
  \(\alpha_f\leq\beta\leq\gamma\leq\omega_1\).

  Let \(\tau(\emptyset,0)=[\mathbf{0},\{\omega_1\},4]\),
  \(\tau(\<f\>,1)=[\mathbf{0};\sigma(\<\alpha_f\>,0)\cup\{\omega_1\};2]\), and
  \[\tau(\<f,g\>,n+2)=[\mathbf{0};\sigma(\<\alpha_f,\alpha_g\>,n+1)\cup\{\omega_1\};2^{-n}].\]

  Let \(\nu\) be a legal attack by \(\plII\) against \(\sigma\).
  For \(\beta\leq\omega_1\), if \(\beta<\alpha_{\nu(n)}\)
  for co-finitely many \(n<\omega\), then
  \(\beta\in\sigma(\<\alpha_{\nu(n)},\alpha_{\nu(n+1)}\>)\) for
  infinitely-many \(n<\omega\), and thus \(0\in\operatorname{cl}\{\nu(n)(\beta):n<\omega\}\).
  Otherwise \(\beta\geq\alpha_{\nu(n)}\) for infinitely many \(n<\omega\),
  and thus \(0\in\operatorname{cl}\{\nu(n)(\beta):n<\omega\}\) as well.
  Thus \(\mathbf{0}\in\operatorname{cl}\{\nu(n):n<\omega\}\).
\end{proof}

\section{combining game results}

\begin{theorem}
The following are equivalent for \(T_{3.5}\) spaces \(X\).
\begin{enumerate}[a)]
\item \(X\) is \(R^+\), that is, \(\plII\win \schStrongSelGame{\mc O_X}{\mc O_X}\).
\item \(\plI\win PO(X)\). 
\item \(\plI\win FO(X)\).
\item \(\plI\win \gruConGame{C_p(X)}{\mathbf 0}\).
\item \(\plI\win CL(C_p(X),\mathbf 0)\).
\item \(\plI\win CD(C_p(X))\).
\item \(X\) is \(\Omega R^+\), that is, 
  \(\plII\win \schStrongSelGame{\Omega_X}{\Omega_X}\).
\item \(C_p(X)\) is \(sCFT^+\), that is, 
  \(\plII\win \schStrongSelGame{\Omega_{C_p(X),\mathbf 0}}{\Omega_{C_p(X),\mathbf 0}}\).
\item \(C_p(X)\) is \(sCDFT^+\), that is,
  \(\plII\win \schStrongSelGame{\mc D_{C_p(X)}}{\Omega_{C_p(X),\mathbf 0}}\).
\end{enumerate}
\end{theorem}

\begin{proof}
  (a) \(\Leftrightarrow\) (b) is a well-known result of Galvin.

  (b) \(\Leftrightarrow\) (c) is 4.3 of [Telgarksy 1975].

  The equivalence of (b), (d), (e), and (f) are given as 3.8 of [Tkachuk 2017].

  The equivalence of (g), (h), and (i) are due to Clontz.

  (i) \(\Leftrightarrow\) (e) follows from 3.18a of [Tkachuk 2017].
\end{proof}

Let \(\Omega PO(X)\) be the point-open game where \(\plI\) wins
if they force \(\plII\) to create an \(\omega\) cover. Likewise for
\(\Omega FO(X)\). In 3.9 of [Tkachuk 2017] it's shown that
\(\plII\win\Omega FO(X)\) implies \(\plI\win CD(C_p(X))\); it's
taken for granted (but unproven) that \(\plII\win \Omega FO(X)\) if and only if
\(\plII\win \Omega PO(X)\) 
(this holds for \(FO(X),PO(X)\) due to Telgarsky). 

\begin{theorem}
  The following are equivalent for a \(T_{3.5}\) space \(X\).
  \begin{enumerate}[a)]
  \item \(\plI\win\Omega PO(X)\).
  \item \(\plI\win\Omega FO(X)\).
  \item \(X\) is \(R^+\).
  \end{enumerate}
\end{theorem}

\begin{proof}
  (a) implies (b) follows trivially, and (b) implies (c) because (c) is
  equivalent to \(\plI\win FO(X)\).

  So assume (c), which is equivalent to \(\Omega R^+\). Let \(\sigma\)
  be a winning strategy for \(\plII\) in 
  \(\schStrongSelGame{\Omega_X}{\Omega_X}\). Let \(T(X)\) be the non-empty
  open sets of \(X\), and let \(s\in T(X)^{<\omega}\).
  Assume \(\tau(t)\in[X]^{<\omega}\) is defined for all \(t<s\),
  and \(\mc U_t\in\Omega_X\) is defined for all 
  \(\emptyset<t\leq s\). 

  Suppose that for all \(F\in[X]^{<\omega}\), there existed \(U_F\in T(X)\)
  containing \(F\) such that for all \(\mc U\in\Omega_X\),
  \(U_F\not=\sigma(\<\mc U_{s\rest 1},\dots,U_s,\mc U\>)\).
  Let \(\mc U=\{U_F:F\in[X]^{<\omega}\}\in\Omega_X\).
  Then \(\sigma(\<\mc U_{s\rest 1},\dots,U_s,\mc U\>)\)
  must equal some \(U_F\), demonstrating a contradiction.

  So there exists \(\tau(s)\in[X]^{<\omega}\) such that for all \(U\in T(X)\)
  containing \(\tau(s)\),
  there exists \(\mc U_{s\concat\<U\>}\in\Omega_X\) such that
  \(U=\sigma(\<\mc U_{s\rest 1},\dots,\mc U_s,\mc U_{s\concat\<U\>}\>)\).
  (To complete the induction, \(\mc U_{s\concat\<U\>}\) may be chosen
  arbitrarily for all other \(U\in T(X)\).)

  So \(\tau\) is a strategy for \(\plI\) in \(\Omega FO(X)\).
  Let \(\nu\) legally attack \(\tau\), so
  \(\tau(\nu\rest n)\subseteq \nu(n)\) for all \(n<\omega\).
  It follows that 
  \(\nu(n)=\sigma(\<\mc U_{\nu\rest 1},\dots,\mc U_{\nu\rest n},\mc U_{\nu\rest n+1}\>)\).
  Since \(\<\mc U_{\nu\rest 1},\mc U_{n\rest 2},\dots\>\) is a legal attack
  against \(\sigma\), it follows that
  \(\{\sigma(\<\mc U_{\nu\rest 1},\dots,\mc U_{\nu\rest n+1}\>):n<\omega\}=\{\nu(n):n<\omega\}\)
  is an \(\omega\) cover. Therefore \(\tau\) is a winning strategy, verifying (b).

  TODO: (b) implies (a).
\end{proof}

  \bibliographystyle{plain}
  \bibliography{../bibliography}


\end{document}
