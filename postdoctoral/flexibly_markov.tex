\documentclass[11pt]{article}

\usepackage{../clontzStyle}
\usepackage{../clontzDefinitions}

\begin{document}

  \begin{definition}
    A \term{flexibly Markov} strategy for a game \(\<G,M\>\)
    is a pair of functions
    \(\<\sigma,\rho\>\in M^{M\times\omega}\times M^{M^{<\omega}}\).
    Intuitively, \(\sigma\) is a Markov strategy using a single move
    of the opponent and the round number, and \(\tau\) is a strategy
    which applies the Markov strategy to each of the opponent's
    previous moves individually each round.

    Let \(\sigma':M^{<\omega}\to M^{<\omega}\) be
    defined by \(|\sigma'(s)|=|s|\) and
    \(\sigma'(s)(m)=\sigma(s(m),|s|)\).
    Then more rigorously, \(\<\sigma,\rho\>\)
    is a winning flexibly Markov strategy if and only if
    \(\tau:M^{<\omega}\to M\) defined by
    \(\tau(s)=\rho\circ\sigma'(s)\) is a winning strategy.
  \end{definition}

  \begin{definition}
    A \term{semi-flexibly Markov} strategy for a game \(\<G,M\>\)
    is a pair of functions
    \(\<\sigma,\phi\>\in M^{M\times\omega}\times\omega^{\omega}\).
    Intuitively, \(\sigma\) is a Markov strategy using a single move
    of the opponent and the round number, and \(\phi\) is a strategy
    for choosing how far in the past the move of the opponent should be
    chosen.

    More rigorously, \(\<\sigma,\phi\>\) is a winning semi-flexibly
    Markov strategy if and only if \(\tau:M^{<\omega}\to M\) defined
    by \(\tau(s)=\sigma(s(|s|-1-\phi(|s|)),|s|)\) is a winning
    strategy.
  \end{definition}

  \begin{proposition}
    \(\pl F\semiflexmarkwin\schFillWeakGame{\kappa}\)
  \end{proposition}

  \begin{proof}
    Let \(\phi\) satisfy \(|\phi^{-1}(n)|=\omega\) and
    \(\sigma(C,n)=C\rest n\). Then since \(\sigma\) sees every move
    of \(\pl C\) infinitely often, it follows that it covers every move
    of \(\pl C\).
  \end{proof}

  \begin{proposition}
    For compact \(X\),
    \(\pl D\win\bellAbsConGame{X}\) if and only if
    \(\pl D\flexmarkwin\bellAbsConGame{X}\)
  \end{proposition}

  \begin{proof}
    The forward direction is the only interesting direction. In this case,
    \(X\) is Corson compact and embeddable in the \(\Sigma\)-product of
    reals.
    Let \(\phi\) satisfy \(|\phi^{-1}(n)|=\omega\), and let \(supp(x)\)
    be the countable support of \(x\), with
    \(supp_{n+1}(x)\supseteq supp_{n}(x)\) finite and
    \(\bigcup_{n<\omega}supp_n(x)=supp(x)\).

    Let \(\sigma(x,n)=D(2^{-n},supp_n(x))\). Then since \(\sigma\) sees
    every move infinitely often, the non-zero coordinates are all Cauchy
    and converge.
  \end{proof}


\end{document}