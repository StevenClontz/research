\documentclass[11pt]{article}

\usepackage{../clontzStyle}
\usepackage{../clontzDefinitions}

\begin{document}

  \begin{definition}
    Let a V-map be a u.s.c. idempotent surjection.
  \end{definition}

  \begin{definition}
    For any LOS \(\<L,\leq\>\), let \(\hat L\) be the collection of
    left-closed subsets of \(L\)
    (closed subsets for which \(b\in L,a\leq b\Rightarrow a\in L\))
    linearly ordered by \(\subseteq\).
  \end{definition}

  It's well-known that \(\hat L\) is compact. If \(L\) lacks a least element,
  \(L\) embeds as a dense subspace (consider the sets
  \((\leftarrow,l]\) for \(l\in L\)).

  \begin{definition}
    For any compact LOTS \(K\) with minimum \(0\) and maximum \(1\),
    let \(\Gamma\) be the V-map on \(K\) where \(\Gamma(0)=K\) and \(\Gamma(t)=\{1\}\)
    for \(t>0\).
  \end{definition}

  \begin{theorem}
    \(X=\varprojlim \{2, \Gamma, L\}\cong \hat L\)
  \end{theorem}

  \begin{proof}
    We start by placing an order on \(X\). Let \(\vec x<\vec y\) if
    there exists \(a\in L\) with \(\vec x(a)=0,\vec y(a)=1\). We claim this is
    a total order inducing the topology on \(X\).

    We first observe that if \(\vec x(b)=1\), then for all \(a\leq b\),
    \(\vec x(a)\in\Gamma(1)=\{1\}\). If \(\vec x\not=\vec y\), then assume
    without loss of generality that \(\vec x(a)=0,\vec y(a)=1\), so
    \(\vec x<\vec y\). Also, whenever \(\vec x(b)=1\), we have that \(b<a\),
    so \(\vec y(b)=1\), preventing \(\vec y<\vec x\). Finally if
    \(\vec x<\vec y\) and \(\vec y<\vec z\), take \(a,b\) with
    \(\vec x(a)=0\), \(\vec y(a)=1\),\(\vec y(b)=0\),\(\vec z(b)=1\). It
    follows that \(a<b\) so \(\vec z(a)=1\) and \(\vec x<\vec z\).

    Consider the basic open set \(B(\vec x,F)\) for a finite set
    \(F\in [L]^{<\omega}\)
    about the sequence \(\vec x\in X\) which contains all sequences
    \(\vec y\) agreeing with \(\vec x\) on \(F\). If \(\vec x(a)=1\) for all
    \(a\in F\), then let \(\vec w\in X\) be \(0\) on the maximum of \(F\),
    and \(1\) for anything less. It follows that
    \(B(\vec x,F)=(\vec w,\rightarrow)\). If \(\vec x(a)=0\) for all
    \(a\in F\), then let \(\vec y\in X\) be \(1\) on the minimum of \(F\),
    and \(0\) for anything greater. It follows that
    \(B(\vec x,F)=(\leftarrow,\vec y)\). Finally if \(\vec x(a)=1\) and
    \(\vec x(b)=0\) for \(a<b\) in \(F\) and nothing between \(a,b\) is in
    \(F\), then let \(\vec w\in X\) be \(0\) on \(a\)
    and \(1\) for anything less, and let \(\vec y\in X\) be \(1\) on \(b\)
    and \(0\) for anything greater. It follows that
    \(B(\vec x,F)=(\vec w,\vec y)\).

    Let \(\phi\) evaluate each \(\vec x\in X\subseteq 2^L\) as the
    characteristic function for a subset of \(L\). It's easy to see that
    \(\phi\) is an order isomorphism between \((X,\leq)\) and \((\hat L,\subseteq)\).
  \end{proof}

\newpage
\bibliographystyle{plain}
\bibliography{../bibliography}

\end{document}