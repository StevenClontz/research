\documentclass[11pt]{article}

\usepackage{../clontzStyle}
\usepackage{../clontzDefinitions}

\renewcommand{\int}{\textrm{int}}
\renewcommand{\cl}{\textrm{cl}}
\newcommand{\lexTimes}{\times_{\textrm{lex}}}
\newcommand{\vect}{\vec}

\usepackage{stmaryrd}

\newcommand{\lb}{\llbracket}
\newcommand{\rb}{\rrbracket}

\begin{document}

  \begin{definition}
    Let a V-map be a u.s.c. idempotent surjection.
  \end{definition}

  \begin{definition}
    A u.s.c. map \(f\subseteq X^2\)
    satisfies condition \(\Gamma\) if there exists \(x,y\in X\) such that
    \(\<x,x\>,\<y,y\>,\<x,y\>\in f\).
  \end{definition}

  \begin{theorem}
    Every V-map on a compact connected LOTS satisfies condition \(\Gamma\).
  \end{theorem}

  This was proven in \cite{CLONTZVARAGONA} for the case \(M\) is a closed
  interval of real numbers; however, the same proof applies to this more
  general case.





\newpage

  \begin{definition}
    For any LOS \(\<L,\leq\>\), let \(\check L\) be the collection of
    leftward subsets of \(L\)
    (subsets for which \(b\in L,a\leq b\Rightarrow a\in L\))
    linearly ordered by \(\subseteq\), and let \(\hat L\) be the collection
    of left-closed subsets of \(L\) (leftward subsets which are closed)
    linearly ordered by \(\subseteq\).
  \end{definition}

  \begin{proposition}
    \(\check L\), \(\hat L\) are compact.
  \end{proposition}

  \begin{proof}
    Each subset \(S\) has an infimum \(\cap S\) and a supremum \(\cup S\)
    (or \(\cl(\cap S)\)).
  \end{proof}

  \begin{theorem}
    \(\check L\) is second-countable if and only if \(L\) is countable.
  \end{theorem}

  \begin{proof}
    If \(L\) is uncountable, let \(\mc B\) be a base for \(\check L\).
    For each \(l\in L\), there is some neighborhood \(B_l\in\mc B\) of
    \((\leftarrow,l)\) which is a subset of
    \([\emptyset,(\leftarrow,l])=[\emptyset,(\leftarrow,l)]\). Each \(B_l\)
    is unique, so \(\check L\) is not second-countable.

    If \(L\) is countable, then note that the union of
    \(\mc S=\{((\leftarrow,l),L]:l\in L\}\) and
    \(\mc T=\{[\emptyset,(\leftarrow,l]):l\in L\}\)
    generates a countable base for \(\check L\).
  \end{proof}

  \begin{corollary}
    \(\check L\) is metrizable if and only if \(L\) is countable.
  \end{corollary}

  Note that \(\check L\)
  is not a ``compactification'' in the usual sense of word
  as \(L\) does not necessarily
  embed as a dense subspace of \(\check L\): if \(L=I\), we might attempt to embed
  \(t\mapsto [0,t)\), but then note that the subspace topology induces the
  reverse Sorgenfrey interval as \(([0,s),[0,t])=([0,s),[0,t)]\) is open.
  However \(\hat L\) is
  the typical way of compactifying a linearly ordered space \(L\),
  provided \(L\) lacks a least element (otherwise the empty set is an [easily
  removable] isolated point in \(\hat L\)). Note that we \textbf{always}
  assume that \(\emptyset\in\hat L\):

  \begin{example}
    \(\hat I\cong\{-1\}\cup [0,1]\) where
    \(\emptyset\mapsto-1\) and \([0,t]\mapsto t\).
  \end{example}

  \begin{example}
    For limit ordinals \(\alpha\), \(\hat \alpha\cong \alpha+1\),
    and for all other infinite ordinals, \(\hat\alpha\cong\alpha\).
    (The addition of a new least isolated point is of course irrelevant).
  \end{example}

  \begin{definition}
    For any space \(M\) containing the points \(0\) and \(1\),
    let \(\gamma\) be the surjective u.s.c. map on \(K\) where
    \(\gamma(0)=M\) and \(\gamma(t)=\{1\}\)
    for \(t\not=0\).
  \end{definition}

  Note when \(M\) is a Hausdorff continuum, then \(\gamma\) is a V-map.
  It's usually assumed \(0\) is the minimum and \(1\) is the maximum
  when \(M\) is a LOTS.

  \begin{theorem}
    \(X=\varprojlim \{2, \gamma, L\}\cong \check L\)
  \end{theorem}

  \begin{proof}
    We start by placing an order on \(X\). Let \(\vect x<\vect y\) if
    there exists \(a\in L\) with \(\vect x(a)=0,\vect y(a)=1\). We claim this is
    a total order inducing the topology on \(X\).

    We first observe that if \(\vect x(b)=1\), then for all \(a\leq b\),
    \(\vect x(a)\in\gamma(1)=\{1\}\). If \(\vect x\not=\vect y\), then assume
    without loss of generality that \(\vect x(a)=0,\vect y(a)=1\), so
    \(\vect x<\vect y\). Also, whenever \(\vect x(b)=1\), we have that \(b<a\),
    so \(\vect y(b)=1\), preventing \(\vect y<\vect x\). Finally if
    \(\vect x<\vect y\) and \(\vect y<\vect z\), take \(a,b\) with
    \(\vect x(a)=0\), \(\vect y(a)=1\),\(\vect y(b)=0\),\(\vect z(b)=1\). It
    follows that \(a<b\) so \(\vect z(a)=1\) and \(\vect x<\vect z\).

    Consider the basic open set \(B(\vect x,F)\) for a finite set
    \(F\in [L]^{<\omega}\)
    about the sequence \(\vect x\in X\) which contains all sequences
    \(\vect y\) agreeing with \(\vect x\) on \(F\). If \(\vect x(a)=1\) for all
    \(a\in F\), then let \(\vect w\in X\) be \(0\) on the maximum of \(F\),
    and \(1\) for anything less. It follows that
    \(B(\vect x,F)=(\vect w,\rightarrow)\). If \(\vect x(a)=0\) for all
    \(a\in F\), then let \(\vect y\in X\) be \(1\) on the minimum of \(F\),
    and \(0\) for anything greater. It follows that
    \(B(\vect x,F)=(\leftarrow,\vect y)\). Finally if \(\vect x(a)=1\) and
    \(\vect x(b)=0\) for \(a<b\) in \(F\) and nothing between \(a,b\) is in
    \(F\), then let \(\vect w\in X\) be \(0\) on \(a\)
    and \(1\) for anything less, and let \(\vect y\in X\) be \(1\) on \(b\)
    and \(0\) for anything greater. It follows that
    \(B(\vect x,F)=(\vect w,\vect y)\).

    Let \(\phi\) evaluate each \(\vect x\in X\subseteq 2^L\) as the
    characteristic function for a subset of \(L\). It's easy to see that
    \(\phi\) is an order isomorphism between \(\<X,\leq\>\) and
    \(\<\check L,\subseteq\>\).
  \end{proof}

  \begin{corollary}
    If \(f\) is a non-trivial V-map satisfying condition \(\Gamma\), then
    \(\varprojlim \{X, f, L\}\) is metrizable if and only if
    \(X\) is metrizable and \(L\) is countable.
  \end{corollary}

  \begin{proof}
    If \(X\) is metrizable and \(L\) is countable, then of course
    \(\varprojlim \{X,f,L\}\subseteq X^L\) is metrizable. If \(X\) is
    nonmetrizable, then \(\varprojlim \{X,f,L\}\) cannot be metrizable
    as it contains a copy of \(X\). Finally, if \(L\) is uncountable,
    then \(\varprojlim \{X,f,L\}\) cannot be metrizable as it
    contains a copy of \(\varprojlim \{2,\gamma,L\}\cong\check L\).
  \end{proof}

  \begin{corollary}
    If \(f\) is a nontrivial V-map
    and \(K\) is a compact connected LOTS, then
    \(\varprojlim \{K, f, L\}\) is metrizable if and only if
    \(K=I\) and \(L\) is countable.
  \end{corollary}

  \begin{proof}
    \(I\) is the only nontrivial compact connected metrizable LOTS, and all V-maps
    on compact connected LOTS satisfy condition \(\Gamma\).
  \end{proof}

  \begin{corollary}
    \(
      \varprojlim \{2, \gamma, \alpha\}
      \cong
      \alpha+1
    \)
    for every ordinal \(\alpha\).
  \end{corollary}

  \begin{proof}
    Since \(\check\alpha=\alpha+1\) (actually equals, not just homeomorphic!),
    we get \(\varprojlim^\star \{2, \gamma, \alpha\}
      \cong \check\alpha =
    \alpha+1\) for free.
    Note that C and Varagona used this in (TODO create citataion) to break
    metrizability in uncountable-ordinal-indexed inverse limits (for any V-map
    there exists a two-point set \(2\) such that \(f\rest 2\supseteq\gamma\),
    that is, ``\(f\) has condition \(\Gamma\)'').
  \end{proof}

  \bigskip

  We may generalize theorem whatevs as follows:

  \begin{theorem} %TODO fix first \sim to use A,B instead of l_0,l_1
    If \(M\) is a LOTS with minimum \(0\) and maximum \(1\),
    then \(\varprojlim\{M,\gamma,L\}\cong \hat L\lexTimes M/\sim\),
    where \(\<(\leftarrow,l_0],1\>\sim\<(\leftarrow,l_1],0\>\) if
    \(l_0<l_1\) and \((l_0,l_1)=\emptyset\),
    and where \(\<A,m\>\sim\<A,m'\>\) if \(A\in\hat L\setminus L\).
  \end{theorem}

  \begin{proof}
    Let \(\rho(\vect x)=\cl\{l\in L:\vect x(l)>0\}\), \(v(\vect 0)=0\),
    and \(v(\vect x)=\min\{\vect x(l):l\in\rho(\vect x)\}\) otherwise.
    Say \(\vect x<\vect y\) if \(\rho(\vect x)\subsetneq\rho(\vect y)\)
    or both \(\rho(\vect x)=\rho(\vect y)\) and \(v(\vect x)<v(\vect y)\).
    The reader may verify that this is a linear order on
    \(\varprojlim\{M,\gamma,L\}\), and
    \(\theta(\vect x)=\<\rho(\vect x),v(\vect x)\>\in\hat L\lexTimes M/\sim\)
    preserves order.
    For each left-closed set \(A\) and \(m\in M\),
    let \(\vect x_{A,m}(l)=1\) for \(l\in A\) unless \(l\) is the supremum
    element of \(A\),
    \(\vect x_{A,m}(l)=m\) if \(l\) is the supremum of \(A\),
    and \(\vect x_{A,m}(l)=0\) for \(l\not\in A\).
    To complete the proof, we should demonstrate that the linear order we
    defined induces the topology of the inverse limit, and that \(\theta\)
    is a surjection.

    A basic open set in \(\varprojlim\{M,\gamma,L\}\subseteq L^m\)
    is of the form
    \(\lb U,F \rb\) where \(U(l)\) is an open interval in \(M\) for each
    \(l\in F\in [L]^{<\omega}\), and
    \(\lb U,F\rb=\{\vect x:l\in F\Rightarrow\vect x(l)\in U(l)\}\).
    If we assume that \(\lb U,F \rb\) is non-empty, one of the following
    must hold:

    \begin{itemize}
      \item
        \(U[l_0]=(a,b)\) for some \(l_0\in F\).
        Then \(\lb U,F\rb=\lb U,\{l_0\}\rb \), and note that
        \(
          \lb U,\{l_0\}\rb
            =
          (\vect x_{(\leftarrow,l_0],a},\vect x_{(\leftarrow,l_0],b})
        \).
      \item
        \(U(l_0)=(a,1]\) and \(U(l_1)=[0,b)\) for some \(l_0<l_1\in L\)
        and \(\lb U,F\rb=\lb U,\{l_0,l_1\}\rb \). Then
        \(\lb U,\{l_0,l_1\}\rb =(\vect x_{l_0,a},\vect x_{l_1,b})\).
    \end{itemize}
    In the other direction, consider \(\vect y\in(\vect x,\vect z)\).
    \begin{itemize}
      \item
        In the case that \(l_0\in\rho(\vect y)\setminus\rho(\vect x)\)
        and \(l_1\in\rho(\vect z)\setminus\rho(\vect y)\), let
        \(U(l_0)=(0,1]\), \(U(l_1)=[0,v(\vect z))\) and note
        \(\vect y\in\lb U,\{l_0,l_1\}\rb \subseteq(\vect x,\vect z)\).
      \item
        In the case that \(l_0\in\rho(\vect y)\setminus\rho(\vect x)\),
        \(\rho(\vect y)=\rho(\vect z)\), and \(v(\vect y)<v(\vect z)\),
        it follows that \(\rho(\vect y)=\rho(\vect z)=(\leftarrow,l_1]\),
        so let \(U(l_0)=(0,1]\), \(U(l_1)=[0,v(\vect z))\) and note
        \(\vect y\in\lb U,\{l_0,l_1\}\rb \subseteq(\vect x,\vect z)\).
      \item
        In the case that
        \(\rho(\vect x)=\rho(\vect y)\), \(v(\vect x)<v(\vect y)\),
        and \(l_1\in\rho(\vect z)\setminus\rho(\vect y)\),
        it follows that \(\rho(\vect x)=\rho(\vect y)=(\leftarrow,l_0]\),
        so let \(U(l_0)=(v(\vect x),1]\), \(U(l_1)=[0,v(\vect z))\) and note
        \(\vect y\in\lb U,\{l_0,l_1\}\rb \subseteq(\vect x,\vect z)\).
      \item
        In the case that
        \(\rho(\vect x)=\rho(\vect y)=\rho(\vect z)\) and
        \(v(\vect x)<v(\vect y)<v(\vect z)\),
        it follows that
        \(\rho(\vect x)=\rho(\vect y)=\rho(\vect z)=(\leftarrow,l_0]\),
        so let \(U(l_0)=(v(\vect x),v(\vect z))\) and note
        \(\vect y\in\lb U,\{l_0\}\rb =(\vect x,\vect z)\).
    \end{itemize}

    We conclude by
    showing that \(\theta\) is a surjection. If \(B\in\hat L\setminus L\)
    and \(m\in M\), consider \(\<B,m\>\).
    \(B\) lacks a supremum in \(L\), so
    \(\vect x_{B,0}(l)=1\) for \(l\in B\) and \(\vect x_{B,0}(l)=0\) otherwise.
    So \(\theta(\vect x_{B,0})=\<\cl B,1\>=\<B,1\>\sim\<B,m\>\) for all
    \(m\in M\). Otherwise, \(B=(\leftarrow,l_1]\) for some \(l_1\in L\).
    Let \(m>0\).
    Then
    \(
      \theta(\vect x_{(\leftarrow,l_1],m})
        =
      \<\cl (\leftarrow,l_1],v(\vect x_{(\leftarrow,l_1],m})\>
        =
      \<(\leftarrow,l_1],m\>
    \).
    Finally, we want to map onto \(\<(\leftarrow,l_1],0\>\).
    If there exists \(l_0<l_1\) with \((l_0,l_1)=\emptyset\), then
    \(
      \theta(\vect x_{(\leftarrow,l_1],0})
        =
      \theta(\vect x_{(\leftarrow,l_0],1})
        =
      \<(\leftarrow,l_0],1\>
    \)
    will suffice. Otherwise,
    \(
      \theta(\vect x_{(\leftarrow,l_1],0})
        =
      \<\cl (\leftarrow,l_1),v(\vect x_{(\leftarrow,l_1),0})\>
        =
      \<(\leftarrow,l_1],0\>
    \).
  \end{proof}

  Here are some applications:

  \begin{example}
    \(
      \varprojlim\{2,\gamma,I\}
        \cong
      (\hat I\setminus\emptyset)\lexTimes 2
        \cong
      I\lexTimes 2
        \cong
      \check I
    \) (of course, this could be found quicker with theorem 8).
  \end{example}

  \begin{example}
    \(
      \varprojlim\{I,\gamma,I\}
        \cong
      (\hat I\setminus\emptyset)\lexTimes I
        \cong
      I\lexTimes I
    \).
  \end{example}

  \begin{example}
    For infinite ordinals \(\alpha\),
    \(
      \varprojlim\{I,\gamma,\alpha\}
        \cong
      (\alpha\lexTimes[0,1))\cup\{\infty\}
    \).
    In particular, \(\alpha=\kappa\) for an infinite cardinal \(\kappa\)
    gives the closed long ray of length \(\kappa\).
  \end{example}



  We now discard the linear order on \(M\). Denote points in the topological
  sum \(\bigoplus_{l\in \hat L} M\) of \(\hat L\) copies of \(M\)
  by the ordered pairs \(\<A,m\>\)
  for \(A\in \hat L\), \(m\in M\). The previous proof may be generalized to
  show the following:

  \begin{theorem}
    If \(M\) is a Hausdorff space with points \(0\) and \(1\),
    then \(\varprojlim\{M,\gamma,L\}\cong \bigoplus_{l\in \hat L} M/\sim\),
    where \(\<(\leftarrow,l_0],1\>\sim\<(\leftarrow,l_1],0\>\) if
    \(l_0<l_1\) and \((l_0,l_1)=\emptyset\),
    and where \(\<A,m\>\sim\<A,m'\>\) if \(A\in\hat L\setminus L\).
  \end{theorem}

  % \begin{proof}
  %   Partially order \(M\) by \(0\leq m\leq 1\) for all \(m\in M\).
  %   Note that the range of any
  %   \(\vect x\in\varprojlim\{M,\gamma,L\}\) is totally ordered by \(\leq\).

  %   Let \(\rho(\vect x)=\cl\{l\in L:\vect x(l)>0\}\), \(v(\vect 0)=0\),
  %   and \(v(\vect x)=\min\{\vect x(l):l\in\rho(\vect x)\}\) otherwise.
  %   Let
  %   \(\theta(\vect x)=\<\rho(\vect x),v(\vect x)\>\in\hat L\lexTimes M/\sim\).
  %   For each left-closed set \(A\) and \(m\in M\),
  %   let \(\vect x_{A,m}(l)=1\) for \(l\in A\) unless \(l\) is the supremum
  %   element of \(A\),
  %   \(\vect x_{A,m}(l)=m\) if \(l\) is the supremum of \(A\),
  %   and \(\vect x_{A,m}(l)=0\) for \(l\not\in A\).
  %   We wish to show that \(\theta\) is a homeomorphism.

  %   We begin by
  %   showing that \(\theta\) is a surjection. If \(B\in\hat L\setminus L\)
  %   and \(m\in M\), consider \(\<B,m\>\).
  %   \(B\) lacks a supremum in \(L\), so
  %   \(\vect x_{B,0}(l)=1\) for \(l\in B\) and \(\vect x_{B,0}(l)=0\) otherwise.
  %   So \(\theta(\vect x_{B,0})=\<\cl B,1\>=\<B,1\>\sim\<B,m\>\) for all
  %   \(m\in M\). Otherwise, \(B=(\leftarrow,l_1]\) for some \(l_1\in L\).
  %   Let \(m>0\).
  %   Then
  %   \(
  %     \theta(\vect x_{(\leftarrow,l_1],m})
  %       =
  %     \<\cl (\leftarrow,l_1],v(\vect x_{(\leftarrow,l_1],m})\>
  %       =
  %     \<(\leftarrow,l_1],m\>
  %   \).
  %   Finally, we want to map onto \(\<(\leftarrow,l_1],0\>\).
  %   If there exists \(l_0<l_1\) with \((l_0,l_1)=\emptyset\), then
  %   \(
  %     \theta(\vect x_{(\leftarrow,l_1],0})
  %       =
  %     \theta(\vect x_{(\leftarrow,l_0],1})
  %       =
  %     \<(\leftarrow,l_0],1\>
  %   \)
  %   will suffice. Otherwise,
  %   \(
  %     \theta(\vect x_{(\leftarrow,l_1],0})
  %       =
  %     \<\cl (\leftarrow,l_1),v(\vect x_{(\leftarrow,l_1),0})\>
  %       =
  %     \<(\leftarrow,l_1],0\>
  %   \).

  %   To see that \(\theta\) is injective, first let
  %   \(A\subsetneq B\)... TODO

  %   % A basic open set in \(\varprojlim\{M,\gamma,L\}\subseteq L^m\)
  %   % is of the form
  %   % \(\lb U,F \rb\) where \(U(l)\) is an open interval in \(M\) for each
  %   % \(l\in F\in [L]^{<\omega}\), and
  %   % \(\lb U,F\rb=\{\vect x:l\in F\Rightarrow\vect x(l)\in U(l)\}\).
  %   % If we assume that \(\lb U,F \rb\) is non-empty, one of the following
  %   % must hold:

  %   % \begin{itemize}
  %   %   \item
  %   %     \(U[l_0]=(a,b)\) for some \(l_0\in F\).
  %   %     Then \(\lb U,F\rb=\lb U,\{l_0\}\rb \), and note that
  %   %     \(
  %   %       \lb U,\{l_0\}\rb
  %   %         =
  %   %       (\vect x_{(\leftarrow,l_0],a},\vect x_{(\leftarrow,l_0],b})
  %   %     \).
  %   %   \item
  %   %     \(U(l_0)=(a,1]\) and \(U(l_1)=[0,b)\) for some \(l_0<l_1\in L\)
  %   %     and \(\lb U,F\rb=\lb U,\{l_0,l_1\}\rb \). Then
  %   %     \(\lb U,\{l_0,l_1\}\rb =(\vect x_{l_0,a},\vect x_{l_1,b})\).
  %   % \end{itemize}
  %   % In the other direction, consider \(\vect y\in(\vect x,\vect z)\).
  %   % \begin{itemize}
  %   %   \item
  %   %     In the case that \(l_0\in\rho(\vect y)\setminus\rho(\vect x)\)
  %   %     and \(l_1\in\rho(\vect z)\setminus\rho(\vect y)\), let
  %   %     \(U(l_0)=(0,1]\), \(U(l_1)=[0,v(\vect z))\) and note
  %   %     \(\vect y\in\lb U,\{l_0,l_1\}\rb \subseteq(\vect x,\vect z)\).
  %   %   \item
  %   %     In the case that \(l_0\in\rho(\vect y)\setminus\rho(\vect x)\),
  %   %     \(\rho(\vect y)=\rho(\vect z)\), and \(v(\vect y)<v(\vect z)\),
  %   %     it follows that \(\rho(\vect y)=\rho(\vect z)=(\leftarrow,l_1]\),
  %   %     so let \(U(l_0)=(0,1]\), \(U(l_1)=[0,v(\vect z))\) and note
  %   %     \(\vect y\in\lb U,\{l_0,l_1\}\rb \subseteq(\vect x,\vect z)\).
  %   %   \item
  %   %     In the case that
  %   %     \(\rho(\vect x)=\rho(\vect y)\), \(v(\vect x)<v(\vect y)\),
  %   %     and \(l_1\in\rho(\vect z)\setminus\rho(\vect y)\),
  %   %     it follows that \(\rho(\vect x)=\rho(\vect y)=(\leftarrow,l_0]\),
  %   %     so let \(U(l_0)=(v(\vect x),1]\), \(U(l_1)=[0,v(\vect z))\) and note
  %   %     \(\vect y\in\lb U,\{l_0,l_1\}\rb \subseteq(\vect x,\vect z)\).
  %   %   \item
  %   %     In the case that
  %   %     \(\rho(\vect x)=\rho(\vect y)=\rho(\vect z)\) and
  %   %     \(v(\vect x)<v(\vect y)<v(\vect z)\),
  %   %     it follows that
  %   %     \(\rho(\vect x)=\rho(\vect y)=\rho(\vect z)=(\leftarrow,l_0]\),
  %   %     so let \(U(l_0)=(v(\vect x),v(\vect z))\) and note
  %   %     \(\vect y\in\lb U,\{l_0\}\rb =(\vect x,\vect z)\).
  %   % \end{itemize}

  %   % We conclude by
  %   % showing that \(\theta\) is a surjection. If \(B\in\hat L\setminus L\)
  %   % and \(m\in M\), consider \(\<B,m\>\).
  %   % \(B\) lacks a supremum in \(L\), so
  %   % \(\vect x_{B,0}(l)=1\) for \(l\in B\) and \(\vect x_{B,0}(l)=0\) otherwise.
  %   % So \(\theta(\vect x_{B,0})=\<\cl B,1\>=\<B,1\>\sim\<B,m\>\) for all
  %   % \(m\in M\). Otherwise, \(B=(\leftarrow,l_1]\) for some \(l_1\in L\).
  %   % Let \(m>0\).
  %   % Then
  %   % \(
  %   %   \theta(\vect x_{(\leftarrow,l_1],m})
  %   %     =
  %   %   \<\cl (\leftarrow,l_1],v(\vect x_{(\leftarrow,l_1],m})\>
  %   %     =
  %   %   \<(\leftarrow,l_1],m\>
  %   % \).
  %   % Finally, we want to map onto \(\<(\leftarrow,l_1],0\>\).
  %   % If there exists \(l_0<l_1\) with \((l_0,l_1)=\emptyset\), then
  %   % \(
  %   %   \theta(\vect x_{(\leftarrow,l_1],0})
  %   %     =
  %   %   \theta(\vect x_{(\leftarrow,l_0],1})
  %   %     =
  %   %   \<(\leftarrow,l_0],1\>
  %   % \)
  %   % will suffice. Otherwise,
  %   % \(
  %   %   \theta(\vect x_{(\leftarrow,l_1],0})
  %   %     =
  %   %   \<\cl (\leftarrow,l_1),v(\vect x_{(\leftarrow,l_1),0})\>
  %   %     =
  %   %   \<(\leftarrow,l_1],0\>
  %   % \).
  % \end{proof}
















  \newpage

  \begin{definition}
    For any \(M\) containing a point \(0\),
    let \(\nu\) be the V-map on \(M\) where \(\nu(0)=M\) and \(\nu(t)=\{t\}\)
    for \(t>0\).
  \end{definition}

  Note for \(M=2\) that \(\nu=\gamma\).

  \begin{corollary}
    \(\varprojlim\{2,\nu,L\}\cong\check L\).
  \end{corollary}

  \begin{lemma}
    If \(M\) is \(T_2\), then
    \(
      \varprojlim\{M,\nu,L\}\setminus\{\vect 0\}
        \cong
      (\check L\setminus\{\emptyset\})\times(M\setminus\{0\})
    \) with the usual product topology.
  \end{lemma}

  \begin{proof}
    Each point in \(\varprojlim\{M,\nu,L\}\setminus\{\vect 0\}\) is of the
    form \(\vect x_{C,m}\) where \(C\in\check L\setminus\{\emptyset\}\)
    and \(m\in M\setminus\{0\}\) defined by \(\vect x_{C,m}(l)=m\) for \(l\in C\)
    and \(x_{C,m}(l)=0\) otherwise.

    We claim that the bijection \(\theta(\vect x_{C,m})=\<C,m\>\) is a
    homeomorphism. Note that basic open sets of
    \(\varprojlim\{M,\nu,L\}\) are of the form \(\lb U,F \rb\) where \(U(l)\) is
    an open subset of \(M\) for each \(l\in F\in[L]^{<\omega}\).

    Consider the point \(\<C,m\>\) in the basic open set \(V\times W\) in
    \((\check L\setminus\{\emptyset\})\times(M\setminus\{0\})\). Note that
    \(V\) is either of the form \((A,L]\) or \((A,B)\), and we may choose
    \(l_0\in C\setminus A\). We also may assume that \(W\) misses an open
    neighborhood \(Z\) of \(0\) as \(M\) is \(T_2\).

    In the case that \(V=(A,L]\) we let \(U(l_0)=W\).
    Then since \(l_0\in C\) and \(m\in W=U(l_0)\), it follows that
    \(\vect x_{C,m}\in\lb U,\{l_0\}\rb \).
    For any \(\vect x_{D,n}\in\lb U,\{l_0\}\rb \) we have that \(\vect x_{D,n}(l_0)=n\in W\);
    in particular, it's nonzero. So
    \(A\subsetneq (\leftarrow,l_0]\subseteq D\), putting \(D\in(A,L]=V\). Thus
    \(\theta(\vect x_{D,n})=\<D,n\>\in V\times W\).

    In the case that \(V=(A,B)\), we may also choose \(l_1\in B\setminus C\).
    We again let \(U(l_0)=W\), and we also let \(U(l_1)=Z\). Then as
    \(\vect x_{C,m}(l_0)=m\in W=U(l_0)\) and
    \(\vect x_{C,m}(l_1)=0\in Z=U(l_1)\), we have shown
    \(\vect x_{C,m}\in\lb U,\{l_0,l_1\}\rb \). For any \(\vect x_{D,n}\in\lb U,\{l_0,l_1\}\rb \),
    \(\vect x_{D,n}(l_0)\in W\) and \(\vect x_{D,n}(l_1)\in Z\).
    This shows that \(\vect x_{D,n}(l_0)=n\in W\) and \(\vect x_{D,n}(l_1)=0\), so
    \(
      A
        \subsetneq
      (\leftarrow,l_0]
        \subseteq
      D
        \subsetneq
      (\leftarrow,l_1]
        \subseteq
      B
    \), putting \(D\in(A,B)=V\). Thus
    \(\theta(\vect x_{D,n})=\<D,n\>\in V\times W\).

    On the other hand, consider the point \(\vect x_{C,m}\) in the basic
    open set \(\lb U,F \rb\) of
    \(\varprojlim\{M,\nu,L\}\setminus\{\vect 0\}\).
    It follows that \(m\in U(l)\) for all \(l\in F\cap C\),
    and \(0\in U(l)\) for all \(l\in F\setminus C\).

    If \(F\subseteq C\), then let \(l_0\) be the maximum element of \(F\) and
    \(U'=\bigcap_{l\in F}U(l)\). Note \(\<C,m\>\in((\leftarrow,l_0),L]\times U'\).
    So let \(\<D,n\>\in((\leftarrow,l_0),L]\times U'\). Since \(l_0\in D\),
    \(\vect x_{D,n}(l_0)=n\in U'\). Since \(n\not=0\), we have
    \(\vect x_{D,n}(l)=n\in U'\subseteq U(l)\) for all \(l\in F\), so
    \(\vect x_{D,n}\in\lb U,F\rb\).

    If \(F\cap C=\emptyset\), then
    let \(l_1\) be the minimum element of \(F\setminus C\).
    Note \(\<C,m\>\in(\emptyset,(\leftarrow,l_1])\times (M\setminus \{0\})\).
    So let \(\<D,n\>\in(\emptyset,(\leftarrow,l_1])\times (M\setminus \{0\})\).
    Since \(l_1\not\in D\), \(\vect x_{D,n}(l)=0\in U(l)\) for all \(l\in F\),
    giving \(\vect x_{D,n}\in\lb U,F\rb\).

    Otherwise, let \(l_0\) be the maximum element of \(F\cap C\) and
    \(l_1\) be the minimum element of \(F\setminus C\).
    Let \(U'\subseteq \bigcap_{l\in C\cap F}U(l)\) and
    \(U''\subseteq \bigcap_{l\in F\setminus C}U(l)\).
    Note \(\<C,m\>\in((\leftarrow,l_0),(\leftarrow,l_1])\times U'\).
    So let \(\<D,n\>\in((\leftarrow,l_0),(\leftarrow,l_1])\times U'\).
    Since \(l_0\in D\) and \(l_1\not\in D\), we have
    \(\vect x_{D,n}(l_0)=n\in U'\) and
    \(\vect x_{D,n}(l_1)=0\in U''\). Furthermore,
    \(\vect x_{D,n}(l)=n\in U'\subseteq U(l)\) for all \(l\in F\cap C\), and
    \(\vect x_{D,n}(l)=0\in U''\subseteq U(l)\) for all \(l\in F\setminus C\),
    so \(\vect x_{D,n}\in\lb U,F\rb\).
  \end{proof}

  \begin{theorem}
    If \(M\) is \(T_2\), then
    \(
      \varprojlim\{M,\nu,L\}
        \cong
      (\check L\times M)/\sim
    \) with the usual product topology, where \(\sim\) identifies every
    point in \(\{\emptyset\}\times M\cup\check L\times \{0\}\).
  \end{theorem}

  \begin{proof}
    We will extend \(\theta\) as defined in the previous lemma so that
    \(\theta(\vect 0)\) is sent to the points identified in
    \(\{\emptyset\}\times M\cup\check L\times \{0\}\).

    Consider the basic open neighborhood \(\lb U,F]\) of \(\vect 0\) in
    \(\varprojlim\{M,\nu,L\}\). Let \(l_1\) be the least element of \(F\),
    and \(U'=\bigcap_{l\in F}U(l)\). Note that
    \(
      \{\emptyset\}\times M\cup\check L\times \{0\}
        \subseteq
      [\emptyset,(\leftarrow,l_1])\times M
      \cup
      \check L\times U'
    \). So let
    \(
      \vect x_{C,m}
        \in
      [\emptyset,(\leftarrow,l_1])\times M
    \),
    and note that as \(l_1\not\in C\), \(\vect x_{C,m}(l)=0\in U(l)\) for all
    \(l\in F\), and thus \(\vect x_{C,m}\in\lb U,F\rb\).
    Likewise if we let
    \(
      \vect x_{C,m}
        \subseteq
      \check L\times U'
    \),
    we have that \(x_{C,m}(l)\in U'\subseteq U(l)\) for all \(l\in L\), and
    thus \(\vect x_{C,m}\in\lb U,F\rb\).

    Now in \((\check L\times M)/\sim\)
    consider the basic open set formed by the union of
    \(V\times M\) containing \(\{\emptyset\}\times M\) and
    \(\check L\times W\) containing \(\check L\times \{0\}\).
    \(V=[\emptyset,B)\) for some nonempty leftward set \(B\),
    so choose \(l_0\in B\) and note that as
    \((\leftarrow,l_0]\subseteq B\) we have
    \([\emptyset,(\leftarrow,l_0])\subseteq[\emptyset,B)=V\).
    So let \(U(l_0)=W\) and note \(\vect 0\in\lb U,\{l_0\}\rb \). For any
    \(\vect x_{C,m}\in\lb U,\{l_0\}\rb \), we have two cases.

    If \(l_0\in C\), then \(\vect x_{C,m}(l)=m\in U(l_0)=W\) for all \(l\in C\),
    and \(\vect x_{C,m}(l)=0\in W\) otherwise. Thus
    \(\<C,m\>\in\check L\times W\).

    Otherwise \(l_0\not\in C\), and thus \(C\subsetneq(\leftarrow,l_0]\).
    Then as \(C\in [\emptyset,(\leftarrow,l_0])\subseteq V\),
    we have \(\<C,m\>\in V\times M\).
    \end{proof}




















  \newpage

  We introduce an alternate definition of an arbitrarily indexed
  inverse limit.

  \begin{definition}
    Let \(\varprojlim^\star\{X,f,L\}\subseteq\varprojlim\{X,f,L\}\) satisfy
    that \(\vect x(a)=\lim_{t\to a}\vect x(t)\) for all \(a\in L\)
    (for any open neighborhood
    \(U\)of \(\vect x(a)\) there is \(b<a\) where \(\vect x(t)\in U\)
    for all \(t\in(b,a]\)).
  \end{definition}

  \begin{theorem}
    \(Y=\varprojlim^\star \{2, \gamma, L\}\cong \hat L\).
  \end{theorem}

  \begin{proof}
    Consider \(Y\) as a subspace of \(X=\varprojlim \{2, \gamma, L\}\) with
    the linear order described above. We claim that if \(\phi\) is the
    characteristic function for a subset of \(L\), then \(\phi\)
    is an order isomorphsim between \(\<Y,\leq\>\) and
    \(\<\hat L,\subseteq\>\).

    Let \(A\) be a left-closed subset of \(L\). Let \(\vect x(a)=1\) when
    \(a\in A\) and \(\vect x(a)=0\) otherwise. Then \(\vect x\in Y\) and
    \(\phi(\vect x)=A\).

    Let \(\vect x,\vect y\in Y\). If
    \(\phi(\vect x)=\phi(\vect y)=A\), then \(A\) is a
    left-closed set where \(\vect x(a)=\vect y(a)=1\) for \(a\in A\)
    and \(\vect x(a)=\vect y(a)=0\) otherwise, so \(\vect x=\vect y\).

    Finally let \(\vect x<\vect y\), so there exists \(a\in L\) with
    \(\vect x(a)=0\), \(\vect y(a)=1\). Then
    \(
      \phi(\vect x)
        \subseteq
      (\leftarrow,a)
        \subseteq
      \phi(\vect y)
    \). Thus \(\phi\) preserves order.
  \end{proof}

  \begin{corollary}
    \(
      \varprojlim^\star \{2, \gamma, \alpha\}
      \cong
      \alpha+1
    \)
    for every infinite limit or finite ordinal \(\alpha\).
  \end{corollary}

  \begin{proof}
    If \(\alpha\) is finite, then of course all (leftward) sets are
    closed and we get \(\hat\alpha=\check\alpha=\alpha+1\) for free.
    Otherwise, as observed previously \(\hat\alpha\)
    is homeomorphic to its usual compactification \(\alpha+1\) for
    limit ordinals.
  \end{proof}

  In fact, \(\hat\alpha=\alpha+1\setminus L(\alpha)\) where \(L(\alpha)\)
  is the colleciton of all limit ordinals less than \(\alpha\), which also
  shows \(\hat\alpha\cong\alpha\) for infinite successor ordinals \(\alpha\).

\newpage
\bibliographystyle{plain}
\bibliography{../bibliography}

\end{document}