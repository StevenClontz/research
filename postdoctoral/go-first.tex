\documentclass{article}
\usepackage{amsmath}
\usepackage{amsthm}
\usepackage{amssymb}
\newtheorem{theorem}{Theorem}
\newtheorem{lemma}[theorem]{Lemma}
\newtheorem{proposition}[theorem]{Proposition}
\theoremstyle{definition}
\newtheorem{definition}[theorem]{Definition}
\newcommand{\rest}{\upharpoonright}
\newcommand{\term}{\textbf}
\begin{document}

By convention, \(n=\{0,1,\dots,n-1\}\) for each natural number \(n\).

\begin{definition}
    A function \(F:n\to S\) is called a \term{formula} for a 
    set \(S\) of dice, where \(F(k)\) returns which die in the set \(S\)
    has a face with the number \(k<n\).
\end{definition}

For example, the formula \(F:4\to\{A,B\}\) defined by \(ABBA\)
represents an assignment of values to the faces of dice in the set \(\{A,B\}\).
In particular \(A\) has two sides \(\{0,3\}\) while die \(B\) has sides \(\{1,2\}\).

\begin{definition}
    Given a formula \(F:n\to S\) and \(T\subseteq S\), let \(F_T:m\to T\) be the 
    \term{subformula} obtained by removing the dice in \(S\setminus T\) from the sequence.
\end{definition}

So for \(F:7\to\{A,B,C\}\) defined by \(ABCACBA\), \(F_{\{A,C\}}\) is defined by
\(ACACA\).

\begin{definition}
    Given a formula \(F:n\to S\) and \(m\leq n\), let \(F\rest m:m\to S\) be the 
    \term{restriction formula} where \((F\rest m)(k)=F(k)\) for \(k<m\).
\end{definition}

So for \(F:7\to\{A,B,C\}\) defined by \(ABCACBA\), \(F\rest 4\) is defined by
\(ABCA\).

\begin{definition}
    A \term{sample} for a formula \(F:n\to S\) is a function \(f:S\to n\) such
    that \(F(f(A))=A\) for all \(A\in S\).
\end{definition}

\begin{definition}
    The \term{winner} of a given sample is \(A\in S\) such that
    \(f(A)=\max\{f(B):B\in S\}\).
\end{definition}

\begin{definition}
    A formula \(F:n\to S\) is called \term{go-first-fair} if each \(A\in S\)
    is the winner of an equal number of samples of \(F\).
\end{definition}

\begin{definition}etc.\end{definition}

\begin{proposition}
    Suppose \(F:n\to S\) is a permutation-fair (resp. place-fair, go-first-fair)
    formula. Then \(F_T\) is permutation-fair (resp. place-fair, go-first-fair)
    for all \(T\subseteq S\).
\end{proposition}

\begin{theorem}
    Suppose \(F:n\to S\cup\{X\}\) is a go-first-fair formula such that 
    for each \(m\leq n\) where \(f(m)=X\), 
    \((F\rest m)_S\) is permutation-fair.
    Then \(F\) is permutation-fair.
\end{theorem}

\begin{proof}
    Since go-first-fair implies permutation-fair in the base case \(|S|=0\),
    assume the theorem holds when \(|S|\leq k\), and let \(|S|=k+1\).
    
    For each \(T\subsetneq S\), we note that \(F_{T\cup\{X\}}\) is a go-first-fair 
    formula such that for each \(m\leq|F_{T\cup\{X\}}|\) where \(F_{T\cup\{X\}}(m)=X\), 
    \(F_{T\cup\{X\}}\rest m=F\rest m'\) for some \(m\leq m'< n\) and
    \(F(m')=X\). Thus \((F_{T\cup\{X\}}\rest m)_T=(F\rest m')_T=((F\rest m')_S)_T\)
    is permutation-fair, and since \(|T|\leq k\), \(F_{T\cup\{X\}}\) is permutation-fair.


\end{proof}
\end{document}
