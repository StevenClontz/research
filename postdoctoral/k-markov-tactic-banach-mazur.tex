\documentclass[11pt]{article}

\usepackage{amssymb}
\usepackage{amsfonts}
\usepackage{amsmath}
\usepackage{mathtools}
\usepackage{amsthm}

\usepackage[letterpaper,margin=1in]{geometry}

\usepackage{enumerate}

      \theoremstyle{plain}
      \newtheorem{theorem}{Theorem}
      \newtheorem{lemma}[theorem]{Lemma}
      \newtheorem{corollary}[theorem]{Corollary}
      \newtheorem{proposition}[theorem]{Proposition}
      \newtheorem{conjecture}[theorem]{Conjecture}
      \newtheorem{question}[theorem]{Question}
      \newtheorem{claim}[theorem]{Claim}

      \theoremstyle{definition}
      \newtheorem{definition}[theorem]{Definition}
      \newtheorem{example}[theorem]{Example}
      \newtheorem{observation}[theorem]{Observation}
      \newtheorem{game}[theorem]{Game}

      \theoremstyle{remark}
      \newtheorem{remark}[theorem]{Remark}

      \theoremstyle{plain}
      \newtheorem*{theorem*}{Theorem}
      \newtheorem*{lemma*}{Lemma}
      \newtheorem*{corollary*}{Corollary}
      \newtheorem*{proposition*}{Proposition}
      \newtheorem*{conjecture*}{Conjecture}
      \newtheorem*{question*}{Question}
      \newtheorem*{claim*}{Claim}

      \theoremstyle{definition}
      \newtheorem*{definition*}{Definition}
      \newtheorem*{example*}{Example}
      \newtheorem*{observation*}{Observation}
      \newtheorem*{game*}{Game}

      \theoremstyle{remark}
      \newtheorem*{remark*}{Remark}

\title{Tactics and Marks in Banach Mazur Games}
\author{Steven Clontz}

\usepackage{../clontzDefinitions}

\newcommand{\bmPoGame}[2]{BM_{po}(#1,#2)}

\begin{document}

\maketitle

  \section*{marks and tactics}

  My notes on Galvin/Telgarsky's Theorem 5 from \cite{MR831181}.

  \begin{definition}
    Let \(\mb P\) be partially ordered by \(\leq\).
    Let \(\mb P^{\downarrow}=\{f\in\mb P^\omega : f(n)\geq f(n+1)\}\).
    Then for \(f,g\in\mb P^\downarrow\), we say that \(f,g\) zip into each
    other if for all \(m<\omega\) there exists \(n<\omega\) such that
    \(f(m)\geq g(n)\) and \(g(m)\geq f(n)\).
  \end{definition}

  \begin{definition}
    \(\bmPoGame{\mb P}{W}\) is a game defined for all non-empty partial orders
    \(\mb P\) and all subsets \(W\subseteq\mb P^\downarrow\).
    During round \(0\), \(\plI\) chooses \(a_0\in\mb P\),
    and then \(\plII\) chooses \(b_0\leq a_0\); during around \(n+1\),
    \(\plI\) chooses \(a_{n+1}\leq b_n\), and then \(\plII\) chooses
    \(b_{n+1}\leq a_{n+1}\). \(\plII\) wins this game if
    \(\<a_0,a_1,\dots\>\in W\).
  \end{definition}

  \begin{theorem}
    Let \(W\subseteq\mb P^\downarrow\) be closed under zipping.
    \(\plII\markwin\bmPoGame{\mb P}{W}\) if and only if
    \(\plII\tactwin\bmPoGame{\mb P}{W}\).
  \end{theorem}

  \begin{proof}
    Let \(\tau(p,n+1)\) be a winning mark for \(\plII\), where \(p\)
    is the most recent move by \(\plI\) and \(n+1\)
    is the number of moves made by \(\plI\).
    Define \(\tau^0(p)=p\) and \(\tau^{n+1}(p)=\tau(\tau^n(p),n+1)\).
    Let \(\preceq\) well-order \(\mb P\).

    For \(p,q\in \mb P\), say \(p\geq_n q\) if there exist
    \(s_m(p)\in\mb P\) for \(m\leq n\)
    such that
    \[
      p
    \geq
      s_m(p)
    \geq
      \tau(s_m(p),n+1)
    \geq
      q
    .\]
    Note that \(p'\geq p\geq_n q\geq q'\) implies \(p'\geq_n q'\),
    and \(p\geq_n \tau^n(p)\).

    Say \(p\geq_\omega q\) whenever \(p\geq_n q\) for all \(n<\omega\).
    If \(p\geq_\omega l(p)\) for some \(l(p)\), then say \(p\) is long;
    otherwise call \(p\) short.

    For \(p\) short, let
    \[
      \mu(p)
        =
      \min_{\preceq}\{
        r\text{ short}
          :
        r\geq p
      \}
    \]
    and since \(\mu(p)\not\geq_n p\) for some \(n\), let
    \[
      N(p)
        =
      \min\{
        n<\omega
      :
        \mu(p)\not\geq_n p
      \}
    .\]
    Note that whenever \(\mu(p)=\mu(q)\) for \(p\geq_n q\),
    it follows that \(\mu(p)\geq_n q\) and therefore \(N(p)<N(q)\).

    We define
    \[
      \sigma( p)
        =
      \begin{cases}
        l(p) & p \text{ is long} \\
        \tau^{N(p)+1}( p) &  p \text{ is short}
     \end{cases}
    .\]

    Suppose \(\sigma\) is legally attacked by \(a\in\mb P^\omega\).
    For \(n\leq\omega\), if \(a(n)\) is long, then \(a(n)\geq_n l(a(n))\).
    Therefore,
    \[
      a(n)
        \geq
      s_n(a(n))
        \geq
      \tau(s_n(a(n)),n+1)
        \geq
      l(a(n))
        =
      \sigma(a(n))
        \geq
      a(n+1)
    .\]
    Thus if \(a(n)\) is long for \(n<\omega\), it follows that
    \(c\in\mb P^\downarrow\) defined by \(c(n)=s_n(a(n))\)
    is a legal attack against \(\tau\). Since \(\tau\) is winning,
    \(c\in W\), and since \(c\) zips into \(a\),
    \(a\in W\) as well.

    Otherwise, we may choose a final subsequence \(b\) of \(a\) such that
      \begin{itemize}
        \item \(b(n)\) is short for all \(n<\omega\),
              since \(a(m)\) short implies \(a(n+m)\) short for all
              \(n<\omega\).
        \item \(\mu(b(n))=\mu'\) is fixed for all \(n<\omega\), since
              there cannot be an infinite \(\preceq\)-decreasing sequence.
      \end{itemize}
    As a result,
    \[
      b(n)
        \geq_{N(b(n))}
      \tau^{N(b(n))+1}(b(n))
        =
      \sigma(b(n))
        \geq
      b(n+1)
    \]
    and therefore \(N(b(n))<N(b(n+1))\). In particular, \(N(b(n))\geq n\).

    Thus for \(n<\omega\),
    \[
      b(n)
        \geq
      \tau^{n}(b(n))
        \geq
      \tau(\tau^{n}(b(n)),n+1)
        \geq
      \tau^{N(b(n))+1}(b(n))
        =
      \sigma(b(n))
        \geq
      b(n+1)
    .\]
    As a result, \(c\in\mb P^\downarrow\) defined by \(c(n)=\tau^n(b(n))\)
    is a legal attack against the winning strategy \(\tau\). Therefore
    \(c\in W\), and since \(c\) zips into \(b\) and \(a\), we conclude
    \(a\in W\).
  \end{proof}

  \begin{observation}
    When \(\mb P=T(X)\setminus\{\emptyset\}\) is ordered by set-inclusion
    and \(W=\{U\in\mb P^\downarrow:\bigcap_{n<\omega}U(n)\not=\emptyset\}\),
    then \(\bmPoGame{\mb P}{W}\) is exactly the topological Banach Mazur game
    \(\bmGame{X}\). Note \(W\) is closed under zipping.
  \end{observation}

  \begin{corollary}
  \(\plII\markwin\bmGame{X}\) if and only if
  \(\plII\tactwin\bmGame{X}\).
  \end{corollary}


  \section*{2+ marks and tactics}

  And this stuff is based on section 4.5.1 of \cite{bartoszynski1993covering}.

    % \begin{proposition}
    %   Suppose that for every finite collection \(\{\mb P_i:i<n\}\)
    %   such that \(\mb P=\bigcup_{i<n}\mb P_i\),
    %   some \(\mb P_i\) contains two incompatible points.
    %   Then \(\mb P\) contains an infinite antichain.
    % \end{proposition}
    %
    % \begin{proof}
    %   Suppose \(a_i\in\mb P_i\) have been defined for \(i<n\) such that
    %   \(a_i\not\in\bigcup_{j<i}\mb P_j\) and
    %   each \(\mb P_i\) is pairwise-compatible. It follows that
    %   \(\mb P\not=\bigcup_{i<n}\mb P_i\), so choose
    %   \(a_n\not\in\bigcup_{i<n}\mb P_i\) and let \(\mb P_n\) be a maximal family
    %   containing \(a_n\) such that \(\mb P_n\) is pairwise-compatible.
    %
    %   It follows that \(A=\{a_n:n<\omega\}\) is an antichain, since for
    %   every pair \(a_i,a_j\in A\) with \(i<j\), \(a_j\not\in\mb P_i\), so
    %   \(a_j\) is not compatible with something in \(\mb P_i\)...
    % \end{proof}

  \begin{definition}
    Let \(f\in S^{\leq\omega}\). Then \(f\rest n\in S^n\) is defined by
    \((f\rest n)(i)=f(i)\). (\(f\rest n\) gives the first \(n\) terms of \(f\).)

    Let \(t\in S^{<\omega}\). Then \(t\last k\in S^k\) is defined by
    \((t\last k)(i)=t(i+|t|-k)\). (\(t\last k\) gives the last \(n\) terms
    of \(t\).)
  \end{definition}

\begin{definition}
  For every partial order \(\mb P\) and compatible \(p,q\in\mb P\),
  let \(p\wedge q\) satisfy \(p\wedge q\leq p,q\).
\end{definition}

  \begin{claim}
    \(\mb P\) contains no infinite antichains if and only if every
    antichain in \(\mb P\) is of size \(n\) or less for some \(n<\omega\).
  \end{claim}

  \begin{proof}
    MAYBE? Apparently true for \(\mb P=\tau\setminus\{\emptyset\}\) due to
    Lemma 2.10 of \cite{comfort_negrepontis_1982}.
  \end{proof}

  \begin{proposition}
    Let \(W\subseteq\mb P^\downarrow\) be closed under zipping.
    Suppose every antichain in \(\mb P\) is of size \(n<\omega\) or less,
    and \(\plII\win\bmPoGame{\mb P}{W}\). Then
    \(\plII\autowin\bmPoGame{\mb P}{W}\) (i.e. \(\plII\) wins every
    play of \(\bmPoGame{\mb P}{W}\), i.e. \(W=\mb P^\downarrow\)).
  \end{proposition}

  \begin{proof}
    First, let \(\{p_i:i<n\}\) be an antichain of size \(n<\omega\),
    then let \(\mb P_i\) be a maximal pairwise-compatible subset of \(\mb P\)
    containing \(p_i\). Note that if there existed
    \(q\in\mb P\setminus\bigcup_{i<n}\mb P_i\),
    \(q\) must be incompatible with some \(q_i\in\mb P_i\) for \(i<n\).
    Since \(p_i,q_i\in\mb P_i\), they are compatible, so let
    \(r_i=p_i\wedge q_i\). Since \(q\) is incompatible with \(q_i\) for \(i<n\),
    \(q\) is incompatible with \(r_i\) for \(i<n\). Since \(p_i\) is
    incompatible with \(p_j\) for \(i<j<n\), \(r_i\) is incompatible with
    \(r_j\) for \(i<j<n\). But that makes \(\{q\}\cup\{r_i:i<n\}\) an
    antichain of size \(n+1\), contradicting the assumption of the proposition.
    Thus \(\mb P=\bigcup_{i<n}\mb P_i\).

    We now show that if \(s\in\mb P_i^\downarrow\) for some \(i\),
    then \(s\in W\). Let \(\sigma\) be a
    winning strategy for \(\plII\) in \(\bmPoGame{\mb P}{W}\),
    and attack \(\sigma\) with \(q(0)=s(0)\wedge p_i\) and
    \(q(n+1)=s(n+1)\wedge\sigma(\<q(0),\dots,q(n)\>)\). Note that the choice
    of \(q(0)\) is valid as \(s(0),p_i\in\mb P_i\). Similarly,
    \(\sigma(\<q(0),\dots,q(n)\>)\leq q(0)\leq p_i\), so
    \(\sigma(\<q(0),\dots,q(n)\>)\) cannot be compatible with any \(p_j\)
    where \(j\not=i\). Thus \(s(n+1),\sigma(\<q(0),\dots,q(n)\>)\in\mb P_i\),
    making the choice of \(q(n+1)\) valid. Since \(\sigma\) is winning for
    \(\plII\), we see that \(q\in W\), and therefore \(s\in W\).

    Finally, consider any play of \(\bmPoGame{\mb P}{W}\). It must contain
    have a subsequence \(s\in\mb P_i^\downarrow\) for some \(i<n\),
    so \(s\in W\) and therefore
    the play is also in \(W\), securing a victory for \(\plII\).
  \end{proof}

  \begin{lemma}
    Let \(W\subseteq\mb P^\downarrow\) be closed under zipping.
    Suppose that for every \(p\in\mb P\), there exists an infinite
    antichain \(A_p=\{a_p(n):n<\omega\}\subseteq\{q\in\mb P:q\leq p\}\).
    Then \(\plII\kmarkwin{(k+2)}\bmPoGame{\mb P}{W}\) if and only if
    \(\plII\ktactwin{(k+2)}\bmPoGame{\mb P}{W}\).
  \end{lemma}

  \begin{proof}
    Let \(\sigma\) witness \(\plII\kmarkwin{(k+2)}\bmPoGame{\mb P}{W}\).
    Define \(\tau(t)=\sigma(\<a_{t(0)}(0)\>,1)\) for \(t\in\mb P^1\).
    Since \(\tau(t)=\sigma(\<a_{t(0)}(0)\>,1)\leq a_{t(0)}(0)\leq t(0)\),
    this is a legal move.

    Consider \(t\in\mb P^{j+2}\) for \(j\leq k\).
    If there exists \(l_t<\omega\) such that
    \(t(j+1)\leq a_{t(j)}(l_t+j)\),
    define \(t'\in\mb P^{j+2}\) by \(t'(i)=a_{t(i)}(l_t+i)\) and let
    \(\tau(t)=\sigma(t',l_t+|t|)\). Note that since
    \[
      \tau(t)
        =
      \sigma(t',l_t+|t|)
        \leq
      t'(j+1)
        =
      a_{t(j+1)}(l_t+j+1)
        \leq
      t(j+1)
    \]
    this is a legal move. (If \(l_t\) failed to exist,
    we could arbitrarily let, say,
    \(\tau(t)=t(|t|-1)\); as we will see, this case will never occur
    for any legal attack against \(\tau\).)

    Let \(f\) be a legal attack against \(\tau\). The intuition of the
    following proof is simple: by construction, \(l_t\) will produce
    the number of \(\plI\)'s moves forgotten by \(\plII\)'s \((k+2)\)-tactic,
    allowing the \((k+2)\)-tactic to deduce the round number and thus exploit
    the winning \((k+2)\)-mark.

    We may quickly verify
    that \(l_{f\rest 2}=0\) since
    \[
      (f\rest 2)(1)
        =
      f(1)
        \leq
      \tau(f\rest 1)
        =
      \sigma(\<a_{f(0)}(0)\>,1)
        \leq
      a_{f(0)}(0)
        =
      a_{(f\rest 2)(0)}(0+0)
    .\]
    We claim in general that \(l_{f\rest(j+2)}=0\) for \(j\leq k\).
    Assuming \(l_{f\rest(j+2)}=0\) for \(j<k\),
    \[
      (f\rest(j+3))(j+2)
        =
      f(j+2)
        \leq
      \tau(f\rest(j+2))
        =
      \sigma(f\rest(j+2)',0+j+2)
        \leq
      f\rest(j+2)'(j+1)
    \]
    \[
        =
      a_{(f\rest(j+2))(j+1)}(j+1)
        =
      a_{(f\rest(j+3))(j+1)}(0+(j+1))
    \]
    proving \(l_{f\rest(j+3)}=0\).

    Now we show that \(l_{f\rest(j+2)\last(k+2)}=j-k\) for \(j\geq k\).
    We've just shown that this is true for our base case \(j=k\).
    Now assuming \(l_{f\rest(j+2)\last(k+2)}=j-k\), we show...
  \end{proof}

  \bibliographystyle{plain}
  \bibliography{../bibliography}


\end{document}
